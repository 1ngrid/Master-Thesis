\documentclass[11pt,english,a4paper]{article}
\usepackage[utf8]{inputenc}
\usepackage[T1]{fontenc}
\usepackage{graphicx, url}
\usepackage[bottom]{footmisc}
\usepackage{babel,textcomp,csquotes,varioref,graphicx}
\usepackage{fancyvrb}

%\usepackage{babel,textcomp,csquotes,ifimasterforside,varioref,graphicx}
%\usepackage[backend=bibtex,style=numeric-comp]{biblatex}
%\usepackage{wrapfig}


\DefineVerbatimEnvironment{code}{Verbatim}{fontsize=\small}

%\newcommand{\textt}[1]{%
 % \begingroup
 % \ttfamily
 % \begingroup\lccode`~=`/\lowercase{\endgroup\def~}{/\discretionary{}{}{}}%
 % \begingroup\lccode`~=`[\lowercase{\endgroup\def~}{[\discretionary{}{}{}}%
 % \begingroup\lccode`~=`.\lowercase{\endgroup\def~}{.\discretionary{}{}{}}%
 % \begingroup\lccode`~=`-\lowercase{\endgroup\def~}{-\discretionary{}{}{}}%
 % \catcode`/=\active\catcode`[=\active\catcode`.=\active
 % \scantokens{#1\noexpand}%
 % \endgroup
%}

%\newcommand{\defpill}[3]{%
%  \expandafter\newcommand\csname blue#1\endcsname{#2}%
% \expandafter\newcommand\csname red#1\endcsname{#3}%
%}

%\begin{document}
%Let me define a pill (with apologies to chemists).%
%  \defpill{aspirin}{$A_2Sp_3$}{$A_2Sp_5$}
%This should not print anything.

%But I should now be able to use my new commands, for example the one
%giving the formula of the red aspirin: \redaspirin. Or the blue one:
%\blueaspirin.

\newcommand{\enwikicatlink}{\texttt{enwiki-latest-categorylinks.sql.gz} }
\newcommand{\enwikipage}{\texttt{enwiki-latest-page.sql.gz} }
\newcommand{\enwikicategory}{\texttt{enwiki-latest-category.sql.gz} }
\newcommand{\catlinkprogram}{\texttt{Split\_catlink.py}}
\newcommand{\catgraphbuilderprogram}{\texttt{Categorygraph\_builder.py}}

\title{Parsing the Wikipedia Dumps}
\author{Ingrid Grønlie Guren}

\begin{document}

\maketitle

There are two ways of accessing Wikipedia's ecyclopedic information. The first way is to look up runtime as most users do when they are looking for information. The other way is more common when the information is used by other programs and

%s, which means that the program access Wikipedia's 

The other, and most common way is to download database dumps from Wikipedia. All Wikipedia articles, images and categories are stored in a database which are accessed when a user are searching for an article online. A database dump is therefore a backup of the database which are usually stored in the case of some data is lost. \footnote{TODO: reference: en.wikipedia.ort/wiki/Database\_dump} This backup is available for anyone interested at \emph{TODO: insert link}\footnote{TODO: insert reference?}. 




A database dump is defined as the table structures which is used to get the information to load

For our purpose there are some dumps that are relevant: 


I have chosen to work on the English Wikipedia, which is the largest database in with *** articles and *** pages. 

The relevant files are: 
\begin{itemize}
\item \enwikicatlink
\item \enwikipage
\item \enwikicategory
\end{itemize}

All of these files are compressed sql-files, which means that they represent files to put information into a SQL-database. Each of the files can be used to build up a database table with insert-statements, so all the information is stored in the table. 

\enwikicatlink describes all the links between categories, which describes two different types of relationships in Wikipedia. The first relationship is between a Wikipedia article and a category, i.e the category *** points to the article. The other relationship is between two categories which means that one of the categories is a subcategory of the other category. 

The file contains the table "categorylinks". Since the file is quite large (1.5GB compressed), it is desirable to split the file into two files; files containing information of the relationships between categories and files that contain information about the relationship between categories and pages. 

This was solved by creation the program. The program takes the \enwikicatlink as input, and goes through each INSERT-statement in the file. Since all INSERT statements contains information about many relationships, the statements are split to represent one relationship at the time. Then the statement is sorted into "Link between two categories" or "Link between a category and a page" depending on the output of the relationship described in the statement. 

Some of the information about the relationships between categories are not relevant for this problem, for instance information about hidden categories like "Unprinthworthy categories". This categories are removed in the process to reduce the number of category elements considered in the later programs. 

Wikipedia also contains lots of information about redirecting between categories and pages, for instance are article names in plural redirected to the article name in singular.  This information is also stored in \enwikicatlink and is sorted out during the process to be considered later in the process where it is desirable to redirect in the same way, but still not relevant in the early steps of the programming. 

%Category graph builder
One of the output files of the \catlinkprogram is a file containing all relationships between categories. The next step is to sort this information so that a category knows all its subcategories. The program \catgraphbuilderprogram takes the category information as output and creates a structure to represent the information. The output of the program is a file where all parents to a subcategory is stored. %This program should maybe consider loops as well
The program sorts all the categories and output a file where all subcategories of a category is listed under the name of their parent category.

\enwikicatlink also contains some shortcuts for saving space when information about the categories are inserted into the database. An example of such a statement is \footnote{TODO: insert reference: part of the insertion statement from the file \enwikicatlink}: 

\begin{code}
1517681,'fictional_birds','ducks\nfictional ducks','2014-10-26 03:30:11',
'ducks','uppercase','subcat' 
\end{code}

This part of the insertion statements means that the category \emph{fictional ducks} is both a subcategory of the category \emph{Fictional birds} and the category \emph{ducks}. This means that this statement has to result in the following information: 
\begin{code}
Fictional birds: 
* fictional ducks

ducks:
* fictional ducks
\end{code}

One of the problem in the results of the program is that there are potential for loops within the structure. This is because a category may be subcategory of a category, but also the parent category of the same category's parent. This means that the whole structure cannot be represented as a three, but is rather a graph where the connected categories are linked together. 

The other output file of \catlinkprogram is a file where the each line represents an article its immediately closest categories, these categories are the same as those represented at the bottom of the article page. 

It is desirable to get each article's full path for our problem. The next program made is therefore a program that takes creates an output where all articles and their immediate closest categories are stored. Here is also some cleaning done so that all categories which contains the words \emph{Wikipedia} or \emph{category} are removed since these are part of the hidden structure for sorting and not relevant for our task. 

To create the full path of each article are there 

the category graph and all the articles with it's 


%Article builder
After all category links are split into list describing relationships between categories and relationships between articles and their subcategories 


\end{document}