%This was solved by creation the program. The program takes the \path{enwiki-latest-categorylinks.sql.gz} as input, and goes through each INSERT-statement in the file. 

%Since all INSERT statements contains information about many relationships, the statements are split to represent one relationship at the time. Then the statement is sorted into "Link between two categories" or "Link between a category and a page" depending on the output of the relationship described in the statement. 

%Some of the information about the relationships between categories are not relevant for this problem, for instance information about hidden categories like "Unprinthworthy categories". This categories are removed in the process to reduce the number of category elements considered in the later programs. 

%Wikipedia also contains lots of information about redirecting between categories and pages, for instance are article names in plural redirected to the article name in singular.  This information is also stored in \enwikicatlink and is sorted out during the process to be considered later in the process where it is desirable to redirect in the same way, but still not relevant in the early steps of the programming. 

%Category graph builder
%One of the output files of the \catlinkprogram is a file containing all relationships between categories. The next %step is to sort this information so that a category knows all its subcategories. The program \catgraphbuilderprogram takes the category information as output and creates a structure to represent the information. The output of the program is a file where all parents to a subcategory is stored. %This program should maybe consider loops as well
%The program sorts all the categories and output a file where all subcategories of a category is listed under the name of their parent category.

%\enwikicatlink also contains some shortcuts for saving space when 


%One of the problem in the results of the program is that there are potential for loops within the structure. This is because a category may be subcategory of a category, but also the parent category of the same category's parent. This means that the whole structure cannot be represented as a three, but is rather a graph where the connected categories are linked together. 

%The other output file of \catlinkprogram is a file where the each line represents an article its immediately closest categories, these categories are the same as those represented at the bottom of the article page. 

%It is desirable to get each article's full path for our problem. The next program made is therefore a program that takes creates an output where all articles and their immediate closest categories are stored. Here is also some cleaning done so that all categories which contains the words \emph{Wikipedia} or \emph{category} are removed since these are part of the hidden structure for sorting and not relevant for our task. 


%Article builder
%After all category links are split into list describing relationships between categories and relationships between articles and their subcategories 
