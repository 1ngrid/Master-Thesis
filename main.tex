\documentclass[a4paper,english]{book}
\usepackage[utf8]{inputenc}         % or whatever you use

\usepackage[T1]{fontenc, url} %andreas
%\usepackage[T1]{url}
\urlstyle{sf}
\usepackage{ifipackages/ifikompendiumforside}
\usepackage{textcomp} %andreas slutt

\usepackage{graphicx, wrapfig}
%\usepackage{caption}
\usepackage{booktabs}
\usepackage[labelfont=it,textfont=it]{caption}
\usepackage{subcaption}
\usepackage{amsmath,amsfonts,amssymb}
\usepackage[hypcap]{caption}
\usepackage{xcolor,colortbl}
\usepackage{mathtools}
\usepackage{multirow}
\usepackage{array}
\usepackage{fancyvrb, babel, csquotes, varioref, graphicx}
%\usepackage{fancyvrb, csquotes, varioref, graphicx}
\usepackage{float}
%\floatstyle{plaintop}
\restylefloat{table}
\usepackage[section]{placeins}
\usepackage{tabularx}
\usepackage{listings}
%\usepackage{subcaption}
\usepackage{comment}
%\usepackage[caption=false]{subfig}
\usepackage{hyperref, enumitem}

%\usepackage[romanian]{babel}
\usepackage{combelow}
\usepackage{layout}
\usepackage{hhline}
\usepackage{dcolumn}
%\usepackage[sorting=none, backend=bibtex]{biblatex}
\usepackage[style=numeric,backend=bibtex]{biblatex}
%\usepackage{flushright}
%\usepackage{multibib}
%\newcites{wikipedia}{Wikipedia References}%


\addbibresource{mybib.bib}

\makeatletter
%\renewcommand{\@chapapp}{}% Not necessary...
\newenvironment{chapquote}[2][2.5em]
  {\setlength{\@tempdima}{#1}%
   \def\chapquote@author{#2}%
   \parshape 1 \@tempdima \dimexpr\textwidth-2\@tempdima\relax%
   \itshape}
  {\par\normalfont\hfill--\ \chapquote@author\hspace*{\@tempdima}\par\bigskip}
\makeatother


%\usepackage{amsmath,amssymb}
\makeatletter
\newsavebox\myboxA
\newsavebox\myboxB
\newlength\mylenA

\newcommand*\xoverline[2][0.75]{%
    \sbox{\myboxA}{$\m@th#2$}%
    \setbox\myboxB\null% Phantom box
    \ht\myboxB=\ht\myboxA%
    \dp\myboxB=\dp\myboxA%
    \wd\myboxB=#1\wd\myboxA% Scale phantom
    \sbox\myboxB{$\m@th\overline{\copy\myboxB}$}%  Overlined phantom
    \setlength\mylenA{\the\wd\myboxA}%   calc width diff
    \addtolength\mylenA{-\the\wd\myboxB}%
    \ifdim\wd\myboxB<\wd\myboxA%
       \rlap{\hskip 0.5\mylenA\usebox\myboxB}{\usebox\myboxA}%
    \else
        \hskip -0.5\mylenA\rlap{\usebox\myboxA}{\hskip 0.5\mylenA\usebox\myboxB}%
    \fi}
\makeatother

%\usepackage[margin=0.5in]{geometry}
\hypersetup{
    linktocpage,
    colorlinks, 
    citecolor=black, 
    filecolor=black,
    linkcolor=black,
    urlcolor=black,
    linktoc=all
    }

%\floatsetup[table]{capposition=top}
\DefineVerbatimEnvironment{code}{Verbatim}{fontsize=\small}

% The programs I have made
\newcommand{\catlinkprogram}{\texttt{Split\_catlink.py} }
\newcommand{\catgraphbuilderprogram}{\texttt{Categorygraph\_builder.py} }
\newcommand{\artbuildprogram}{\texttt{}}

\newcommand\Chapter[2]{
  \chapter[#1: {\itshape#2}]{#1\\[2ex]\Large\itshape#2}
}

\renewcommand{\lstlistingname}{Code}
\lstdefinestyle{customasm}{
  basicstyle=\ttfamily,
  belowcaptionskip=1\baselineskip,
  frame=trLb}

\lstset{ captionpos=b, style=customasm, xleftmargin=\parindent,breaklines=true,}
% aboveskip=20pt,belowskip=20pt,  
% own environment for code
%\newcounter{codecnt}
%\labelformat{codecnt}{Code~#1}


\newcommand*\justify{%
  \fontdimen2\font=0.4em% interword space
  \fontdimen3\font=0.2em% interword stretch
  \fontdimen4\font=0.1em% interword shrink
  \fontdimen7\font=0.1em% extra space
  \hyphenchar\font=`\-% allowing hyphenation
}

\newcommand{\enwikipageprops}{\texttt{\justify{enwiki-latest-page\_props.sql.gz}}}
\newcommand{\enwikipage}{\texttt{\justify{enwiki-latest-page.sql.gz}}}
\newcommand{\enwikicatlink}{\texttt{\justify{enwiki-latest-categorylinks.sql.gz}}}
\newcommand{\allhidden}{\texttt{\justify{All\_hidden\_categories.txt.gz}}}
\newcommand{\enwikiredirect}{\texttt{\justify{enwiki-latest-redirect.sql.gz}}}
\newcommand{\outputredirect}{\texttt{\justify{output-redirect-titles.txt.gz}}}
\newcommand{\subcat}{\texttt{\justify{Sub-categories.txt.gz}}}
\newcommand{\pagecat}{\texttt{\justify{Page-categories.txt.gz}}}
\newcommand{\subcatlink}{\texttt{\justify{Subcat\_links.txt.gz}}}
\newcommand{\catinfo}{\texttt{\justify{category-info.txt}}}
\newcommand{\artinfo}{\texttt{\justify{article-info.txt.gz}}}
\newcommand{\enwikicategory}{\texttt{\justify{enwiki-latest-category.sql.gz}}}
\newcommand{\enwikilanglinks}{\texttt{\justify{enwiki-latest-langlinks.sql.gz}}}

\newcommand{\enwikidatabasedumps}{\texttt{\justify http://dumps.wikimedia.org/enwiki}}
\newcommand{\ingridthesis}{\texttt{\justify https://github.com/ingridguren/Master-Thesis-2015}}




\title{Content Categorization for Contextual Advertising Using Wikipedia}
\author{Ingrid Grønlie Guren}

%\bibliography{mybib}

\begin{document}
%\ififorside{}

\ififorside
\frontmatter{}
\maketitle{}


\chapter*{Abstract}
Automatic categorization of content is an important %piece of 
functionality in online advertising and automated content recommendations, both for ensuring contextual relevancy of placements and for building up behavioral profiles for users that consume the content. Within the advertising domain, the taxonomy tree that content is classified into is defined with some commercial application in mind to somehow reflect the advertising platform’s ad inventory. The nature of the ad inventory and the language of the content might vary across brokers (i.e., the operator of the advertising platform), so it was of interest to develop a system that can easily bootstrap the development of a well-working classifier. 

We developed a dictionary-based classifier based on titles from Wikipedia articles where the titles represent entries in the dictionary. The idea of the dictionary-based classifier is so simple that it can be understood by users of the program also those who lack technical experience. Further, it has the advantage that its users easily can expand the dictionary with desirable words for specific advertisement purposes. The process of creating the classifier includes a processing of all Wikipedia article titles to a form more likely to occur in documents, before each entry is graded to their most describing Wikipedia category path. The Wikipedia category paths are further mapped to categories based on the taxonomy of Interactive Advertising Bureau (IAB), which are categories relevant for advertising. The results of this process is a dictionary with entries connected to categories from the taxonomy, and forms the base of our classifier. Finally, we explored the possibilities of using Wikipedia's internal links to translate the English classifier's dictionary to a Norwegian dictionary.

The evaluation of the classifier was performed on \texttt{rappler.com} for the English classifier and \texttt{adressa.no} for the Norwegian classifier. The results of the classifiers were compared with a class tag within the url structure of published articles, and we could see that the classifiers were able to correctly categorize most articles. 
%The results show that the classifier is able to correctly categorize most articles. 
%the vast majority of
However, there is room for further improvement of the classifier in order to achieve higher evaluation scores. This is partly because our dictionary-based classifier is a one-to-many classifier, while we compare the results to a one-to-one classification.

Overall, we found that we are able to create a varied and thorough dictionary by just exploring %the utilising
the titles of Wikipedia articles, and that the classifier gives a good indication of the content of articles. 


\begin{comment}

form??? 

"explain" -> tanken bak er enkel. 


Brokers are often not very technical and by experience will have severe problems developing training sets or otherwise contribute to the process, so any required involvement from their side has to be relatively simple. Furthermore, it is a practical requirement that the classifier can “explain” its classification to the broker in some way, and that the broker can have a simple way to manually override or influence the classification of known problem cases.

We explore the use of Wikipedia to develop a simple dictionary-based classifier. A dictionary-based classifier offers a simple way to “explain” the classification, and allows the classification vocabulary (i.e., the dictionary entries) to be easily edited. Wikipedia exists in a large number of languages, has a large number of article keywords covering most domains, and explicitly associates article names with categories. We describe a set of tools that automate the process of building up dictionaries that map Wikipedia keywords (or cleansed versions thereof) into Wikipedia categories (or modified versions thereof). Creating a mapping that maps Wikipedia categories into the broker’s custom taxonomy tree is a relatively straightforward task that brokers (or people working on their behalf) are assumed capable of. We will here use predefined a taxonomy, the one provided by IAB, as a working example of such a custom taxonomy tree.

Given such a classifier, we describe an experiment using a real advertising platform to validate its use in real life using real data. We also provide a brief overview of related work described in the literature.

\end{comment}
\chapter*{Acknowledgements}
This study has been a project for Xcense (\url{http://www.cxense.com}) and the Department of Informatics at the University of Oslo, and was started in the Spring semester 2014 and finished ** 2015. 

I would like to express my gratitude to my supervisor Aleksander Øhrn for all his incredibly important feedback and for all the advices he has given me through the process. 
%all advices he has given me and his scrunity of my thesis. 
I would also like to thank Gisle Ytrestøl for all his help, including all the quick responses on email, long discussions about implementations and his never-ending support and optimism about the project. 

Finally I would like to thank my family and study friends for all help, comments and discussions, and most important for supporting me every day. I could never have done this without any of you. 
  
\pagenumbering{roman}
\tableofcontents{}
\listoffigures{}
\listoftables{}

\pagenumbering{arabic}
\mainmatter{}
%\pagenumbering{arabic}
\chapter{Introduction}
The goal of this study has been to develop an automatic content categorization which could take any article as input, and determine the most likely category based on its content. Our approach of determining the most likely category is by creating a dictionary-based classifier from Wikipedia, where titles of Wikipedia articles are used as entries, and the entries are connected to one or more suitable categories describing the content of the Wikipedia article. 


%which corresponds to one or more suitable categories. 

% The output categories can e determined by the user to fit different user settings. 

\section{The Project}
It is believed that computers may perform better in many settings if they are able to determine the meaning of a text. There are different ways of achieving this result, and one of them is by automatic content categorization. Automatic content categorization is a process where the text is categorized to the most describing category from a set of desirable categories. There are various ways of performing automatic content categorization. This project focuses on categorizing text based on which keywords occur in the text and these keywords' categories. 
%The approach for this consists of finding relevant keywords and desirable categories, create a mapping between the keywords and the categories, and then determine the most likely category for the text based the keyword occurrences. 

\begin{figure}[h]
\centering
\includegraphics[width=0.65\textwidth]{Chapters/Introduction/keywordstocategories}
\caption{Illustration of the mapping between keywords and categories.}
\label{fig:keywordstocategories}
\end{figure}


Creating this automatic content categorization consists of three main steps. The first step is to create a list of keywords and a set of desirable categories for the categorization process. For this project have we chosen the keywords to be titles of Wikipedia articles, and taxonomy from  \emph{Interactive Advertising Bureau} (IAB) taxonomy as the set of desirable categories. 
%The Wikipedia article titles need be processed before they can be used as keywords and some changes might be necessary to create a 
Both Wikipedia article titles and IAB's taxonomy need to be processed before they are suitable as keywords and category set. 

The next step is creating a mapping between the keywords and the categories (see figure \ref{fig:keywordstocategories}). This step takes advantage of the underlying structure of Wikipedia to find the meaning of the Wikipedia article so that the keywords map to categories describing their content.

The final step of the categorization process is determining the category of any given text. Figure \ref{fig:categorizetext} shows this process, where all keywords are found in the given text and the text's category is determined from the keywords' categories. It is also essential to use some technique for finding keywords in the text, and this project has used software from Cxense for this step. 

\begin{comment}

Example of a categorization could be the simple collection of the two texts: $t_{1}$ = \emph{Zlatan and Messi play soccer} and $t_{2}$ = \emph{The sun in yellow}.  \emph{soccer}, \emph{Messi} and \emph{Zlatan} are keywords mapping to the category \emph{Sports/soccer}

As example should an article which contains the words \emph{soccer}, \emph{Messi} and \emph{Zlatan} should probably be categorized as \emph{Sports/Soccer} if these are available keywords mapped to this category.  


\end{comment}
%nderstanding text is important for 
%utomatic content categorization is useful for determining the meaning of a text which is useful in many settings. 
%Given a set of desirable categories it
%To be able to categorize articles, it is 
%Thus our overall goal is defined as creating a classifier that maps keywords from a predefined keyword list and to one or more pre-defined categories. The automatic categorization will include both the creation of the predefined keyword list and the mapping function, which are both essential for categorizing collections of texts based on their content. 

%Creating this automatic content categorization consists of *** steps. the first step is to find all keywords 


\begin{figure}[h]
\centering
\includegraphics[width=\textwidth]{Chapters/Introduction/categorizetext}
\caption{Simplified illustration of the categorization process.}
\label{fig:categorizetext}
\end{figure}

\begin{comment}
\begin{figure}[H]
\centering
\includegraphics[width=1\textwidth]{Chapters/Introduction/classification_process.jpeg}
\caption{A simplified illustration of the categorization process.}
\label{fig:classification_process}
\end{figure}
\end{comment}
%This classifier will be used as part of an automatic categorization process. 
%Our overall goal is therefore to make an automatic categorization that have a predefined keyword list and start by creating a mapping from each keyword to a category from another predefined list. Where the category of a text can be defined from the keywords found in the text. 

\section{An Overview of Challenges}
This project has encountered some challenges within different fields. Some of the challenges were solved better than others. 

\subsubsection{Representing the Structure}
The structure of Wikipedia is found in multiple files containing lots of information. The underlying structure is quite complex, and is poorly documented from a developer's point of view. The first challenge encountered was deciding the information required for representing the structure and determining a suitable structure for the representation.



\begin{comment}
The main challenges encountered had to do with Wikipedia, mainly because the encyclopedia is maintained by thousands of volunteers and is poorly documented from a developer's point of view. The structure of Wikipedia is complex and the size makes it hard to check if the results found are correct. 

\end{comment}

%which also leads to a complicated and complex structure. 

\subsubsection{Encoding}
Wikipedia is available in multiple languages and is written by volunteers from all over the world. This makes Wikipedia a multilingual encyclopedia with knowledge available from everywhere and it is possible for experts form various fields and from different parts of the world to contribute with knowledge. There are both advantages and disadvantages with a multilingual encyclopedia. One of the disadvantages is that users might write with different encoding (e.g. \emph{utf8}, \emph{ascii} or \emph{unicode}). Problems occur when going through all the names of Wikipedia categories and Wikipedia article titles because titles written in different encoding might not be viewed as identical by the computer. 

An example of a category name which leads to trouble with encoding is  \emph{Communes in Cara\cb{s}-Severin County}, which is either written with the letter \emph{\cb{s}} \cite{swithcomma} or \c{s} \cite{swithcedilla}. These letters are examples of characters that makes matching category names difficult, because \emph{Communes in Cara\cb{s}-Severin County} and \emph{Communes in Cara\c{s}-Severin County} will not be equal to the computer even though it is clear to most users that they should be the same. 

This problem was partly solved by changing all category names and article titles to the same encoding by transforming all text to \emph{utf-8}, including escape of \emph{unicode} characters with \cite{unidecode}. This solved most of the problem, but some category names did not become equal even though most humans would consider them equal. A total of 10800 categoroies was not able to be matched out of \textbf{INSERT NUMBER}. These categories represent a very small part of all categories  (equivalent to 0.9\%), and were therefore disregarded. 


\subsubsection{Disambiguation}
Antoher problem encountered is disambiguation. Wikipedia contains many titles that could have various meanings (see figure \ref{fig:disambiguation_example}). This means that the titles is ambiguous and leads to the common problem in natural language processing: disambiguation \cite{wiki:disambiguation}. A complete section (\ref{sec:disambiguation}) is dedicated to different solutions to this specific problem. 
%others work within in the topic see \ref{sec:disambiguation}

\begin{figure}[h]
\centering
\includegraphics[width=\textwidth]{Chapters/Introduction/Ice_cream_disambiguation}
\caption{Example of disambiguation in Wikipedia \cite{wiki:icecreamdisambiguation}.}
\label{fig:disambiguation_example}
\end{figure}
%\section{Different Approaches}

Here I'm gonna write something about the approaches I have chosen for my project - what solutions I chose?
\section{Thesis Outline}
Chapter 2 is dedicated to an introduction of the background needed for understanding the purpose of the project and the methods used in the implementation. The background covers the purpose and definition of content analysis, categorization, IAB, Wikipedia and its structure, and finally the challenges of the project. 

After describing the challenges, the next natural step is to look at previous work and research which is given in chapter 3, along with discussing if these findings can be used in our project. 

Chapter 4 is focusing on the implementation of the project and explains the approach for solving the problem. The chapter describes the process of finding full path of all Wikpedia articles from Wikipedia categories, how the paths are graded to find the most relevant paths for each article and the mapping process to the desirable output category set. 

Chapter 5 shows the results and improvements made to the implementation, which led to  
\chapter{Background Materials}
\label{sec:background_materials}

\section{Automatic content analysis}

\subsection{What is Content Analysis?}
Content analysis is the task of analysing and understanding collections of texts, in other words finding out what a text "is about". The task can be performed by both humans and computers, where both of the approaches have their advantages and disadvantages.

The concept of manual content analysis is easy, where the task is split into first reading and understanding the text, and then summarizing the content of the text and/or categorize it into suitable categories. For instance an article about \emph{Ole-Johan Dahl} (the famous Norwegian computer scientist \cite{Olejohandahleng}) would probably be summarized as an article about a famous Norwegian computer scientist and might be categorized under the category "Norwegian computer scientists" if this category is present.  
%as an article about authors of children books and Swedish people. 
There are two main disadvantages of manual content analysis which make manual content analysis impossible on large copllections of texts. The first disadvantage is that the task is time consuming, i.e. it takes time to read and understand an article for humans. The second disadvantage is that manual content analysis requires resources that might be expensive, for instance experts needed for understanding the content of an article if the article is about something beyond common knowledge.

%because some articles needs experts for understanding the content. 
%first has to be read and understood and then we could summarize the content of the text or categorize it under relevant topics.
Automatic content analysis is based on a different approach; instead of reading and understanding the text, the machine looks for known words or phrases and uses these to determine the meaning of the text. This approach has disadvantages as well; computers lack commonsense knowledge usually known to ordinary humans, for instance physical description or function of objects. Another disadvantage with automatic content analysis is dealing with disambiguation. Some words have more than one meaning where the meaning is usually found from the other words in the sentence. The task of determining the true meaning of a word is a difficult process which becomes harder if the sentence is complex. 

The easiest way to describe the meaning of the text is to group texts with similar content together, in other word categorize the texts. 

%Some of the advantages with automatic content analysis are that miss


%There are different ways to perform automatic content analysis, our approach is to find the most likely category for texts given as input by first categorizing all articles from Wikipedia. 

%involves using categorization of articles from Wikipedia to determine 
\subsection{Content Analysis in Advertising}
Automatic content analysis can be found useful in many different settings, but the 
%There are many areas where content analysis (the task of understanding the content of the text) is useful, but the 
two most dominating areas are advertising and improvement of user experience. The focus for this project is to improve advertising, hence, this is our domain.
%Content analysis is useful for understanding texts, and is often used in two domains; advertising and improving user experience. 

Advertising is the main income of most online companies that provide free services. The alternative to advertising is to charge users a fee before they are allowed to use the services. The online market is very competitive and most users expect everything on the Internet to be free. Thus, the most common approach is to provide the services for free, but earn money on advertising instead. 

There are two different roles within online advertising. The advertiser is a person or company that has advertisements available for display, while the publisher is the role that integrates the advertisements on the web page or choose what advertisements the users will see \cite{wiki:onlineadvertising}. Advertisements on frequently visited web pages are usually more expensive than advertisements on less common web pages. Advertisers are also more willing to pay more for advertisements directed towards users with a higher potential of buying the products. 

%free services with advertisement is to sell access to to their services, 

%but this is in many cases not possible because the online market is very competitive and many users expect everything on the Internet to be free. 

%difficult in many cases because of how competitive the market is and the expectations of a free Internet. 
%Online advertising is a growing market and is the main income of many online companies. 
%Online companies have two ways of earning money, they could either let the user pay to use the services or earn money on advertisement. Most web pages choose the second option because the market is very competitive and many web users expects the web pages to be free. %A newspaper might loose readers when trying to charge them. 

%Hence the companies have to earn money without charging the user for their services, and in most cases  this leads to advertisement. 
There are different approaches of paying for advertisements on web pages. The most common ones are Pay-Per-Click (PPC) \cite{wiki:ppc} and Cost-Per-Mille (CPM) \cite{wiki:ppc}. PPC is an approach where the advertiser pays per click on the advertisements, while CPM means that the advertiser pays for showing the advertisement to thousand viewers. 
%Both of these are more profitable if the advertisements are shown to interested users.
%, while PPC means that an advertisement is shown and the advertiser has to pay for each time a user clicks on the advertisement. 
Both of the techniques are more valuable for all parts if the advertisements are shown to people that are interested in the products and more likely to buy the product. The advertiser has a greater chance of getting customers if the advertisement is shown to the right crowd and the publisher can charge a higher price for the advertisement if the advertiser is more pleased with the result of the advertisements.

This means that the advertiser needs to know the users and their interests. Content analysis can be part of building up a profile of the user, since knowing the content of the text viewed to the user can provide information of the interests of the users. 

% TODO: write something about improving user experience
\section{Categorization}
The process of grouping collections of text into categories can, as already mentioned, be done by either humans or computers. Computer categorization is part of machine learning and is defined as the technique of teaching the machine how to behave. The process is about finding patterns, for instance recognize similarities between input, or to decide rules for the machine so that is is able to predict the category for an input. The goal for the process is to make the behavior of the machine so optimal as possible. Categorization with machine learning can be split into further types; statistical classification which is a \textit{supervised} learning process, and clustering which can be performed as both an \textit{supervised} and an \textit{unsupervised} learning process. Supervised learning is a technique where the machine is given a training data set, where the set contains the correct output in addition to the input we want to classify. The classifier use this data to learn the machine how to behave, also called training the classifier. Unsupervised learning, on the other hand, is the task of trying to find a hidden structure in unlabeled data. The difference is that unlabeled data gives no feedback to the classifier. 
%the data is unlabeled is no feedback sent to the classifier. 
The classifier will therefore not know if the result is correct, but will continue to classify assuming that the classification performed so far is correct.  

Our main assumption for content analysis is that articles which contains the same keywords also belong to some of the same categories in Wikipedia. This means that we want create a group of these articles so that similar articles are grouped together. Supervised classification requires, as already mentioned, a training set. The training set of our problem can be defined as articles in Wikipedia since they are already connected to a category within Wikipedia. The task is to create the classifier that use  this information and is able to classify all other articles. 

%does not have a training set because it is almost impossible to create a training set representiing such a large data set. We still have, however,  information about the underlying category structure of the articles in Wikipedia. The goal is therefore to use this information to group similar articles together.
%Our problem is not suitable for supervised machine learning. Trying to solve the problem with supervised classification would lead to some problems that are difficult to solve; it is for instance almost impossible to create a training set to represent such a large data set and it is therefore not possible to create a classification model based on the data. The categorization should therefore be done with unsupervised machine learning, for instance clustering.

The formal definition of cluster analysis or clustering is the task of grouping similar elements together.Hence the group or cluster should contain elements that share similarities or that are more similar to each other than to the rest of the elements. %This means that elements within a group are more similar to each other than to the rest of the elements, or that the elements within the group have some similarities that make them stand out from the others. 
Our problem could therefore be defined as a clustering problem, where each cluster or group is the articles which contains some of the same keywords. We want to sort the texts in such a way that texts with similar content are classified to the same cluster and therefore to same category. The problem needs a mapping process so that collections of texts get clustered together within the predefined set of categories. The predefined set of categories will change depending on the purpose of the classification, for instance would advertisement need a different set of categories than categorization of news articles. A proposal to a predefined category set for advertisement is the category set of IAB. 
\section{Wikipedia}
%It has already been mentioned that content analysis needs a keyword list for recognizing words or phrases that are useful for classification. We require that the list is so large that it contains almost all the words that give information of possible categories for the content where the keyword is found.  We have chosen to use Wikipedia to create such a keyword list. 
Wikipedia is a free, online encyclopedia and community that was created by Jimmy Wales in 2001. The encyclopedia is edited by the Wiki-principle, which means that everyone can create and edit articles. To understand the importance of Wikipedia it is worth mentioning that the web page has been ranked as the fifth globally most important web page (New York Times, February 2014), with  more than 30 million articles and almost 500 million unique users a month \cite{wiki:wikipedia}. 

Wikipedia contains a multitude of articles within many subjects and is maintained by thousands of people. 
%is a good choice for a base for the keyword list since it contains lots of articles within many subjects and is maintained by thousands of people. 
Hence, the idea is to base the list on all the titles in Wikipedia, but the list has to be modified to contain only relevant titles. It is for instance not relevant to have common words in the keyword list which will occur in most articles and not provide any useful information. 
%, like "the" for example, because these will not provide any useful information. 
It is also important to remove or weight down ambiguous words, i.e. words that could confuse the categorization process or apply wrong information. 

One of the main advantages of using Wikipedia is the underlying structure that is already provided. All articles are already categorized which gives information about the content of the article connected to the title. 

%There are many advantages of using Wikipedia, one of them is that all articles are already categorized which gives information about a possible category for each category. This means that the process of mapping between keyword and category  easily can be done by the computer. Another advantage is that it contains articles within various fields and is well maintained.

%TODO: Utdype mer om vekting osv. 
\section{Structure of Wikipedia}
The structure of Wikipedia is web based, where articles with similarities are linked together. Since Wikipedia is language-based, articles only link to other articles within the same language. Wikipedia does also have a category structure, where all articles are classified under at least one category. A category could have articles, but could also have subcategories, where the subcategories have their own articles and subcategories. The categories form a large category graph where articles are put under the most describing category, as an example is Ole Johan Dahl \footnote{\olejohandahleng} (Norwegian computer scientist) placed under the category \textit{Norwegian computer scientists} instead of the parent category \textit{Computer scientists by nationality}. 

The category graph is created so there is a link between a category and its subcategories. There is no beginning of the category graph, but here are some categories which have most other categories as their subcategories. These can be though of as beginning categories, often called root categories, and are important when we want to look through all categories in the graph and observe the relationships between them.  Two ategories that can be viewed as potential root categories are \textit{Fundamental Categories} and \textit{Main Topic Classifications}. If one of these are chosen as the root category, we can continue through the graph by looking at its subcategories and proceed by looking at each of the subcategory's subcategories an so on.
%An important 
%The easiest way of looking at all categories in the graph is to choose a root category and follow the links to its subcategories and then continue to look 

Figure \ref{fig: subcat_lindgren} is an example of a structure for the category \textit{Astrid Lindgren}, the swedish children's writer. The figure shows a tree structure for the category from the category graph. The figure shows that the category \textit{Astrid Lindgren} has 7 pages directly under the category, and 3 subcategories: \textit{Astrid Lindgrens karakterer} (7 pages), \textit{Astrid Lindgrens bøker} (9 pages) and \textit{Pippi Langstrømpe} (16 pages).  This means that there are indirectly 39 pages under the category \textit{Astrid Lindgren}. 


%is created and how it is fetched from the page for category information.\footnote{\categorytree}

\begin{figure}[H]
\centering
\includegraphics[height=2.5cm]{Dumps/imgs/Kategorier-Astrid-Lindgren.png}
\caption{Subcategories of the category \textit{Astrid Lindgren}. }
\label{fig: subcat_lindgren}
\end{figure}

% HVORFOR IKKE BRUKE WIKIPEDIA SINE KATEGORIER!
Articles of Wikipedia are already classified under categories, but the categorization between articles and categories cannot be used as a pre-defined category set. The reason is that articles in Wikipedia is categorize, but the categories are not guaranteed to be in the pre-defined category set. Hense it is essential that the classifier creates a connection from the article and to a category that is know to exist in the set. The classifier should instead be based on the category information provided by Wikipedia. 

%We will therefore need a mapper to a category we know exists. 
%but we cannot use either the categorization from articles to categories nor 
%this categorization is not ideal. It is not possible to use the categorization since a topic might lead
Another reason why it is not possible to use the category set in Wikipedia as the predefined category set because the category set in Wikipedia is too large. Some categories do not  provide information of the actual content, and some are too specified. There are also cases where articles are categorized under categories where the combination of the categories does not provide any new information. An example is the article of \textit{Ole Johan Dahl}. Some of the article's categories are showed in figure \ref{fig: olejohandahl_categories}. This is an example of an article where the categories provide the same information, i.e., we already know that he was from Norway since he is in the category \textit{People from Mandal, Norway}, so it would be enough to add that he was a computer scientist instead of specifying that he was a Norwegian computer scientist. 

\begin{figure}[H]
\centering
\includegraphics[width=\textwidth]{Dumps/imgs/olejohandahl-categories.png}
\caption{Some categories from the article of Ole Johan Dahl}
\label{fig: olejohandahl_categories}
\end{figure}

%set is not ideal as a predefined category set for our classifier. 
%There are two main reasons why the categories cannot be based on the categories from Wikipedia. 
%A reason for this is that an article in Wikipedia can be categorized under more than one category and these categories might not be the relevant category set. 
%In many cases are the categories directly subcategories of another category, but in some cases could it be a larger path until a common parent category and the category structure would therefore have to be flatten to make sure it is not classified under conflicting categories. 

%Hva tenkte jeg her? Another reason is that Wikipedia contains


\subsection{Accessing Information from Wikipedia}
There are two ways of accessing Wikipedia’s encyclopedic information; the most common way is to enter the webpage and search for the information needed, but it is also possible to download database dumps and access them directly to find information. All Wikipedia articles, images and categories are stored in a database which is accessed whenever a user search for information online, and the information retrieved from the database is returned to the webpage for example in the form of an article. To ensure that all data is safe at all times, files containing the information needed to recover the database is stored and regularly updated.\cite{wiki:databasedownload} This type of backup is called a database dump and is available for anyone interested at \url{http://dumps.wikimedia.org/enwiki/}.
%When a user search for information on the webpage this database is accessed. 
%which are accessed when a user are searching for an article online. 
%A database dump is therefore a backup of the database, and usually stored in the case of some data is lost\footnote{TODO Insert some link here. }. This backup is available for anyone interested at \footnote{TODO:insert link}. 

The files associated with the database dumps contains different information needed i.e., some files contains all the articles' titles, some contains information about which images belong to which articles and so on. Together they provide all information needed to restore Wikipedia if data is lost. 
%Just a few of these files where relevant for our task, the information needed was links between categories, between categories and articles and some special information about page properties. 
As mentioned, the first step is to find the full path of all articles. Since the Wikipedia articles are placed under categories describing their content, the full path of each article can be found by following the links between categories until an article is found. Table \ref{tab:databasedumpfiles} shows the files determined to be relevant for our task and a short description on what they contain. 


%This depends on creating a way to represent the structure of the categories and the articles. 

%and we can therefore define an article's path as the way to reach it from a given category. 
%The first step towards classification of Wiipedia articles is to find all full paths for the articles. There will be more than one way to reach many of the articles. 

%\begin{code}
%[INSERT EXAMPLE]
%\end{code}

%This task depends on different files from Wikipedia and should be split into smaller steps, hence several programs were made to complete the first task. 

%Several files were needed for the task, and the files depended on the language chosen. English is the language with most articles in Wikipedia, hence English were chosen and the 

%The files needed for this task we

\begin{table}[ht]
\renewcommand{\arraystretch}{1.25}
\begin{tabularx}{\textwidth}{l|X}
\textbf{File name} & \textbf{Information contained}\\ \hline
\texttt{enwiki-latest-categorgylinks.sql.gz} &  Containing information about links between categories, and between categories and articles. \\ \hline
\texttt{enwiki-latest-page.sql.gz} & Containing information about all pages in Wikipedia, including the type of page (category, article, user) and whether the page is a redirecting page or not\\ \hline
\texttt{enwiki-latest-page\_props.sql.gz} & Containing information about the properties of each page, including if the category is a hidden category or if the page a disambiguation page.
\end{tabularx}
\\[10pt]
\caption[Relevant files from Wikipedia database dump]{The relevant files from the Wikipedia database dump and a short description on what they contain}
\label{tab:databasedumpfiles}
\end{table}

\section{Interactive Advertising Bureau (IAB)}
The predefined category set should be well-defined and fit for the purposes of the task. Since the focus of this project is improving advertising, the predefined category set should be a category set useful for advertising. 


%The machine learning need a predefined set of categories for the clustering. It is already mentioned that Wikipedia has articles stored under categories and that the categories form a tree or graph structure. The problem is that there are too many categories that are not relevant for our categorization.

%The problem is therefore using IAB's categories for the clustering. 

IAB is a business organization that develops, researches and maintains industry standards for the online advertising industry. The organization works for creating, coalescing and maintaining standards and practices in online advertising. In addition, IAB research and share knowledge on the advertisement, and is responsible for distributing 86 \% of all the online advertisement in the US \cite{IABabout}.

IAB provides different guidelines for advertising, including \emph{Quality Assurance Guidelines Taxonomy} (QAGT). This taxonomy is a well-defined for advertising, and can be viewed as a category set. The set is split into two \emph{layers} also called \emph{tiers}. The layers are created for varying the grade of speciality between first tier (a general or broad level) and the second tier (a deepening level). The first tier contains a total of 23 categories, examples are \emph{Business} and \emph{Food \& Drinks}. The second tier contains 371 subcategories, where each subcategory is a more specified category of a category in the first tier. 

%, where the categories are subcategories of a category in the first tier.
Figure \ref{fig:IAB} shows the taxonomy of IAB as defined on their web page where the first tier is all the category names written in white (e.g. \emph{Food \& Drinks}) and the second tier is followed under the first tier (eg. \emph{American Cuisine}). Table \ref{tab:taxonomyascategories} is an example of how parts of the taxonomy for \emph{Food \& Drinks} and \emph{Hobbies \& Interests} is written as a category set, where the second tier is placed under the first tier. This means that an article mapping to \emph{Chinese Cuisine} maps to the category \emph{Food \& Drinks/Chinese Cuisine}.

\begin{table}[h]
\centering
\begin{tabular}{l|l}
%\textbf{Tier 1} & \textbf{Tier 2} \\ \hline
\textbf{Food \& Drinks} & \textbf{Hobbies \& Interests} \\ \hline
American cuisine & Art/Technology\\
Barbecues \& Grilling & Arts \& Crafts\\
Cajun/Creole & Beadwork \\
Chinese Cuisine & Birdwatching\\
Cocktails/Beer & Board Games/Puzzles\\
Coffee/Tea & Candle \& Soap Making\\
Cuisine-Specific & Card Games\\
Desserts \& Baking & Chess \\
... & ...
\end{tabular}
\caption{Example of how the IAB taxonomy changed to a category set}
\label{tab:taxonomyascategories}
\end{table}




%The category set from IAB's taxonomy is a well-defined category set to use of our clustering problem.  




% I stedet kan vi bruke Wikipedia-kategoriene for en sjekk for å se om v har kategoriesert rett?
%
\begin{figure}[t]
%\centering
\begin{subfigure}{\textwidth}
\includegraphics[width=\textwidth]{Chapters/Background/Taxonomy-1.png}
%\caption{Categories of the IAB Taxonomy}
%\label{fig:IAB1}
%\end{figure}
\end{subfigure}
\begin{subfigure}{\textwidth}
%\begin{figure}[H]
\centering
%\newline
\includegraphics[width=\textwidth]{Chapters/Background/Taxonomy-2.png}
\end{subfigure}
\caption{Categories of the IAB Taxonomy}
\label{fig:IAB}
%\label{fig:IAB-categories}
\end{figure}
%Figure \ref{fig:IAB-categories} 


%When categorizing a collection of texts, similar texts will be in the same cluster and therefore in the same category. This can be used to determine the content of the text. 

%The content analysis need a list of keyword to look for in texts. 


\section{Cxense}
\label{sec:cxense}
%Our project is a collaboration project between \emph{University of Oslo} and \emph{Cxense}, where Cxense provids lost of th



Cxense is a software company that collects and analyzes online information about Internet users. This information is used to create content profiles and user profiles, which can be used to understand the Internet activity. Their main goal is to understand the user's interest. Cxense provides software for companies, including tools to provide advertising, user recommendations and targeting emails \cite{aboutcxense}.

This project is a collaboration project between University of Oslo and Cxense. Thus, Cxense's software is used in our process of categorizing texts. Figure \ref{fig:categorization_figure} illustrates the complete categorization process of the keywords. When this process is complete, we have a list of categorized keywords which is needed for categorizing any input text. Cxense's software is part of our classifier in the process of finding keywords within the text. 
\begin{figure}[h]
\centering
\includegraphics[width=\textwidth]{Chapters/Background/Categorization_figure}
\caption[The categorization process of the keywords]{Illustration of the categorization process of the keywords.}
\label{fig:categorization_figure}
\end{figure}

There are different ways of finding the keywords in a text. Figure \ref{fig:cxensematching} illustrates the general approach where the whole dictionary is intersected with the document, and the result is a list of all entries found in both the dictionary and the document. 

\begin{figure}[h]
\centering
\includegraphics[width=\textwidth]{Chapters/Background/Cxense_matching}
\caption[Illustration of entry matching process]{Illustration of the categorization of a text, where the classifier finds all dictionary entries that occur in the text. }
\label{fig:cxensematching}
\end{figure}

Cxense's software allows the user to specify how the matching should occur. This is done by adding flags (specifications) in the intersection process. We have chosen to use exact matching in our project which means that the words have to be identical to be considered a match. We have also chosen to use case insensitivity (words in lower case can be identical match to words in upper case) and normalization of the words. The normalization of the words were to make sure all words were in the same character encoding. 

It is also possible to decide the lower limit of keyword occurrences from a category before the class is assigned to an article. This is done by creating weights of the keywords' classes and can be used to optimize the classifier.

%normalized text. 


%Det er “overlap” mode som brukes, typisk case-insensitivt (se “key-normalization-flags”), og “count” og “unique-count” styrer en precision/recall tradeoff. Ja, det er eksakt matching (modulo “key-normalization-flags”) og ikke matching på lemmatiserte verdier — vi har en opsjon for det siste også, men den er ikke dokumentert. Som default er vel “leftmost-longest-matching” true, IIRC.

\chapter{Previous Work}
Categorization is not a new topic, neither is taking advantage of Wikipedia in the categorization process. To avoid problems already solved by other projects, has some researching on other projects been done. It has been useful to look at other projects within the same topic and their approaches to solve similar problems. 

\section{Classification of tweets}
One of the similar projects is described in the article \emph{Entity Extraction, Linking, Classification, and Tagging for Social Media: A Wikipedia-Based Approach}\cite{entityextraction}. The initial problem described is a content analysis problem, where the goal is to categorize tweets based on their content. The chosen solution to the problem was automatic lssification of tweets, where the categorization is based on the content. This problem resemble our problem since both problems are based on understanding texts by recognizing keywords that provide information of the most likely categories. 


%where they wanted to let the computer understand what the tweets are about and then sort them based on their content. The chosen solution to the problem was to use automatic classification of the tweets, where the tweets were categorized by their content. This problem is very similar to our problem because both the problems are based on understanding texts by recognizing keywords that provide information of categories, i.e., what the tweets are about.

The article describes the categorization as the machine learning process where tweets with similar content are placed in the same class. Understanding the content requires the machine to have some basic knowledge, which is usually called a knowledge base or a repository for information. The atuhors chose to use Wikipedia as the knowledge base, for a numereous of reason, including that it is the largest online encyclopedia, that it is based on volunteering hence rapidly updated and since it is constantly crawling and therefore is able to have a fresh, dynamic and timely knowledge base. 


%The authors chose to use Wikipedia as their knowledge base (a repository for information) for finding information of the different categories. The reasons given for why they chose Wikipedia are similar to our reasons;  it is the largest online encyclopedia, it is based on volunteering which means that it is rapidly updated, and it is constantly crawling which is important since it is an advantage to have a fresh, dynamic and timely knowledge base. 

A difference between classifying tweets and articles are the preprocessing phase. Tweets require lots of preprocessing since they are quite short (max 160 characters). 


%while articles need preprocessing in order to 

The processing of the tweets required a lot of preprocessing before they could be classified content, especially since tweets are quite sort (max 160 characters).  The preprocessing of the tweets contained several steps before the actual categorization could start, including language detection, cleaning of the tweets (removing everything that is not text), and a tokenizing process (separating sentences into tokens, where a token is defined as a sequence of characters, usually normal words). The described preprocessing is similar to the preprocessing intended for our content analysis because the classifier will only find the keywords if they are an exact match. The classifier depend on tokenizing and cleaning of the words in the text to make them similar to keywords in the keyword list.  

The described tweet classification required some structure to keep information of the tweets and their possible categories. The solution was to create a structure of mentions where a mention is defined on the form ($m_{i}$, $n_{i}$, $s_{i}$), where $m_{i}$ is the string in the tweet we refer to, $n_{i}$ is the node in the knowledge base, and  $s_{i}$ is the< score of the node. All tokens with a connection to the knowledge base (i.e., Wikipedia) were considered relevant, while the others were removed to reduce the complexity. The structure of mentions could be too complex for our case with collections of texts since each text can be much longer than 160 characters, but the idea of keeping track of possible categories is the same. 

A scoring function was used for deciding categories for the tweets, where all the mentions were filtered and some hand-crafted rules were applied. Our project would also need some function to decide what categories are relevant if many categories are proposed. 




\section{Text Categorization with Encyclopedic Knowledge}
The task of automatic content analysis has many challenges that need to be solved. 

One of the most difficult challenges is how to deal with ambiguous words or phrases. Ambiguous words are words that have more than one meaning, and the meaning is usually found from the other words in the sentence. The problem is that complex sentences and advanced grammer makes it harder for the computer to decide the meaning of a word, hence it also makes it difficult to deide the meaning of the sentence. 
%The task of making the computer understand the content of text has many difficult challenges that most humans won't encounter. Ambiguous words are for example usually not a problem for humans because it is usually easy to use the context to understand the meaning of the word. 
%Computers on the other hand depend on a dictionary or statics about the word to decide the meaning of the word. Complex sentences could also be a problem for computers, advanced grammar for example can make it difficult to interpret the meaning of the sentence.
The paper \emph{Overcoming Brittleness Bottleneck using Wikipedia: Enhancing Text Categorization with Encyclopedic Knowledge} \cite{brittleness} focus on some of these challenges and presents a solution to these. 

Some of these findings are relevant for our project or interesting for further works, hence some solutions to the common ones are mentioned here. 

One of the main problems in  content analysis performed by a computer is to decide the meaning of a word. Humans have a larger background of knowledge and experience which makes it easier to interpret the meaning of a word. Computers depend on either representing documents as bag of words (BOW) or by learning the context of the word by observation. Context is difficult for a computer because it has to decide the number of words that are needed to decide the context of a word, which obviously depends on the word. 

%Ambiguous words are foten understood from the context, but a computer is then either depending on observing the word in a similar context 

%Ambiguous words or sentences are therefore seldom a  problem for humans because the meaning can be found in the context, but a computer could encounter problems with deciding the meaning if the context is unobserved.

The paper presents a text categorization feature to make it easier for the computer to understand the meaning of a word without analysing the context. A feature is a measure for a property of the observation, for instance is number of occurrences a normal feature for each word in the  BOW. If more than one feature measured for an observation are  usually put together as a feature vector for the observation. The feature generator described in the paper takes text fragments as input and maps these to the most relevant Wikipedia articles. The concepts in the relevant articles are used to find new features, which are added to the augmented bag of words. The authors have chosen the feature generator to be a multi-resolution to generate the best relevant Wikipedia concepts, i.e., it generates features at different levels: individual word level, sentence level, paragraph level and for the whole document. This means that there is a large number of features for each document, and these features are useful to understand the content of the document. 

The feature generator depends on  a text classifier that match documents with the most relevant articles of Wikipedia. The classifier starts by manipulating the text into the same form as the encyclopedic articles. This part resembles our categorization problem since we are also interested in linking texts to Wikipedia or information from Wikipedia.  The discussion about ambiguity is also relevant for our problem, where ambiguous words should either be dropped from the keyword list or the computer will have to find the meaning of the words. 


\chapter{Methods}
This chapter can be viewed as an introduction to the methods we chose for the implementation of the project. It gives a brief introduction to how we determined the meaning of Wikipedia articles and the structure chosen for representing the information.

%This chapters describes methods chosen for the project and the structures 

%What do I need methods for?

\section{Finding the meaning of Wikipedia Articles}
It is essential to know the meaning of the Wikipedia articles to be able to categorize them. 
Our assumption is that the meaning of Wikipedia articles can be found by looking at the categories leading to the article in the underlying category structure of Wikipedia. We base this assumption on the fact that all Wikipedia articles are placed under at least one category, and that the articles' categories should be representative for the article. This means that we need to find a representation for the underlying structure and a way of deciding the best way of reaching each article within this structure.
%One of the most common ways
%One way of finding the meaning of WIkipedia articles is by looking at Wikipedia's underlying structure since all Wikipedia articles are placed in categories. 

% One of the most commonly used strategies of finding the meaning of the articles is by looking at the 

\subsection{Representing the underlying structure}
Taking advantage of the underlying structure of Wikipedia requires a way of representing it. Each category has links to its subcategories, and links to the articles which are placed under the category (see figure \ref{fig:graphstructure}). Representing the structure could be split into two parts; one structure representing the underlying category structure (see figure \ref{fig:categorystructure}), and one structure representing the categories of each article (see figure \ref{fig:articlestructure}).  %representing the structure between categories, and representing the structure between categories and articles. 

\begin{figure}
\centering
\includegraphics[width=\textwidth]{Chapters/Methods/graphstructure}
\caption{Simplified illustration of the underlying structure of Wikipedia.}
\label{fig:graphstructure}
\end{figure}

\subsubsection{Category graph}
A category graph is a way of representing the links between categories. This structure contains information about which subcategories can be reached from each category. Figure \ref{fig:categorystructure} illustrates the category graph for representing the structure found in figure \ref{fig:graphstructure}. The nodes in the graph (rectangles with rounded corners) represent categories, and the edges (arrows) represent the relationships between categories. The graph illustrated is a directed graph since each edge represents the relationship between the two categories (e.g. \emph{Subcategory 1} is subcategory of \emph{Category} since the arrow points from \emph{Category} to \emph{Subcategory 1}).

\begin{comment}


The graph has to contain directed edges, which means that a category is considered the subcategory of another if the category is 

can be created by representing categories as nodes and links as a 


Graph: nodes are categories and the edges are hyperlinks. 
--> nodes are categories and the edges are links between categories 


Such a graph is represented with all subcategories of a category listed under the category. The results of this is a structure like the illustration in figure \ref{fig:categorystructure}).
\end{comment}

%A category graph is a way of representing links between categories i.e., which categories can be reached from each category. The file containing all links between categories can be used to create such a graph. This is done by finding all subcategories of each category and remove all duplicate links. The results of this is a structure like the illustration in figure \ref{fig:categorystructure}).
\begin{figure}[h]
\centering
%\begin{subfigure}[b]{0.4\textwidth}
\includegraphics[width=0.7\textwidth]{Chapters/Methods/category-subcategories}
\caption{The structure where each category knows its subcategories}
\label{fig:categorystructure}
%\end{subfigure}
%\begin{subfigure}[b]{0.4\textwidth}
\end{figure}

\begin{figure}
\centering
\includegraphics[width=0.5\textwidth]{Chapters/Methods/categories-articles}
\caption{The structure where each category know the title of its articles}
\label{fig:articlestructure}
%\end{subfigure}
%\caption[The representation of the Wikipedia structure]{Combined this is the structure needed to represent the Wikipedia's underlying category structure as a graph}
\end{figure}


\subsubsection{Article graph}
A similar structure is desirable for representing articles and their most describing categories (the categories shown at the bottom of the article page in Wikipedia). Figure \ref{fig:articlestructure} illustrates how we represent each category's immediate articles by creating edges (arrows) between categories and their articles. 

%The file containing all links between categories and their articles can be used to create a structure where each category knows its articles. Figure \ref{fig:articlestructure}) illustrates this structure. 

%It is desirable to remove articles whose titles are not relevant for our project. Numbers without context is an example of Wikipedia article titles that are difficult to determine the meaning since a number could have various meanings, including temperatures, grades or years. Hence, all article titles which only contains numbers could be disregarded. Wikipedia contains many such articles, and a total of 23 227 articles where found. This reduces the number of links betweeen articles and categories as shown in table  \ref{tab:withoutnumber}.


\subsubsection{Representing category and article names}
\emph{Id mapping} is a storage efficient way of representing category names and article titles. Category names and article titles are usually longer than their representing ids because ids can be chosen as increasing digits. The id mapper is implemented by creating a counter that assigns a unique number to each category name or article title not already observed. 

\begin{comment}
The files containing the results becomes extremely large due to the size of the results. When writing all the results to file, the files becomes extremely large. All paths of all Wikipedia articles is more than 20 GB of compressed data. It is desirable to reduce the space needed for storing all results on the computer. The solution was to create an id mapping for each category name and article name. Id mapping gives all names a unique id, and instead of writing the full path of category names to the file, the full paths with category ids is written to file. 

The id mapping is implemented by creating a counter that assigns numbers to each category name or article name that is not found yet, i.e., a unique number represents each name. Figure \ref{fig:idmapping} shows an excerpt of the id mapping created for our purpose, where the id \emph{4600570} corresponds to the article about \emph{Ole-Johan Dahl}, which means that this id is used everywhere \emph{Ole-Johan Dahl} is used in paths. 

Id mapping is storage efficient because category names and article names usually are a  lot longer than their representing ids. 

Working with ids is also faster in many implementations concerning lookups in the program. This depends on the structure chosen for the programs, but when using dictionaries as done in our implementation, ids will perform faster than if using full names. An example of this can be seen in figure \ref{fig:id_lookup} where the time to find all categories from the category with id 177678 (corresponding to the category \emph{people}) is 0.955 minutes. Figure \ref{fig:fullname_lookup} shows the time needed to find the same paths for the category when using full names for categories and articles, which is found to be 1.559 minutes. Comparing the times shows that the time is a lot faster when using ids, which is important when many paths have to found.

The last reason to use ids instead of full names is that the full names may include characters useful for describing paths, for instance the characters "/" which is a common way of describing full paths. 

% Fordel 2: Kan bruke "/" in the text. 


\end{comment}



\begin{comment}
It is important that the category names are equal all places they occur. Wikipedia is written by volunteers from all over the worlds, and users might use different encoding depending on where they are from. Thus, both a cleaning process and a normalization process should be performed on all category names. The cleaning process is to make the category names look readable, while the normalization is a process where all words are made equal regardless of character encoding \cite[p.~26]{iirbook}.

The cleaning process includes converting all words to lowercase, replacing underscores with spaces and splitting up all titles containing the code for newline (\emph{\textbackslash n}). Newline is a way of representing how the articles should be sorted, figure \ref{fig:withnewline} is an example of an \texttt{INSERT} statement with newline in the title of the category, where the category should be sorted as if the title was \emph{ducks} as seen in figure \ref{fig:fictionalbirds}. Hence, the relevant part of the category title is the part after the newline, and this is the part that is considered further in the results. 


\begin{equation}\label{eq:removehiddencat}
a = 0
\end{equation}
\end{comment}

\section{Grading Categories}
Many Wikipedia articles can be reached from categories that are not descriptive of the content of the article. We found multiple paths to all Wikipedia articles, but some of them were less descriptive than others. Thus, a grading was done to find the most relevant paths for each article. 

\subsection{Grading based on Inlinks and Outlinks}
%\subsubsection{Inlinks and Outlinks of Categories}
Each category in Wikipedia has a set of super categories (categories that lead to the current category), and a set of subcategories (categories that can be reached from the current category). The super categories and subcategories leading to the current category form a set. The size of this set can be annotated as 
\begin{itemize}
\item \emph{Inlink number} = number of super categories (parent categories)
\item \emph{Outlink number} = number of subcategories (child categories)
\end{itemize}
Figure \ref{fig:Categorywparentandsub2} illustrates how  inlink number and outlink number are connected to a category. 

%We assume  that a category with high inlink and outlink number are more likely to be visited when looking for paths for an article. 

\begin{figure}[h]
\centering
\includegraphics[width=\textwidth]{Chapters/Methods/category_parent_sub}
\caption[Example of \emph{inlink number} and \emph{outlink number} for a category]{Example of how a category has links from parent categories and links to its subcategories. The \emph{inlink number} for the category is 4 and the \emph{outlink number} for the category is 3.}
\label{fig:Categorywparentandsub2}
\end{figure}

We created two assumptions based on this: 
%Two assumptions can be made from this. 
\begin{enumerate}
\item Categories with high inlink number can be reached from categories that are not about the same topic. 
\item Categories with a high outlink number are more likely to reach articles not necessarily connected to the category name since they can reach far in all the subcategories' directions.
\end{enumerate}
%The first assumption is that The other assumption is that  

All categories should be given a score based on their inlink and outlink number, where low score values are given to categories within narrow topics (high inlink and outlink number), and higher score values are given to categories that cover more general topics (low inlink and outlink number). 

%Categories with high inlink and outlink numbers should be given a lower score than categories seldom reached. 

%They cover a general topic, while categories with a low inlink number and low outlink number describe a narrow topic.

\subsubsection{Scoring paths}
Grading based on inlink and outlink number is done by finding the inlink and outlink number for all categories in the structure, and by finding the average inlink and outlink numbers for all categories. The scoring is weighted based on the values of the inlink and outlink numbers. This gives the following formula (equation \ref{eq:categoryscore}) for finding the score of a given category, where $\xoverline{C_{in}}$ is the average inlink number and $\xoverline{C_{out}}$ is the average outlink number.   


\begin{equation} \label{eq:categoryscore}
Score_{C} = \frac{inlink_{c} + outlink_{c}}{\xoverline{C_{in}} + \xoverline{C_{out}}}
\end{equation}

This means that the path score of a path $P$ is the sum of all scores for each category in the path (see equation \ref{eq:scoreinput}). 

\begin{equation} \label{eq:scoreinput}
Pathscore_{P} = \sum_{c} Score_{C}
\end{equation}

The problem with equation \ref{eq:scoreinput} is that short paths will be favored since there are fewer scores to be added together. A way of avoiding favoritism of short paths is by normalizing the path scores. 

\subsection{Normalized Grading based on Inlink and Outlink Numbers}
Grading based on inlink number and outlink number favors short paths even if the paths contains categories considered as bad. One way of handling this problem is by normalizing the score of each path. Equation \ref{eq:normscoreinput} is a way of normalizing the path score of path $P$ so the length of the path does not determine the relevance of the path. 

% TODO: Write something about normalization - why is it good for grading?

\begin{equation} \label{eq:normscoreinput}
Pathscore_{P} = \frac{1}{N} \sum_{c} Score_{C}
\end{equation}
where $N$ is the number of categories in the path.


\subsection{Deciding Relevant Paths}
There are different ways of deciding the relevant paths among all graded paths. One way is by choosing a threshold for the path score. If the path score is lower than a given threshold, it is marked as relevant, while a path score higher than the threshould means that it is not relevant. A threshold can be found by deciding how many paths should be considered relevant.

One way of doing this is by finding the scores of all paths. and sort the scores from lowest to highest (see \ref{eq:sortedscores}). Then a $k$ has to be decided to how many paths are believed to be relevant of all paths, for instance one could assume that only 10\% of the paths are relevant, which leads to $k = .10 \cdot n$. 

\begin{equation} \label{eq:sortedscores}
Sorted\_scores = \left[ S_{1}, S_{2}, ... , S_{k}, ... , S_{n} \right]
\end{equation}



\begin{equation} \label{eq:threshold}
T = Sorted\_scores[k]
\end{equation}


The problem with this method is that not all articles are guaranteed to have any relevant paths. The other problem is that the score of the path will vary a lot within different fields, since some of the Wikipedia articles are categorized under very specified categories. 
% TODO: Finn en kilde som er enig med meg. 

% Problem: 
% Finne hvor mange pather som er tilgjengelig. 

Another approach is to choose the best $k$ paths for each Wikipedia article. This approach is independent of the values on other articles' path score which means all Wikipedia articles are guaranteed at least one path. The disadvantage is that some paths might be marked as relevant even though their path score is lower than path scores marked as irrelevant by other articles. Another disadvantage is that articles with many good paths will still have to choose the best $k$ paths and good paths might be lost. 

\begin{comment}
Fordeler: ser ikke på de andre
alle articler får minst en score. 

Ulemper: mange gode - hvilken er best?
Kan ikke vite om scoren er god

\end{comment}

\section{Evaluation}
An evaluation of the categorization process is essential to know whether the classifier classify correctly or not. This can also be used to find which categories are easy to classify, and which categories are difficult to recognize. The evaluation is based on comparing the results with the correct results (called \emph{Gold Standard} \cite{wiki:goldstandard}). The gold standard in our project is found in the url of articles, and is decided by the journalists when they publish articles. An article about sport contains \emph{sports} in the url, for example \emph{http://www.rappler.com/sports/by-sport/boxing-mma/pacquiao/90563-mayweather-sr-blasts-ariza}.

\subsection{Evaluation of the Classifier}
There are different ways of measuring correctness, but the most common are \emph{accuracy}, \emph{precision}, \emph{recall} and \emph{$F_{1}$-score}. These measures depend on some terms for the evaluation (see table \ref{tab:retrievedescription}).
%Evaluation the categorization is found by evaluating how well the classifier perform, in other words the correctness of the classifier.
%the correctness of the classifier
%The purpose of evaluating the classification is to determine  the correctness of the classifier 
%i.e., how well the classifier perform. 

\subsubsection{Accuracy}
\emph{Rand Index (RI)} accuracy measures the percentage of decisions that are correctly classified by the classifier \cite[p:~330]{iirbook}. Equation \ref{eq:accuracy} \cite{wiki:accuracy} shows how this is computed for evaluating the classifier. 

\begin{equation} \label{eq:accuracy}
\text{acc}=\frac{\text{true positives}+\text{true negatives}}{\text{true positives}+\text{false positives} + \text{false negatives} + \text{true negatives}}
\end{equation}

\begin{table}[ht]
\centering
\renewcommand{\arraystretch}{1.25}
\begin{tabularx}{\textwidth}{l |X}
\textbf{Term}  & \textbf{Description} \\\hline
\textbf{True Postive} (TP) & Text is classified to the class by both classifier and \emph{Gold Standard}, (correct). \\ \hline
\textbf{True Negative} (TN) &  Text is neither classified to the class by the classifier, nor by \emph{Gold Standard}, (correct).  \\ \hline
\textbf{False Negative} (FN) & Text is not classified to the class by the classifier, but by \emph{Gold Standard}, (incorrect). \\ \hline
\textbf{False Positive} (FP) & Text is classified to the class by the classifier, but not by \emph{Gold Standard}, (incorrect).
\end{tabularx}
\\[10pt]
\caption[Explanation of the \emph{TP}, \emph{TN}, \emph{FN} and \emph{FP}]{Explanation of the \emph{True Positive}, \emph{True Negative}, \emph{False Negative} and \emph{False Positive} \cite[p.~330-331]{iirbook}.}
\label{tab:retrievedescription}
\end{table}

\subsubsection{Precision and Recall}
Another way of evaluating the classifier is by using \emph{precision} and \emph{recall} which measures how many elements are correctly categorized and how many of the correct elements were found. 

%of evaluation categorization is with \emph{precision} and \emph{recall} which are measures of how many elements were correctly categorized \cite{wiki:precisionrecall}. P

Precision is defined as in equation \ref{eq:precision} \cite{wiki:precisionrecall}, which measures the fraction of returned results that are relevant \cite[p.~5]{iirbook}. This means that precision can tell how many of the articles were correctly categorized. 

\begin{equation} \label{eq:precision} 
\begin{split}
\text{precision} & =\frac{|\{\text{relevant documents}\}\cap\{\text{retrieved documents}\}|}{|\{\text{retrieved documents}\}|} \\
 & = \frac{\text{TP}}{\text{TP} + \text{FP}}
 \end{split}
\end{equation}
Recall is a measure of finding how many of the relevant documents were found \cite[p.~5]{iirbook}. Equation \ref{eq:recall} \cite{wiki:precisionrecall} would provide information about how many of the correctly categorized elements were found. 

\begin{equation} \label{eq:recall} 
\begin{split}
\text{recall} & =\frac{|\{\text{relevant documents}\}\cap\{\text{retrieved documents}\}|}{|\{\text{relevant documents}\}|} \\
 & = \frac{\text{TP}}{\text{TP}+\text{FN}}
\end{split}
\end{equation}

Combining precision and recall gives a measure of the correctness of the classifier. In addition, the measures can be combined to find the $F_{1}$-measure of the classifier which is way of measuring accuracy in terms of a weighted average of the precision and recall. The $F_{1}$-score is defined as in equation \ref{eq:fscore} \cite{wiki:fscore}. 

\begin{equation} \label{eq:fscore}
F_1 = 2 \cdot \frac{\mathrm{precision} \cdot \mathrm{recall}}{\mathrm{precision} + \mathrm{recall}}.
\end{equation}

The range of the $F_{1}$-score is between $0$ and $1$, where 1 is the best value. 

% TODO: vi kan også evaluaere mappingen mellom wikipedia articler og iab kategorier. 

%\subsection{Evaluate Wikipedia Categorization}
%\subsection{Evaluate Article Categorization}

\begin{comment}
Evaluation is the 


Kan også finne: 
p. 330 i iirbook. 
Rand index : measures the RI percentage of decisions that are correct. 
\end{comment}
\subsection{Optimize the Classifier}
%The measures for evaluating the classifier are best if they are combined. 

The measures for evaluation are used to determine how well a classifier performs and to determine how the classifier could be optimized to perform better. A perfect classifier categorizes all documents to their most describing classes without classifying documents to classes they don't belong to. Figure \ref{fig:perfect_classifier} illustrates a perfect classifier which classify all documents to their correct classes. The classification results can be seen in table \ref{tab:results_perfect_classifier}.


%This is a difficult task, and we want the classifier to achieve high scores in the evaluation.
\begin{comment}
\begin{figure}[h]
\centering
\includegraphics[width=.7\textwidth]{Chapters/Methods/All_classes}
\caption{Caption}
\label{fig:my_label}
\end{figure}

\begin{table}[h]
\centering
\renewcommand{\arraystretch}{1.25}
\begin{tabular}{c|l}
\multicolumn{2}{c}{\textbf{Number of elements in the class}} \\ \hline
\textbf{Class 1} & 5 \\ \hline
\textbf{Class 2} & 6
\end{tabular}
\\[10pt]
\caption{Caption}
\label{tab:my_label}
\end{table}

\end{comment}

\begin{figure}[h]
\centering
\includegraphics[width=0.7\textwidth]{Chapters/Methods/Perfect_classifier}
\caption{Illustration of a perfect classifier.}
\label{fig:perfect_classifier}
\end{figure}

\begin{table}[h]
\centering
\renewcommand{\arraystretch}{1.25}
\begin{tabular}{c|l|l|l}
\multicolumn{4}{c}{\textbf{Perfect classifier: results}} \\ \hline
\textbf{TP} & 5 &\textbf{Precision} & 1 \\ \hline
\textbf{TN} & 5 &\textbf{Recall} & 1 \\ \hline
\textbf{FP} & 0 &\textbf{Accuracy} & 1 \\ \hline
\textbf{FN} & 0 &\textbf{$F_{1}$-score} & 1
\end{tabular}
\caption{Classification results for class 1 for a perfect classifier.}
\label{tab:results_perfect_classifier}
\end{table}

\subsubsection{Why we need more than one measure for evaluation}
Creating a perfect classifier is difficult, and it is difficult to determine if the classifier perform well. The different measurements for evaluation are best when they are combined (as $F_{1}$-score), because accuracy, precision and recall can be have good results separately even if the classifier is far from perfect.  

Figure \ref{fig:high_precision}) and \ref{fig:high_recall}) illustrates classifiers that have respectively high precision and high recall. Their results can be found in table \ref{tab:results_bad_classifiers} where we can see that high precision can be found by classifier that only retrieved a few results and high recall is found for classifiers that retrieve many results. Thus, a good classifier should be neither of these, but instead balance the results. 


\begin{figure}[h]
\centering
\begin{subfigure}[b]{0.6\textwidth}
\includegraphics[width=\textwidth]{Chapters/Methods/High_precision}
\caption[Illustration of bad classifier with high precision]{Classifier A: Illustration of bad classifier with high precision.}
\label{fig:high_precision}
\end{subfigure}
\\[10pt]
\begin{subfigure}[b]{0.7\textwidth}
\includegraphics[width=\textwidth]{Chapters/Methods/High_recall}
\caption[Illustration of bad classifier with high recall]{Classifier B: Illustration of classifier with high recall.}
\label{fig:high_recall}
\end{subfigure}
\end{figure}


\begin{table}[h]
\centering
\renewcommand{\arraystretch}{1.25}
%\begin{tabular}{c|l|l|l|l}
%\multicolumn{2}{c}{\textbf{Classifier A}} & Classifier B \\ \hline
\begin{tabular}{l|l|l|l|l|l|l|l}
\multicolumn{4}{l}{{\bf Classifier 1}} & \multicolumn{4}{l}{{\bf Classifier2}} \\ \hline
\textbf{TP}     & 1     & \textbf{Precision}     &  1   & \textbf{TP}     & 5     & \textbf{Precision}    &   0.5  \\\hline
\textbf{TN}     & 5     & \textbf{Recall}        &  0.2   & \textbf{TN}    & 0     & \textbf{Recall}       & 1    \\\hline
\textbf{FN}     & 4     & \textbf{Accuracy}      &  0.6   & \textbf{FN}     & 5     & \textbf{Accuracy}     & 0.667    \\\hline
\textbf{FP}     & 0     & \textbf{$F_{1}$-score}      &  0.333   & \textbf{FP}     & 0     &  \textbf{$F_{1}$-score}             &    0.667
\end{tabular}
\caption{Evaluation of classifier A and B for class 1. }
\label{tab:results_bad_classifiers}
\end{table}

\begin{comment}
\begin{table}[h]
\centering
\renewcommand{\arraystretch}{1.25}
\begin{tabular}{c|l|l}
%\multicolumn{2}{c}{\textbf{Classification results for class 1}} \\ \hline
& \textbf{Classifier A} & \textbf{Classifier B} \\ \hline
\textbf{Precision} & 1 &  \\ \hline
\textbf{Recall} & 5 \\ \hline
\textbf{Accuracy} & 0 \\ \hline
\textbf{$F_{1}$-score} & 4
\end{tabular}
\caption{Caption}
\label{tab:results_high_precision}
\end{table}
\end{comment}

\begin{comment}
It is not enough to classify all documents in a class to the correct class. A perfext alskd
A classifier which categorizes all documents in a class to 
A perfect classifier categorizes all documents to the right class. We have created a classifier which might classify documents to more than one class. 
the correct results and none of the incorrect ones. 
This is a difficult task, so we try to optimize the classifier so that it retrieves most of the correct results without starting to 

The best classifier should be a classifier that retrieves m

Example of a bad classifier: It categorizes all sports article to the class sports, but also all non-sport articles to the class sport. The 

\end{comment}

\chapter{Methods}

\input{Chapters/Implementation/Finding_Article_Paths}
\subsubsection{Hidden categories}
Wikipedia's category structure contains lots of hidden categories which are not displayed at the bottom of an article page for the general users, even if the article is placed under the category. These categories are useful for editing since it is an easy way to all mark categories with something in common, for instance mark all categories with references that needs to be checked. 

Hidden categories are concerned with maintenance and administration, hence not relevant for normal users or for our problem. The next step is therefore to remove all the links to hidden categories, which led to the task of finding all hidden categories. On Wikipedia's information page about \emph{Hidden Categories}\cite{wiki:hiddencat} are 15 385 subcategories listed as immidiate subcategories, but many of these categories have links to their own hidden subcategories which also have to be found. The first attempt was to look through all the links from the category \emph{Hidden Categories}, where 15 006 subcategories where found and marked as not relevant. Since this did not give the expected number, another attempt was made by looking at the file \texttt{enwiki-latest-page\_props.sql.gz}, where figure \ref{fig:pageprops} shows how  hidden categories are marked in the table. The next attempt was therefore to find all the ids marked with \emph{hiddencat} and find the corresponding category titles in \texttt{enwiki-latest-page.sql.gz}. This approach led to 15 513 categories. To make sure that all hidden categories where found, a test was made to see if all categories from the first attempt was found in the list created from the second attempt. The results showed that all categories found in the first attempt was also found in the second attempt, and the list of all 15 513 category titles whose links should be disregarded from further results. 

\begin{figure}[h]
\centering
\begin{lstlisting}
(747593,'hiddencat','',NULL)
\end{lstlisting}
\caption[Insert statement for hidden category]{Excerpt from the file \texttt{enwiki-latest-page\_props.sql.gz} where we can see that hidden categories are marked with \emph{hiddencat}}
\label{fig:pageprops}
\end{figure}

The hidden categories have to be removed carefully because they might be subcategories of visible categories or have visible categorise as their own subcategories. An example of this can be seen in figure \ref{fig:stevie_wonder_hidden}, where the double rounded rectangle is a hidden category, the rounded rectangle is a normal (visible) category and the rectangle is the article about \emph{Stevie Wonder}.
%Hidden categories can not be disregard
%19103360 article links, 391482 category links skipped

%An example of such a structure can be found from the categories leading to the article about the singer Stevie Wonder. 

\begin{figure}[h]
\centering
\includegraphics[width=\textwidth]{Chapters/Implementation/HiddenCategories/Stevie_wonder_hidden}
\caption[Example path with hidden category]{An excerpt of one path leading to the article about to Stevie Wonder, where the path contains a hidden category. }
\label{fig:stevie_wonder_hidden}
\end{figure}

The desirable visible paths for all articles are paths without hidden categories. The next step is therefore to change the structure so that hidden categories are removed from the structure, but without loosing any of the subcategories which might contain relevant information or  important links. Example of a how a path can be transformed is figure \ref{fig:stevie_wonder} which is the excerpt from the path in figure \ref{fig:stevie_wonder_hidden} without the hidden categories. 

\begin{figure}[h]
\centering
\includegraphics[width=.7\textwidth]{Chapters/Implementation/HiddenCategories/Stevie_wonder}
\caption[Example path without hidden category]{The desirable output of the excerpt of the path leading to the article about Stevie Wonder where the hidden category is removed from the path}
\label{fig:stevie_wonder}
\end{figure}


Table \ref{tab:withouthiddencat} shows how number of links between categories, and between categories and articles are reduced when hidden categories are not considered. 

%The main reason to reduce number of links is to reduce the complexity 

\begin{table}[h]
\centering
\begin{tabular}{l|c|c}
\textbf{Links between...} & \textbf{W/ Hidden Categories} & \textbf{W/o Hidden Categories}  \\ \hline
 \textbf{subcategories} & 1 654 758  & 1 311 275\\
 \textbf{articles and categories} & 4 241 881  & 3 152 873
\end{tabular}
\caption[Number of links without hidden categories]{Number of links removed when all hidden categories are excluded. }
\label{tab:withouthiddencat}
\end{table}

\begin{comment}
Dette må endres, for dette er feil! 
\end{comment}

\subsection{Representing the Underlying Structure}
It is important that the category names are identical at all places they occur. Wikipedia is written by volunteers from all over the worlds, and users might use different encoding depending on where they are from. Thus, both a cleaning process and a normalization process should be performed on all category names. The cleaning process is to make the category names look readable, while the normalization is a process where all words are made equal regardless of character encoding \cite[p.~26]{iirbook}.

Figure \ref{fig:withnewline} is an example of a \texttt{INSERT} statement which represents a link between the category \emph{fictional\_birds} and the subcategory \emph{ducks\textbackslash n fictional ducks}. This statement is an example of two category names that need to be processed so that they appear as \emph{ficitonal birds} and \emph{fictional ducks}. This processing  is usually called a \emph{data cleaning process} \cite{datacleaning}. The data cleaning for our purpose is converting all words to lowercase, replacing underscores with spaces and splitting up titles containing the code for newline (\emph{\textbackslash n}). Wikipedia uses the code for newline to represent how the articles should be sorted. Figure \ref{fig:fictionalbirds} shows that \emph{fictional ducks} are sorted as if it started with the word \emph{ducks}.

%The cleaning process includes converting all words to lowercase, replacing underscores with spaces and splitting up all titles containing the code for newline (\emph{\textbackslash n}). Newline is a way of representing how the articles should be sorted, figure \ref{fig:withnewline} is an example of an \texttt{INSERT} statement with newline in the title of the category, where the category should be sorted as if the title was \emph{ducks} as seen in figure \ref{fig:fictionalbirds}. Hence, the relevant part of the category title is the part after the newline, and this is the part that is considered further in the results. 
% Write something about normalization here. 

%Newline inside a category title is a way for Wikipedia to save space about the category nformation.  An example of such a statement is found in figure \ref{fig:withnewline}. 

%\footnote{TODO: insert reference: part of the insertion statement from the file \enwikicatlink} 

\begin{figure}[h]
\begin{lstlisting}
(1517681,'fictional_birds','ducks\nfictional ducks','2014-10-26 03:30:11',
'ducks','uppercase','subcat')
\end{lstlisting}
\caption[\texttt{INSERT} statement with newline]{Excerpt from \texttt{enwiki-latest-categorylinks.sql.gz} showing an \texttt{INSERT} statement including a newline character. }
\label{fig:withnewline}
\end{figure}

\begin{figure}[h]
\centering
\includegraphics[width=\textwidth]{Chapters/Implementation/Fictional_birds_2}
\caption{The subcategories of the category \emph{Fictional birds} and how its subcategories are sorted based on the defined sortkey instead of the category title }
\label{fig:fictionalbirds}
\end{figure}

%This part of the \texttt{INSERT} statements means that the category \emph{fictional ducks} is both a subcategory of the category \emph{Fictional birds} and the category \emph{ducks}, hence the \texttt{INSERT} statement results in two links, one from \emph{fictional birds} to \emph{fictional ducks} and one from \emph{ducs} to \emph{fictional ducks}.

After processing all titles, they are sorted into two files depending; one for links between categories and one for links between categories and articles. These files are needed for creating the structures for finding full paths of all Wikipedia articles. 

%After the file \texttt{enwiki-latest-categorylinks.sql.gz} was split into two files where the first one contained all links between categories and the second file contained all links between categories and articles. 

\begin{comment}
\subsubsection{Creating the Category graph}
A category graph is a way of representing links between categories i.e., which categories can be reached from each category. The file containing all links between categories can be used to create such a graph. This is done by finding all subcategories of each category and removing all duplicate links. The results of this is a structure like the illustration in figure \ref{fig:catstructure}).

\begin{figure}[h]
\centering
\begin{subfigure}[b]{0.4\textwidth}
\includegraphics[width=\textwidth]{Chapters/Implementation/category-subcategories}
\caption{The structure where each category knows its subcategories}
\label{fig:catstructure}
\end{subfigure}
\begin{subfigure}[b]{0.4\textwidth}
\includegraphics[width=\textwidth]{Chapters/Implementation/categories-articles}
\caption{The structure where each category know the title of its articles}
\label{fig:artstructure}
\end{subfigure}
\caption[The representation of the Wikipedia structure]{Combined this is the structure needed to represent the Wikipedia's underlying category structure as a graph}
\end{figure}


\subsubsection{Creating the Article graph}
It is also relevant to create a system of all the articles and their most describing categories, that is the categories shown at the bottom of the article page. The file containing all links between categories and their articles can be used to create a structure where each category knows its articles. Figure \ref{fig:artstructure}) illustrates this structure. 
\end{comment}
It is desirable to remove articles whose titles are not relevant for our project. Numbers without context is an example of Wikipedia article titles that are difficult to determine the meaning since a number could have various meanings, including temperatures, grades or years. Hence, all article titles which only contains numbers could be disregarded. Wikipedia contains many such articles, and a total of 23 227 articles where found. This reduces the number of links betweeen articles and categories as shown in table  \ref{tab:withoutnumber}.

\begin{table}[h]
\centering
\begin{tabular}{c|c}
\textbf{W/ Number Articles} & \textbf{W/o Number Articles}  \\ \hline
52 611 629 & 52 588 894
\end{tabular}
\caption[Number of links without number articles]{Number of links between categories and articles removed when articles only containing numbers are disregarded}
\label{tab:withoutnumber}
\end{table}


% 23227
\input{Chapters/Implementation/Articlegraph}

\subsection{Finding Full Paths of Wikipedia Articles}
Finding the full paths for each Wikipedia article can be done when the representation of the structure is ready. Each path can be found by following the links between categories until an article is reached, and the links categories visited are the path. 

\begin{figure}[h]
\centering
\includegraphics[width=\textwidth]{Chapters/Implementation/example_path}
\caption[Example of an article path]{Example of one of the article paths of the article \emph{Ole-Johan Dahl}. The rectangles are categories and the rectangle with rounded corners is the article. }
\label{fig:examplepath}
\end{figure}

\subsubsection{Issues with finding the full path}
The structure of Wikipedia is not represented as a tree, but as a graph. This means that there might be loops within the graph. A loop within the graph means that a category already visited in the search of a path can be reached again. Figure \ref{fig:exampleloop} shows an example of a loop in the graph. 

% 177820/907585/173722/572284/531983/173722
% people/fictional characters/fictional characters by species/fictional life forms/legendary creatures in popular culture/fictional characters by species
\begin{figure}[h]
\centering
\begin{lstlisting}
people/fictional characters/fictional characters by species/fictional life forms/legendary creatures in popular culture/fictional characters by species
\end{lstlisting}
\caption{Example of a loop found in the graph.}
\label{fig:exampleloop}
\end{figure}


This leads to problems if the program keep going in loop and does not reach an article. A solution to this problem is to keep track on categories already visited and only follow links to categories not yet visited in the path. 
%This might mean that the path to the category is not the best one, but 

Another issue is to decide the start point for the paths, in other words the start category. Wikipedia contains some natural categories that are better to use as start category. These categories are very general and have links to some of the major categories within different fields, hence, able to reach most other categories in the Wikipedia category structure. The category \emph{Main Topic Classifiers} was chosen for this task, because it has 22 subcategories within various fields and  where all of them have their own subcategories (see figure \ref{fig:mainclassifiers})\cite{wiki:specialtree}.

\begin{figure}[h]
\begin{center}
\includegraphics[width=0.48\textwidth]{Chapters/Implementation/Maintopicclassifiers.png}
\end{center}
\caption[Subcategories of \emph{Main Topic Classifiers}]{Shows the first subcategories of the chosen start category \emph{Main Topic Classifiers}. \emph{C} corresponds the the category's subcategories and \emph{P} corresponds to its pages. The figure is provided by Wikipedia's Category Tree \cite{wiki:specialtree}.}
\vspace{-20pt}
\label{fig:mainclassifiers}
\end{figure}

%The category \emph{Main Topic Classifiers} has a large variety in its subcategories which makes it possible to reach categories within many topics. 



\subsection{Irrelevant Articles and Categories}
The next step is to remove all articles that are not relevant. Some articles does not provide information and should therefore be removed from our structure to reduce number of links that have to be considered at all time. Ideally we only want to consider article titles that can provide useful information. Articles about numbers are an example of articles that does not provide any new information and can be removed. 

The full paths for an article can also be quite long, hence it is useful to reduce the complexity by removing category titles from the path that are not useful. The main reason to remove a category title from the path is if it is too specified, for instance categories that ... : 


%Some of the articles not relevant are articles which are numbers. Numbers can have many meanings, but the meaning in the Wikipedia article does not give any new information when the number is available as an entry in a dictionary. 

%\subsection{Categories not relevant for the path}
%The structure of the categories in Wikipedia are very detailed, which make many of the paths too specified for our task. To simplify the article paths are some categories therefore removed from the path. 

%The categories which where chosen to be removed where: 
\begin{itemize}
\item ... are numbers
\item ... contains number
\item ... contains the word \emph{by}
\end{itemize}

The reason to remove all categories that are or contains numbers are that they usually are connected to a specific year, which is not interesting in our case. Categories containing the words \emph{by} can usually be removed because they are a parent category for sorting categories and usually indicate what the categories are sorted by. 
%Categories containing the word \emph{by} can also be removed because the  category is usually  placed under both of the categories it represents. 
An example of this can be seen in figure \ref{fig:galileogalilei} where one of the paths found for the Italian mathematician \emph{Galileo Galilei} can be simplified. 
%which is placed under the category \emph{Italian Mathematicians by century}.

%Reducing the complexity is useful to make the paths more readable. Example of this can be showed in figure \ref{} where 

\begin{figure}
\centering
\begin{lstlisting}
/mathematics/mathematicians/italian mathematicians/italian mathematicians by century/
16th-century Italian mathematicians/galileo galilei
\end{lstlisting}
\begin{lstlisting}
/mathematics/mathematicians/italian mathematicians/galileo galilei
\end{lstlisting}

\caption[Simplification of an article path]{Simplification of one of the paths of the article about Galileo Galilei}
\label{fig:galileogalilei}
\end{figure}

%\section{Wikipedia Structure}
There are two ways of accessing Wikipedia's encyclopedic information. The first way is to look up runtime as most users do when they are looking for information. The other way is more common when the information is used by other programs and

%s, which means that the program access Wikipedia's 

The other, and most common way is to download database dumps from Wikipedia. All Wikipedia articles, images and categories are stored in a database which are accessed when a user are searching for an article online. A database dump is therefore a backup of the database which are usually stored in the case of some data is lost. \footnote{TODO: reference: en.wikipedia.ort/wiki/Database\_dump} This backup is available for anyone interested at \emph{TODO: insert link}\footnote{TODO: insert reference?}. 




A database dump is defined as the table structures which is used to get the information to load

For our purpose there are some dumps that are relevant: 


I have chosen to work on the English Wikipedia, which is the largest database in with *** articles and *** pages. 

The relevant files are: 
\begin{itemize}
\item \enwikicatlink
\item \enwikipage
\item \enwikicategory
\end{itemize}

All of these files are compressed sql-files, which means that they represent files to put information into a SQL-database. Each of the files can be used to build up a database table with insert-statements, so all the information is stored in the table. 

\enwikicatlink describes all the links between categories, which describes two different types of relationships in Wikipedia. The first relationship is between a Wikipedia article and a category, i.e the category *** points to the article. The other relationship is between two categories which means that one of the categories is a subcategory of the other category. 

The file contains the table "categorylinks". Since the file is quite large (1.5GB compressed), it is desirable to split the file into two files; files containing information of the relationships between categories and files that contain information about the relationship between categories and pages. 

%\input{Chapters/Implementation/Parsing_the_Dumps}

\section{Redirects}
%Wikipedia contains lots of redirects between articles to help the users find the articles they look for, and to keep the encyclopedia well-structured.  
Wikipedia contains lots of redirects to articles for two main reasons. The first is to help users find the articles they are looking for, and the second reason is to keep the encyclopedia well-structured. The redirects are divided into different types depending on the reason for redirecting. Wikipedia lists all the different reasons of redirects. \cite{wiki:redirect} 
\begin{itemize}[noitemsep]
\item[-] Alternative names 
\item[-] Plurals 
\item[-] Closely related words 
\item[-] Adjectives/Adverbs point to noun forms 
\item[-] Less specific forms of names, for which the article subject is still the primary topic. 
\item[-] More specific forms of names 
\item[-] Abbreviations and initialisms 
\item[-] Alternative spellings or punctuation
\item[-] Punctuation issues—titles containing dashes should have redirects using hyphens.
\item[-] Representations using ASCII characters, that is, common transliterations 
\item[-] Likely misspellings
\item[-] Likely alternative capitalizations 
\item[-] To comply with the maintenance of nontrivial edit history
\item[-] Sub-topics or other topics which are described or listed within a wider article
\item[-] Redirects to disambiguation pages which do not contain "(disambiguation)" in the title
\item[-] Shortcuts
\item[-] Old-style CamelCase links 
\item[-] Links auto-generated from Exif information 
\item[-] Finding what links to a section, when links are made to the redirect rather than the section.
\end{itemize}

Most of the redirect pages are places in categories which tells the reason for the redirection. The category types for the classification are \emph{Maintenance}, \emph{Visual} or \emph{Discussion}.

%When a page is marked as a redirect page in 
\subsubsection{Handling redirects}

If a page is supposed to be redirected to another page, this is found in \texttt{enwiki- latest-page.sql.gz} where the page's \texttt{INSERT} statement is marked with '1' in the 6th position if it's redirecting (see figure \ref{fig:isredirect}).

\begin{figure}[h]
\centering
\begin{lstlisting}
(10,0,'AccessibleComputing','',0,1,0,0.33167112649574004,
'20150111235554','20150112004211',631144794,69,NULL)
\end{lstlisting}
\caption{Example of a redirecting \texttt{INSERT} statement}
\label{fig:isredirect}
\end{figure}
The first attempt of handling redirects is to make sure the paths are found to articles with correct names and not to the redirecting pages. Finding all paths to pages that redirect to other pages is unnecessary and creates more data than needed. It is instead better to find all pages that other pages redirect \emph{to}. This can be found in a separate file \enwikiredirect where both page id and page title for all pages are found. After all of these ids and titles are collected, the next step is to connect them with the correct output. As an example would the article from figure \ref{fig:isredirect} be connected to the page title in figure \ref{fig:correctacccomp} after the title is converted to lowercase and underscores are removed. 

\begin{figure}[h]
\centering
\begin{lstlisting}
(10,0,'Computer_accessibility','','')
\end{lstlisting}
\caption[Example of a page redirecting to]{The page title \emph{AccessibleComputing} (figure \ref{fig:isredirect}) redirect to \emph{Computer Accessibility}.}
\label{fig:correctacccomp}
\end{figure}

%Trenger at alle artiklene er lagret med de riktige navnene! 


%
%The next step in our problem is to decide which of the redirects that are relevant. 
%If the page is redirecting 
%All redirect pages are found in a separate file 

%Most redirect pages are not placed in article categories. There are three types of redirect categorization that are helpful and useful:

%    Maintenance categories are in use for particular types of redirects, such as Category:Redirects from initialisms, in which a redirect page may be sorted using the {{R from initialism}} template. One major use of these categories is to determine which redirects are fit for inclusion in a printed subset of Wikipedia. See Wikipedia:Template messages/Redirect pages for a full alphabetical list of these templates. A brief functional list of redirect category (Rcat) templates is found at {{R template index}}.
%    Sometimes a redirect is placed in an article category because the form of the redirected title is more appropriate to the context of that category, e.g. Shirley Temple Black. (Redirects appear in italics in category listings.)
%    Discussion pages. If a discussion/talk page exists for a redirect, please ensure (1) that the talk page's projects are all tagged with the "class=Redirect" parameter and (2) that the talk page is tagged at the TOP with the {{talk page of redirect}} template. If the discussion page is a redirect, then it can also be tagged with appropriate Rcats.


\section{Id Mapping}
When writing all the results to file, the files are extremely large. All path to all Wikipedia articles ended with about 20 GB compressed data of text. The result was to create a id mapping for each category name and article name to reduce space needed for storing all results on the computer. Id mapping gives all names an unique id, and instead of writing the full path to file, the ids of the full path is written to file.

The id mapping is created fixed by creating a counter that assigns numbers to each category name or article name that is not found yet. When a new category name or article name is found, it is assigned an unique number that represents the name. Figure \ref{fig:idmapping} shows an excerpt of the id mapping created for our purpose, where the id 4600570 corresponds to the article about \emph{Ole-Johan Dahl}, which means that this id is used everywhere \emph{Ole-Johan Dahl} is used in paths. 

\begin{figure}[h]
\centering
\begin{lstlisting}
...
4600566 roger matthews
4600567 pesticide drift
4600568 roxy theatre (clarksville, tennessee)
4600569 papadindar
4600570 ole-johan dahl
4600571 red square (university of washington)
...
\end{lstlisting}
\caption[Id mapping example]{Excerpt of the id mapping between id and the name of all categories and articles.}
\label{fig:idmapping}
\end{figure}

One of the reasons to work with ids instead of the full names is the memory needed on the computer. Lots of memory is needed if the full paths are represented by names, but less memory is needed if the full paths are represented by ids since the ids are shorter than the names of the category and article names. 

Working with ids is also a lot faster when considering lookups in the program. This will depend on the structure chosen for the programs, but when using dictionaries as done in our implementation, ids will perform faster than if using full names. An example of this can be seen in figure \ref{fig:id_lookup} where the time to find all categories from the category with id 177678 (corresponding to the category \emph{people}) is 0.955 minutes. Figure \ref{fig:fullname_lookup} shows the time needed to find the same paths for the category when using full names for categories and articles, which is found to be 1.559 minutes. Comparing the times shows that the time is a lot faster when using ids, which is important when many paths have to found.

\begin{figure}[h]
\centering
\begin{lstlisting}
[INFO] Finding all article paths from 177678

[INFO] Time to find all article paths: 0.955 min
\end{lstlisting}
\caption[Time for all paths for \emph{people} when using ids]{Time needed for finding all paths from the category 177678 (corresponding to the category \emph{people}) when ids are used in the program.}
\label{fig:id_lookup}
\end{figure}


\begin{figure}[h]
\centering
\begin{lstlisting}
[INFO] Finding all article paths from people

[INFO] Time to find all article paths: 1.559 min
\end{lstlisting}
\caption[Time for all paths for \emph{people} when using full names]{Time needed for finding all paths from the category people when using full names).}
\label{fig:fullname_lookup}
\end{figure}
 
The last reason to use ids instead of full names is that the full names may include characters useful for describing paths, for instance the characters "/" which is a common way of describing full paths. 

% Fordel 2: Kan bruke "/" in the text. 



\section{Grading of Categories}

\begin{comment}
The structure of Wikipedia could be considered confusing since anyone can edit. This means that the underlying category structure of Wikipedia contains lots of links between all categories. 

This means that it is possible to reach almost all articles from each category. 

This means that there are categories that reach lots of other categories. These should not be considered as important as the other categories. A program was made to find these categories. 

There are 28 top categories (direct subcategories of %\emph{Main Topic Classifications}). 

The main assumption is that if a category leads to many of the top categories, it is possible to reach lots of articles which are not associated with the category. 

%\begin{code}
Eksempel på hvordan kategorier finner artikler som ikke har noen sammenheng med kategorien.
%\end{code}

If a category leads to many of these 
The top categories (28) leads to lots of subcategories. 

% Fan
Another way of finding categories that does not provide information about the path, is to find all categories with many parent categories and with many subcategories since this means that they easily can reach categories not relevant for the category. 

Hence a program was made to find the number of parent categories and subcategories for each category. 
\end{comment}

Many articles can be reached from categories that are not describing of the content at all, for instance is the article about \emph{Ole-Johan Dahl} (the Norwegian programmer) found from links from the category \emph{people}, but also found from links from the categories \emph{politics} and \emph{arts} (see Figure \ref{fig:olejohandahl_paths}). When all paths are found for all Wikipedia articles, the next step was to grade each path depending on how well they describe the article. 

%grade the paths to find the paths most helpful for describing the article's content. 

\begin{figure}[h]
\centering
\begin{lstlisting}
ole-johan dahl:
*people/people categories by parameter/categories by nationality/academics by nationality/norwegian academics/faculty by university or college in norway/university of oslo faculty

[...]

*politics/political activism/leadership/management/quality/software quality/formal methods/formal methods people

[...]

*arts/aesthetics/design/software design/data modeling/formal methods/formal methods people

\end{lstlisting}
\caption[Example of variety in article paths]{Some of the paths for the article about \emph{Ole-Johan Dahl}.}
\label{fig:olejohandahl_paths}
\end{figure}

\input{Chapters/Implementation/Grading/In_out_grading}
\subsection{Grading based on Inlinks and Outlinks} % eller: Grading based on inlink and outlinks 
Our first assumption is that categories with high a inlink number can be reached from categories with different topics. An example of a category with a high inlink number is  \emph{World War II}. This category can be reached from 87 different categories (see figure \ref{fig:high_inlink_number}). 

%great rift valley: 5, 31
%world war ii
%87
%['1940s conflicts', 'wars involving saudi arabia', 'wars involving austria', 'wars involving estonia', 'wars involving costa rica', 'wars involving the republic of china', 'wars involving panama', 'wars involving peru', 'wars involving vietnam', 'the world wars', 'wars involving the netherlands', 'wars involving syria', 'wars involving ethiopia', 'wars involving iraq', 'wars involving norway', 'wars involving nicaragua', 'conflicts in 1941', 'conflicts in 1940', 'conflicts in 1943', 'conflicts in 1942', 'conflicts in 1945', 'conflicts in 1944', 'wars involving liberia', 'wars involving ukraine', 'wars involving egypt', 'wars involving nepal', 'wars involving bolivia', 'wars involving san marino', 'wars involving albania', 'wars involving the philippines', 'wars involving iran', 'commons category wikidata tracking categories', 'wars involving italy', '1930s conflicts', 'wars involving the united states', 'wars involving british india', 'wars involving poland', 'wars involving el salvador', 'global conflicts', 'wars involving the soviet union', 'wars involving luxembourg', 'wars involving venezuela', 'wars involving new zealand', 'wars involving brazil', 'wars involving mexico', '20th-century conflicts', 'wars involving cuba', 'wars involving france', 'wars involving hungary', 'wars involving indonesia', 'wars involving greece', 'wars involving colombia', 'wars involving argentina', 'wars involving ecuador', 'wars involving czechoslovakia', 'modern europe', 'wars involving lebanon', 'wars involving turkey', 'wars involving haiti', 'wars involving chile', 'wars involving thailand', 'wars involving yugoslavia', 'wars involving korea', 'wars involving lithuania', 'wars involving mongolia', 'wars involving cambodia', 'wars involving the dominican republic', 'wars involving bulgaria', 'wars involving belgium', 'wars involving japan', 'wars involving laos', 'wars involving finland', 'wars involving australia', 'wars involving canada', 'wars involving paraguay', 'wars involving uruguay', 'wars involving romania', 'wars involving belarus', 'wars involving guatemala', 'wars involving the united kingdom', 'conflicts in 1939', 'wars involving burma', 'wikipedia category maintenance', 'wars involving south africa', 'wars involving honduras', 'wars involving denmark', 'wars involving germany']

\begin{figure}[h]
\centering
\includegraphics[width=\textwidth]{Chapters/Implementation/Grading/high_inlink_number}
\caption[Example of category with high \emph{inlink number}]{All categories linking to the category \emph{World War II}. This is an example of a category with high inlink number.}
\label{fig:high_inlink_number}
\end{figure}


\begin{comment}
\begin{figure}[h]
\centering
\begin{lstlisting}
ole-johan dahl:
*politics/political activism/leadership/management/quality/software quality/formal methods/formal methods people
\end{lstlisting}
\caption{Example of how \emph{politics} can reach the article about \emph{Ole-Johan Dahl}}
\label{fig:politicstosoftware}
\end{figure}
\end{comment}
%quality: 9, 3
%management: 31, 5
The next assumption is that categories with a high outlink number are more likely to reach categories not relevant since they can reach far in all the subcategories' directions. Figure \ref{fig:high_outlink_number} illustrates number of subcategories found for the category with highest outlink number, which is the category \emph{Albums by artist} and a outlink number of 17 393. 

%Figure \ref{fig:politicstosoftware} shows how the Wikipedia article about \emph{Ole-Johan Dahl} can be reached from the category \emph{politics}. One of the categories with a high \emph{outlink} is the category \emph{management}, which has \emph{outlink} as 31 and hence be reached many categories. 

\begin{figure}[h]
\centering
\includegraphics[width=0.7\textwidth]{Chapters/Implementation/Grading/high_outlink_number}
\caption[Example of category with high \emph{outlink number}]{The category \emph{Albums by artist} is an example of category with high outlink number. }
\label{fig:high_outlink_number}
\end{figure}


%\subsubsection{Grading based on Inlinks and Outlinks}
These assumption combined are the foundation of grading based on inlink and oulink numbers. Categories frequently reached should obtain a lower score than categories rarely reached. We need some way of deciding whether an inlink number is high for each category. This can be done by comparing the inlink number with the average inlink number, and similarly for outlink number and the average outlink number. 

%Thus, our first approach was to find the \emph{inlink} and \emph{outlink} of all categories in the structure. 

The average inlink and outlink numbers are found by summarizing all inlink numbers and outlink number respecitvely, and dividing the result on number of categories. Table \ref{tab:avginlinkoutlink} shows the tesults found. 

%numbers had to be compared with the average number of \emph{inlink} and \emph{outlink} to know whether the number is high or low (see Table \ref{tab:avginlinkoutlink}). 


\begin{table}[ht]
\centering
\renewcommand{\arraystretch}{1.25}
\begin{tabularx}{\textwidth}{c |c}
\textbf{Average \emph{inlink number}} & \textbf{Average \emph{outlink number}}\\ \hline
 5 & 2 \\
\end{tabularx}
\\[10pt]
\caption{Average inlink number and outlink number for all categories.}
\label{tab:avginlinkoutlink}
\end{table}
The score for each category was found by comparing its inlink number and outlink number by the average inlink and outlink number, i.e. 

\begin{equation} \label{eq:scoreinout}
Score_{C} = \frac{inlink_{c} + outlink_{c}}{\bar{C_{in}} + \bar{C_{out}}}
\end{equation}
where $\bar{C_{in}}$ is the average \emph{inlink} and $\bar{C_{out}}$ is the average \emph{outlink}.

%The scoring from formula \ref{eq:scoreinout} means that paths with categories rarely visited will be favoured, hence given a lower score. 

\subsubsection{Evaluation of the scores}
None of the categories can have a score of 0 since all Wikipedia categories are connected to at least one other category \footnote{We mentioned in challenges (Introduction, encoding)that some of our connections were broken. This does not affect the scoring of the categories, since the inlink and outlink numbers are preserved for all categories. }. The lowest score found was 0.376010, which was given to all categories with only one parent category and with none subcategories. This was a total of 104 471 categories.  The category with the highest score is the category \emph{Albums by Artist}, which is the category with most subcategories (17 393), hence a score of $6 512.120784$. The range of the scores is \emph{<0.376010, 6 512.120784>}, which means that all scores is within this range. 

Figure \ref{fig:scorevalue} shows how many categories are found for each of the possible score values. The figure shows that there are many categories with low score values, while there are only a few categories for higher score values. The categories with high score value will have a high impact on the article path, paths containing these categories will have a lower probability of be considered relevant. 


\begin{figure}[h]
\centering
\includegraphics[width=\textwidth]{Chapters/Implementation/Grading/Inlinkoutlink_scorevalue_numberofcategories}
\caption{Number of categories for each possible score value}
\label{fig:scorevalue}
\end{figure}

%This means that the score for all categories are between 0.376010 and 6512.120784.

%This means that the scores for all categories are in the range of 0 and

%Maxgrade: 6512.120784 (albums by artist)
%Mingrade: 0.376010 (user bho-4)
\input{Chapters/Implementation/Grading/Problems_inlink_outlink}
\subsection{Normalized Grading Based on Inlinks and Outlinks}
A way of avoiding favourization of short paths is by normalizing the path scores. 
%Grading based on inlink number and outlink number favors short paths even if the paths contains categories considered as bad. One way of handling this problem is by normalizing the score of each path. 
Equation \ref{eq:normscoreinput} shows how to normalizes the path score for path $P$ so the length of the path does not determine the relevance of the path. 

% TODO: Write something about normalization - why is it good for grading?

\begin{equation} \label{eq:normscoreinput}
Pathscore_{P} = \frac{1}{N} \sum_{c} Score_{C}
\end{equation}
where $N$ is the number of categories in the path.

Figure \ref{fig:norm_alexander_hughes} shows the three best results for the same article (Alexander Hughes) when the paths are normalized. The results here are more descriptive of the content of the article, where all paths contains information that he is associated with football. 

\begin{figure}
\centering
\begin{lstlisting}
alexander hughes:
* health/health by city/health in edinburgh/sport in edinburgh/sports teams in edinburgh/football clubs in edinburgh/heart of midlothian f.c./heart of midlothian f.c. players/ (4.431941375)
* concepts/principles/rules/sports rules and regulations/sports terminology/association football terminology/association football positions/association football players by position/association football defenders/ (5.01043655556)
* sports/sports terminology/association football terminology/association football positions/association football players by position/association football defenders/ (6.08136966667)

\end{lstlisting}
\caption[Example of normalized scores on paths]{The three best paths for \emph{Alexander Hughes} when the path scores are normalized. }
\label{fig:norm_alexander_hughes}
\end{figure}


\begin{comment}
\subsection{Deciding Relevant Paths}
One way of deciding which graded paths are relevant are by choosing a threshold for the path score. If the score is lower than a given threshold, it is marked as relevant, while a  higher score means that it is not relevant. A threshold can be found by deciding how many paths should be considered relevant.

One way of doing this is by finding the scores of all paths. and sort the scores from lowest to highest (see \ref{eq:sortedscores}). Then a $k$ has to be decided to how many paths are believed to be relevant of all paths, for instance one could assume that only 10\% of the paths are relevant, which leads to $k = .10 \cdot n$. 

\begin{equation} \label{eq:sortedscores}
Sorted\_scores = \left[ S_{1}, S_{2}, ... , S_{k}, ... , S_{n} \right]
\end{equation}



\begin{equation} \label{eq:threshold}
T = Sorted\_scores[k]
\end{equation}


The problem with this method is that not all articles are guaranteed to have any relevant paths. The other problem is that the score of the path will vary a lot within different fields, since some of the Wikipedia articles are categorized under very specified categories. 
% TODO: Finn en kilde som er enig med meg. 

% Problem: 
% Finne hvor mange pather som er tilgjengelig. 

Another approach is to choose the best $k$ paths for each Wikipedia article. This approach is independent of the values on other articles' path score which means all Wikipedia articles are guaranteed at least one path. The disadvantage is that some paths might be marked as relevant even though their path score is lower than path scores marked as irrelevant by other articles. Another disadvantage is that articles with many good paths will still have to choose the best $k$ paths and good paths might be lost. 

\end{comment}

\begin{comment}
Fordeler: ser ikke på de andre
alle articler får minst en score. 

Ulemper: mange gode - hvilken er best?
Kan ikke vite om scoren er god

\end{comment}
%\input{Chapters/Implementation/Grading/Disambiguation_grading}

\section{Mapping to Desirable Output Categories}
Our goal for the mapping process is to create a link between Wikipedia article titles and one or more categories from the desirable output categories. It is essential to know the meaning of the WIkipedia articles in order to create such a mapping. Our theory is that this information can be found in the full paths of the articles, where a 
%, the next step is to create a link between all Wikipedia categories and the desirable categories. 
full path of a Wikipedia article contains the categories visited to reach the article. This means that the  machine needs some predefined knowledge to identify the meaning of the paths. Two approaches were tried for this task; creating a mapping between Wikipedia categories and output categories, and creating mapping between path excerpts and output categories. 

%the first was to create a mapping between the Wikipedia categories and a category from the set of output categories and the second approach was to 
%knowing the meaning of the categories is important to understand the meaning of the article.

%\subsection{Mapping Wikipedia Categories to Desirable Output Categories}
\subsection{Deciding Output Categories based on Wikipedia Categories}
%The first step is to decide the output categories. %in other word what categories
The first approach was to create a mapping between each Wikipedia category and one or more categories in the desirable output category set. The idea was that a matching could be performed by matching Wikipedia category names and a output category name. The task of mapping each Wikiepedia category to desirable output categories is too big to be done manually since the Wikipedia category set contains 1 201 373 categories. This means that the process should be automated. One way of doing this is by looking at similarities in the words contained in the Wikipedia category and in the output category.

\subsubsection{Expanding the IAB category}
The categories in the IAB taxonomy were chosen as the desirable output category set for our task. This taxonomy  only consists of two category layers, which are not specified enough for creating a matching based on the category names. Hence the IAB taxonomy was extended with a third and more specified layer to improve the category mapping process. 

%was added to the taxonomy to 
%had to be created for this task, where this layer is specified 
%we want to classify all the Wikipedia categories to. The categories in the IAB taxonomy is not specified enough to categorize all the categories, hence it is necessary to add a third layer to the taxonomy. This layer has to be more specified to make it easier to categorize all the Wikipedia categories. 
%The next layer has to be modified to 

%fit the set of Wikipedia categories, and to be helpful for categorizing correct. 

This third layer can be viewed as common knowledge given to the machine. The second layer \emph{Europe} is an example of a layer where the machine lacks common knowledge since it does not know what countries are part of Europe. Expansion of this tier could be creating a third tier containing all European countries, which means that all Wikipedia categories containing a name of an European country should map to the category \emph{Europe}.

%An example of an expansion to the second layer \emph{Europe} is to add all European countries to its third layer since the machine lacks common knowledge about what countries 

%Such a layer can be viewed as giving the machine common knowledge. An example of 

%One of the second layers in the IAB taxonomy is \emph{Europe} under the first layer \emph{Travel}. The computer lacks common knowledge about what countries are in Europe, hence some information has to be provided to this layer so it can recognize countries in Europe. One way of doing this is by adding all European countries to a third layer under the category \emph{Europe}. 

%Since the task is categorization of Wikipedia, knowledge has been provided from other sources. List of all countries where found from \url{http://www.internetworldstats.com/list1.htm}. 

\subsubsection{Lemmatization}
%The set cof Wikipedia categories contains of 
%Our set of Wikipedia categories contains 1 201 373 categories. 

Figure \ref{fig:catmapping_exactmatch} shows how a matching between Wikipedia categories and output categories, where the output category name \emph{sports} are found as a word in the Wikipedia category name \emph{ministry of yougth affaris and sports}.

\begin{figure}[h]
\centering
\begin{lstlisting}
ministry of youth affairs and sports
sports
\end{lstlisting}
\caption[Exact match on mapping between Wikipedia category and output category]{Exact match on mapping between Wikipedia category and output category, where the output category is found in the Wikipedia category.}
\label{fig:catmapping_exactmatch}
\end{figure}
The problem with this approach is that words like \emph{sport} will not be an exact match of the word \emph{sports}, hence this Wikipedia category will not be included under the desirable output category. The next step is therefore to find matches between the categories regardless of the declension of the word. This part is called lemmatization and is defined as the process where different inflected forms of a word are grouped together \cite{wiki:lemmatisation}\cite[p.~30-33]{iirbook}. There are various lists for lemmatization available online, and a list was chosen from  \url{http://www.lexiconista.com/datasets/lemmatization/} which provided a list of common lemmatization. Both the words in the Wikipedia categories and the desirable output categories were processed by reading the lemmatization file and checked whether the words could be reduced. Figure \ref{fig:catmapping_lemmamatch} shows example of a match found after lemmatization is performed. 

\begin{figure}[h]
\centering
\begin{lstlisting}
sailors at the 1956 summer olympics
*olympics
*sailing
\end{lstlisting}
\caption[Example of match after lemmatization]{Example of a match between Wikipedia category and output category after lemmatization, where \emph{sailors} match with \emph{sailing}}
\label{fig:catmapping_lemmamatch}
\end{figure}

\begin{comment}
\subsubsection{Categories not relevant for classification}
Not all categories are suitable for classification, some categories are still just relevant for maintaining a well-structured encyclopedia. Example of such categories are \emph{container categories}, which are categories only containing subcategories. All container categories where found by looking at the file asdfasdf  . Some of these categories have already been removed because they are also hidden categories, but a total of 69 023 categories could be disregarded for this purpose. 
%Lots of categories are associated with years, but not 
%The next step was to mark all categories associated with years. These categories usually are  
\end{comment}

\subsubsection{Evaluation of Mapping Wikipedia Categories to Output Categories}
The results from this approach were not so good for two main reasons. 
%The results from this approach from this approach were not good for two main reasons: 
The first reason is that it is difficult to perform matching based on words. A perfect result could only be achieved if the computer knows all synonyms, inflections and the true meaning of all words. 

The other problem was with ambiguous words in the category names. An example of this is the categories shown in figure \ref{fig:ambiguous_category_name} where both categories contain the word \emph{Cicero}, but where the first category is for the suburb of Illinois and the other is for the Roman philosopher. Creating mapping rules for these names would be a difficult task. 

\begin{figure}[h]
\centering
\begin{lstlisting}
Category:Cicero, Illinois
Category:Cicero
\end{lstlisting}
\caption{Example of two category names which contains the same word with different meaning, and should be classified to different categories.}
\label{fig:ambiguous_category_name}
\end{figure}


The conclusion for this approach is that it might be possible to create a mapping between each Wikipedia Category and one or more desirable output categories, but this would need a very specified third tier in the IAB category and lots of rules. The task would therefore resemble a manual classification and is not a good approach. 
%-> It is impossible to create a third tier to satisfy this. 


\subsection{Mapping based on Wikipedia Path Excerpts}
\label{sec:mapping_based_on_wikipedia_path_excerpts}
The other attempt was built on the idea that a the mapping from Wikipedia category and output categories needs more information about the Wikipedia categories. The idea is that this information could be found in excerpts of articles' full paths. Thus, the mapping process is based on excerpts of the paths, which should be mapped to one or more output categories. This approach solves the problem with ambiguous category names, because we specify the meaning of the category name in the path excerpt (see figure \ref{fig:solving_disambiguation})

\begin{figure}
\centering
\begin{lstlisting}
ancient philosophers/cicero
towns in illinois/cicero,illinois
\end{lstlisting}
\caption[Avoiding disambiguation with excerpts of category paths]{How disambiguation can be solved if parts of the full path is used to determine the meaning of the category name.}
\label{fig:solving_disambiguation}
\end{figure}

\subsection{Automatic Mapping}
We started out by manually creating mappings between path excerpts and IAB categories, but this is a large task sins there exists so many categories and category links in Wikipedia's structure. Thus, it is desirable to automate the mapping process between.

We tried to find a good way to predict matches between the excerpts and the output categories. We assumed that the IAB subcategory name (e.e., \emph{Auto parts}) is a category, and wanted to find the most likely categories leading to this category. This was done by finding all categories leading to this category among the top 3 category paths for each Wikipedia title, and counting the occurrences. All patch excerpts among the 10 most common were chosen if they seemed logical. 




\begin{comment}

To see if the automated categorization process were successful, these results had to be compared to a manual categorization. We tested our 

These results were compared to a manual categorization for the same

Our conclusion was that it was still necessary to 


Our mapping process between path excerpts of Wikipedia categories and IAB categories has to be controlled by humans. It is desirable to make this task as automatic as possible. 

The mapping process is not completely automated since the mapping between path excerpts of Wikipedia categories and IAB categories has to be evaluated by humans. 

\begin{comment}
INSERT EXAMPLE ABOUT WALKING HERE. 
\end{comment}



\begin{comment}


Alle:
[INFO] Total number of articles found: 152 664/4690240
[INFO] books & literature: 189169 articles



Kun 6 kategorier etter: 33 104/4690240
[INFO] books & literature: 51919 articles

-> len(cats_at_end) > 5: 

Kun 4 kategorier etter: 



\end{comment}
\subsection{Processing Titles}
A match in a random article is found if a phrase or word is an exact match with a Wikipedia article title, hence the Wikipedia article titles can be viewed as entries in a dictionary. The titles should therefore be processed to make sure that matches will be found. 

\subsubsection{Disambiguation or Specification of titles}
Lots of the Wikipedia titles contains parenthesis that specify what the Wikiepdia article is about. Figure \ref{fig:parenthesis_example} shows two Wikipedia titles for the \emph{David Sharpe}, where one article is about David Sharpe (1967-) the British athlete  \cite{wiki:davidsharpeathlete} and the other is about David Sharpe (1910-1980) the American actor  \cite{wiki:davidshapreactor}.

%\emph{General Grant}, where one article is about the largest giant sequoia \cite{wiki:generalgranttree} and the other is about a 1,005-ton ship built in 1864 \cite{wiki:generalgrantship}. A match will mostly likely occur without the parenthesis, so these has to be removed from the entries. 

%david sharpe: david sharpe (athlete) [['sports/running&jogging']]david sharpe (actor) [['arts & entertainment/movies']]

\begin{figure}[h]
\centering
\begin{lstlisting}
david sharpe (athlete)
david sharpe (actor)
\end{lstlisting}
\caption{Wikipedia article titles with parenthesis}
\label{fig:parenthesis_example}
\end{figure}

Lots of Wikipedia articles are about events happening a specific year. Exact matching with these titles will most likely occur, hence the year should be removed from the entry.  Figure \ref{fig:davis_cups} shows an example of two entries which corresponds to the Davis tennis tournaments in 1996 and in 2000. Removing the year from these entries will increase the probability of finding a match, but also make both entries look the same. 

\begin{figure}[h]
\centering
\begin{lstlisting}
Wikipedia article title: 1996 Davis Cup
Wikipedia article title: 2000 Davis Cup
\end{lstlisting}
\caption{Wikipedia article titles which will look the same when removing the year from the title.}
\label{fig:davis_cups}
\end{figure}

% Skrive noe om mens/womens?
Another specification found in Wikipedia articles is specification on gender, like \emph{2015 Dubai Tennis Championships – Women's Singles} a figure \ref{fig:dubai_gender} shows. This specification reduces the probability of an exact match, hence \emph{women's} and \emph{men's} are removed from the title and reduces it to a more general form which are more likely to occur. 

\begin{figure}[h]
\centering
\begin{lstlisting}
2015 Dubai Tennis Championships
2015 Dubai Tennis Championships - Women's Singles
2015 Dubai Tennis Championships - Men's Singles
\end{lstlisting}
\caption{Wikipedia articles specified for gender (women and men) and gender neutral.}
\label{fig:dubai_gender}
\end{figure}

The next step is to decide whether the modified entries mean the same or have different meaning after the parenthesis and years are removed. This was done by looking at the mapping of the entries. Two processed entries are considered identical if they are a match of each other and are mapped to the same category. One of the entries is kept if the entries are identical, both are disregarded otherwise. The entry \emph{David Sharpe} (figure \ref{fig:parenthesis_example}) is an example of an entry that is removed from the dictionary since the two original entries are mapped to different categories , while \emph{Davis Cup} (figure \ref{fig:davis_cups}) is kept since both of the entries are mapped to the same categories. The gender specific entries in figure \ref{fig:dubai_gender} are reduced to one entry \emph{Dubai Tennis Championships - Singles} when gender and year is removed from the entry, and is kept in the dictionary.

There are both advantages and disadvantages with this approach. The main disadvantage is that entries are removed, hence, information is lost. We could still argue that the removed entries are the entries most likely to wrongly classify text, and that the probability to correctly classifiy text is increased when these entries are removed. 

%The advantage is on the other hand that the removed entries are the entries which most likely would lead to wrong information. 

\subsubsection{Removing common words}
Some of the entries are reduced to very common English word. Figure \ref{fig:common_word} shows that Wikipedia article title \emph{(85476) 1997 MY} (a main-belt minor planet \cite{wiki:myplanet}) are reduced to the entry \emph{my} (determiner: belonging of me). This means that the dictionary entry \emph{my} is categorized to the same as \emph{(85476) 1997 MY}, which is \emph{Astronomy\&Space}.

%an example of the entry  \emph{(85476) 1997 MY} (a main-belt minor planet \cite{wiki:myplanet}) which are reduced to the entry \emph{my} when parenthesis and years are removed from the entry. This means that the common word \emph{my} and  \emph{(85476) 1997 MY} is categorized to the same category, \emph{Astronomy\&Space}. 

\begin{figure}[h]
\centering
\begin{lstlisting}
Wikipedia article title: (85476) 1997 MY
Entry: my
\end{lstlisting}
\caption{Example of an entry that has been reduced to a common English word. }
\label{fig:common_word}
\end{figure}

Words that occur extremely often in most documents are more likely to disturb the categorization instead of providing useful information. These words should henceforth be disregarded as entries. This was done by creating a large list containing the most common English words, called a \emph{stop list} \cite[p.~ 25]{iirbook} . An entry is removed if it is reduced to one of these words. The stop list chosen for this implementation was chosen as the 1000 most basic English words according to Wictionary, combined with the 100 most common spoken words according to TV and movie scripts \cite{wiki:freqwordlist}.



% Important: det er viktig at noe mapper til alle ønskede output categorier. Det er ikke viktig at alla kategorier mapper til noe. 


%\section{Numbers}

Numbers occur everywhere in 

%

% About hidden categories. 
%Wikipedia’s category structure contains lots of hidden categories which are not displayed at the bottom of an article page for the general users, even if the article is placed under the category. The first subtask was therefore to remove all links to the hidden categories since they do not provide any relevant information about the article's content. 

%T%he task of finding all hidden categories was more complicated than first assumed. The first approach was to find all subcatogries of the category \emph{Hidden Categories} in the insert statement database dump. This turned out to [TODO:NUMBER]. 


%but is not all hidden subcategories in Wikipedia, because some o%
%Finding all of the subcategories of these turned out to be difficult since some of them included links to categories that should not be removed, 
%\begin{code}
%[ISERT EXAMPLE]
%\end{code}

%The next attempt was to look at the file \enwikipageprops, where each statement is on the following form: 

%\begin{code}
%[INSERT example]
%\end{code}
%The \enwikipageprops  contains information about the pages, ans has a field called \emph{pp\_propname} that give some description about the page. Hidden categories are therefore marked as \emph{hiddencat} (see %TODO: input figure here)
%and finding all hidden categories could be done by collecting all categories which page ids corresponding to those with the field \emph{pp\_propname} marked as \emph{hiddencat}. Collecting all the names gave a list of [NUMBER] hidden categories. 

%Since the two approaches gave different results, was the conclusion to combine the two lists to be sure that all hidden categories had been found, which resulted in a list containing [NUMBER] different hidden categories that could be ignored later in the programs.  


%The structure of Wikipedia is created so that there are more than one way to reach each article

%Since there are more than one category for each article are there also more than one full path for each article. An %example of a way of reaching the article about \emph{Galileo Galilei} could be given by the following path

%\begin{lstlisting}
%/mathematics/mathematicians/
%italian mathematicians/galileo galilei
%\end{lstlisting}

%This is not the only way to reach the article, another way could be given by%

%\begin{lstlisting}
%asdfasf
%\end{lstlisting}


%input{Chapters/Implementation/Div2}
\chapter{Results and Discussion}

\section{Evaluation}
The main purpose of the evaluation is to see whether there are any improvements when the results are applied. We assume that there are improvements if the classifier categorized correctly. This section describes how we validate the classifier's results 


It is difficult to evaluate improvements, the evaluation is based on the assumption that improvement might be achieved if the categorization results are correct. 

%so evaluation of the results are performed instead. 

% Assumption: The results are good if they are correct. 

Which mapping were easy? Which where difficult and why?

- 

\begin{code}

Her må jeg skrive noe om hvordan jeg deployer resultatene til Cxense. 

\end{code}

\begin{table}[ht]
\centering
\renewcommand{\arraystretch}{1.25}
\begin{tabularx}{\textwidth}{l |c|c}
 & \textbf{sport} (iabtaxonomy) & \textbf{not sport} (iabtaxonomy)\\ \hline
 \textbf{sport} (taxonomy) & 1058 & 919 \\ \hline
 \textbf{not sport} (taxonomy) & 10516 & 99895
\end{tabularx}
\\[10pt]
\caption{}
\label{tab:}
\end{table}


\begin{comment}
Date: 15/04/15

\end{comment}




It is natural to assume that there are improvements if the 

The most natural thing would be to assume that 


applying the categorization, but it is also interesting to evaluate the result of the classifier i.e., see whether it correctly assigns categories. Improvements can also be assumed to be better if the classifier has a high probability of categorize correctly.

%Evaluation of the improvement is an evaluation of the results of the user. 
%The evaluation should both cover evaluation t
%This evaluation can be thought of as two evaluation approaches; evaluation of the technical result and evaluation of the overall improvements when applying the classifier. Technical evaluation is 

%There are different parts of the result that can be evaluated,  but the most important evaluation is to evaluate the classifier to see whether it correctly assign categories. 

The evaluation of the classifier can be separated into different evaluations that together cover the whole categorization. 
%plit into different parts to get an evaluation of the different components of the classifier. 
The first evaluation could be of the predefined base components of the classifier; the keyword list and the set of categories. One way of evaluating the list of keywords is to determine if the keywords are relevant for the categorization. This could be done by looking at the size of the list i.e., how many words are included in the list, and what keywords are actually used (which occur in the collection of text). It could also be interesting to see if some of the words are never used. 

The set of categories could be evaluated as how many texts get categories to the the different categories, and try to evaluate if some of the categories seem unnecessary. The set of categories might also depend on the use of the categorization, which means that some categories might be unnecessary in some content analysis and useful in others. 

A more interesting evaluation is the function that decides what category a keyword maps to. It is not possible to do the mapping by hand since the program is operating with many thousand keywords, also in many languages, which means that the mapping has to be done automatically. The best evaluation of this function is comparing with  a true solution. Creating a true solution for the whole categorization, but there are two other approaches for evaluating the result. The first is to create a handmade solution for some small list and compare the classifier's result with this list. This will hopefully give some indication of the result of the classifier, but the result will vary a lot depending on what list we choose. 
%This should be compared by a list made by humans; what category should a keyword link to? The problem with this is that it is time consuming and difficult to make a list like this by hand. 
The other approach is to take advantage of Wikipedia's category structure. 
%
%of the list could therefore be based on Wikipedia's category structure instead. 
All articles are, as already mentioned, already categorized and it is therefore possible to compare the path 
%Since all titles are categorized is it possible to compare the path 
distances from the parent categories and to the most describing category to determine if the keyword is linking to the right categories.


The last part of the classifier's evaluation is deciding the overall result, i.e., how well does the classifier categorize the collection of texts? The best evaluation would again be to compare the classifier's result with a manual categorization and look if the results are the same. The problem with this approach is the same as with the mapping function, we need a true solution to compare with. A proposal to a result is the same; we could create a small set for comparing, but a problem with this solution is that it is a difficult task for comparing, since text can be difficult to categorize. Ole Johan Dahl could for instance be categorized under both \textit{Norwegian computer scientists} and \textit{computer scientist}, and both of them are correct. Such a comparison would therefore depend on using the same categories which can seem unnatural. It is also possible to compare the result of a classification with classifications of very similar texts to see if the categorization decides the same result. 

Some text collections can also be evaluated with help from the text itself. Lots of news articles are for instance already categorized in the URL, for instance would the URL of an article about sport contain  some information that it is about sport:\texttt{.../sport/...}. A possible solution is therefore to look the URL and see if it matches categories proposed by the classifier. 

\chapter{Conclusion and Further Work}
There are both improvements and desirable further work available for this study. 

%that could be added to this solution. 

\section{Conclusion}
Automatic content categorization is useful for building up user profiles and in the task of automatically decide advertisements on web pages. We chose to create a dictionary-based classifier because it is easy to understand for brokers (which are often non-technical) and because it is based on a dictionary that easily can be modified to satisfy specific purposes.  

Our classifier is based a dictionary where the entries are created from titles of Wikipedia articles. Each dictionary entry is connected to category from IAB's taxonomy, where we explored the underlying category structure of Wikipedia in order to create an automatic mapping between these. Our overall goal was to determine whether articles could be correctly categorized based on just the Wikipedia article titles and the underlying category structure. 

%by comparison with the url
We evaluated the classifier's results by comparing the results with url structures of articles. The sites used for the evaluation were \texttt{www.rappler.com} for the English classifier and \texttt{www.adressa.no} for the Norwegian classifier. The English classifier was evaluated with 3 categories: \emph{sports}, \emph{arts \& entertainment} and \emph{technology}.

\begin{comment}

The results of our projects shows that it is possible to determine the content of some articles just by exploring titles of Wikipedia articles. The English classifier was evaluated on 3 categories: 



The classification to these categories showed various results, where the best evaluation scores where found within different fields:
\begin{itemize}
\item best precision was found for \emph{sports} which means that this class has fewest wrongly classified articles, i.e., low number of \emph{FP}.
\item best recall was found for \emph{arts \& entertainment} which means that the classifier found most of the articles connected to this class, i.e., low number of \emph{FN}.
%\item best accuracy was found for \emph{tehnology
\end{itemize}

We wanted a trade-off between precision and recall in our classifier, which were found by using $F_{1}$-score. 


The trade-off between \emph{precision} and \emph{recall} is important to optimize the classifier, and a classifier with 


it is desirable to create a classifier which 

so that the classifier 

\end{comment}

We improved our classifier by creating new versions of its dictionary. The evaluation results showed that the later versions of the classifier were considerable better, i.e., higher evaluation score (as seen in table  \ref{tab:improved_f1} where we compared the $F_{1}$-score was higher when comparing version 3 and version 6). % where  we can see that the $F_{1}$-score is higher when comparing version 3 and version 6 for all three categories \footnote{We chose to compare version 3 and version 6 because all three categories where available for both versions.}.



The results of our classifier showed that it is possible to determine the content of some articles just by exploring titles of Wikipedia articles ad the underlying category structure. However, many articles were wrongly categorized when compared to the url structure. This might be because we developed a one-to-many classifier which means that the classifier can classify an article to more than one class, while the classification results are compared to a one-to-one classification where an article contains only one class within the url structure. We found several examples of articles that were considered wrongly classified by the evaluation scores, but considered correctly classified by us. 



%The improvements on the newer versions of the classifier shows that it classifies better in the later versions. 

We decided to compare the results of our English classifier with \cite{entityextraction}, because this classifier contained all three classes.  Comparison showed that the classifier in \cite{entityextraction} achieved higher evaluation scores than ours. However, it is important to notice that \cite{entityextraction} added knowledge in addition to Wikipedia, including \emph{MusicBrainz} which is most likely very helpful for optimizing the categorization of \emph{arts \& entertainment}. Even though the evaluation scores were higher, we could see that the classification results of \cite{entityextraction} shows similar results as ours; \emph{sports} were found to be easier to classify than \emph{arts \& entertainment} and \emph{technology}.


\begin{comment}
ir classifier had the same 
, where we can see that th

sports: 0.317558
arts: 0.159681
0.163934

Difference between 

\begin{table}[]
\centering
\renewcommand{\arraystretch}{1.25}
\begin{tabular}{|c}
 &  \\
 & 
\end{tabular}
\caption{Caption}
\label{tab:my_label}
\end{table}

\end{comment}

The creation of the Norwegian classifier was based on the simple idea. All English entries in the classifier's dictionary were translated to Norwegian by using the internal language links within Wikipedia. Finally, we removed all words and phrases that were ambiguous in Norwegian and this resulted in a small Norwegian dictionary which could be used by a classifier. Only two of the categories were available on \texttt{www.adressa.no}, so we evaluated \emph{sports} with \emph{sport} and \emph{arts \& entertainment} with \emph{kultur}.

The Norwegian classifier performed surprisingly well considering the simple approach for creation and that it contained few entries in its dictionary. However, an improvement of the classifier would be to add words or phrases that are distinctively Norwegian and not found in the English Wikipedia. %distinctively Norwegian words or phrases 


Finally, our conclusion is that it is possible to get reasonably good results from our classifiers just by exploring the titles of Wikipedia articles and the underlying category structure. The results of the classifier can be improved by modifying the dictionary it is based on, but the classifier is already able to give a good indication of the content of an article. 




\begin{comment}
\subsection{Summary (of essay)}
This paper has given a brief introduction to the automatic categorization problem used for content analysis. The main reason for automatic content analysis is that manual classification is impossible for large collections of text, since it is both time consuming and  depends on experts within the topic of the texts. Automatic content analysis is based on the idea that the computer understands texts by recognizing specific keywords that  connected to one or more predefined categories. The advantages of basing such a keyword list on the titles of articles from Wikipedia is that Wikipedia is a large online encyclopedia that covers lots of subjects and is regularly maintained by lots volunteers. The category set for the classification can vary depending on the purpose of the classification, but it is essential that the predefined set is large enough to cover enough topics, but also so specific that information is preserved. An example of a predefined category set that is well-suited for advertisement is provided by IAB, Interactive Advertising Bureau. 

\subsection{The final results}

\end{comment}
\section{Further Works}
There exists many desirable extensions for our dictionary-based classifier that might improve the results or expand the usage. The most important future improvements for our classifier is solving ambiguity in a better way and extending the dictionary by exploring more than just the titles of Wikipedia articles. Expanding the usage is possible if the classifier is well-defined for other languages than just Wikipedia. 

\subsubsection{Disambiguation}
Our project removed all ambiguous titles. This means that we loose information that might be valuable for classification purposes. Instead of removing the titles, we could keep the titles that are most relevant for our classification, for instance the titles that have the longest articles. We studied different projects for solving disambiguation, and some of their findings could be applied to find the most likely meaning of an ambiguous entry. 

\subsubsection{Stubs}
Another possible change for the program is to remove Wikipedia stubs. Wikipedia stubs are pages that are too short to be considered articles. Wikipedia contains 1 913 507 stubs \cite{wiki:allstubs}, and these articles might provide ambiguous information which are more likely to be removed since they contain so little information that they are not considered articles.  


\subsubsection{Explore more information from Wikipedia}
We have only looked at the titles when creating the dictionary-based classifier. Another extension would be to explore the actual content of the articles before they are classified to the most describing categories. Better categorization of the Wikipedia articles could lead to better results for the classifier, which could improve the results. 

\subsubsection{Adding information}
We have only used information from Wikipedia, but it might be desirable to extend our classifier with information in addition to Wikipedia. Keywords from other sources might improve the results like it did in \cite{entityextraction} when adding information from MusicBrainz, City DB, Yahoo! Stocks, Chrome and Adam. 

\subsubsection{Improve Mapping}
The mapping between keywords and categories could also be improved by creating better decision rules between category paths and IAB categories. We chose to grade our Wikipedia article titles by using inlink and outlink number. Another improvement could be to try other grading algorithms, which might be better for determining the content of the articles. 
%It could also be improved by trying other grading algorithms. 

\subsubsection{Extending for more languages}
The results and implementation is created for English Wikipedia, hence only useful for English articles. We created a Norwegian dictionary-based classifier by using the internal Wikipedia links to translate the English classifier's dictionary to Norwegian. We noticed that the Norwegian classifier lacked important information for being able to classify Norwegian articles, and concluded that this is probably because special Norwegian keywords are missing from the dictionary. 

Thus, a desirable extension would be to create a more general approach for creating the dictionary so that it could be applied to other languages as well. Most of our programs are not dependent on language, except for the mapping rules. A good extension would be to create a language independent mapping process which could create dictionary-based classifiers from Wikipedia in multiple languages. 


%\subsection{Automatic mapping from Wikipedia titles}
%The mapping process from Wikipedia and to desirable output is a 





%It has discussed the different components needed to perform a content analysis on a collection of texts. Two of the most important components of the classifier has been discussed; a predefined list of keywords where the keywords are based on the titles of Wikipeda, and a predefined set of categories for the result. The paper has also discussed important properties of the predefined category set, followed by other similar work and discussion of evaluation of the classifier, i.e., what parts of the classifier can be evaluated and how this should be done. 

\backmatter{}
\cleardoublepage
\renewcommand{\bibname}{References}
%%\bibliography{mybib}

%\cleardoublepage

\addcontentsline{toc}{chapter}{References}
\printbibheading
\printbibliography[keyword=ref,heading=subbibliography,title={Main References}]
\printbibliography[keyword=wikipedia,heading=subbibliography,title={Wikipedia References}]

%e={Wikpedia Reference%\printbibliography[title={References},keyword=ref]

%\printbibliography[title={Wikipedia References},keyword=wikipedia]

%\printbibliography[keyword=primary, title={Primary references}]
%\printbibliography[notkeyword=primary, title={Other references}]


% Fra Sigmund:
%%\bibliographystyle{plain}

% 2. Clear double page. This is necessary to produce the correct page number for the
%    Bibliography-entry in the table of contents. This will however not resolve the problem with the
%    hyperlink of this entry in the table of contents, which will still refer to the previous
%    chapter/section/subsection in the table of contents (this could be fixed by adding the
%    BibTeX-file before including the Bibliography-entry (step 3), but then the page number will be
%    incorrect again).

% 3. Add the Bibliography-entry to the table of contents.


%%\bibliography{mybib}
\end{document}
