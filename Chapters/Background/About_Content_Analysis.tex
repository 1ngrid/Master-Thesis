\section{Automatic Content Analysis}
\label{sec:automatic_content_analysis}

\subsection{What is Content Analysis?}
Content analysis is the task of analysing and understanding collections of texts, in other words finding out what a text "is about". The task can be performed by both humans (manual content analysis) and computers (automatic content analysis), and both of the approaches have their advantages and disadvantages.

The concept of manual content analysis is easy. The task is split into first reading and understanding the text, then summarizing the content of the text and/or categorizing it into suitable categories describing the content. As an example, an article about \emph{Ole-Johan Dahl} (the famous Norwegian computer scientist \cite{Olejohandahleng}) would probably be summarized as an article about a famous Norwegian computer scientist and might be categorized under the category \emph{Norwegian computer scientists} if this category is present or the category \emph{computer scientists} if this is present.  
%as an article about authors of children books and Swedish people. 
There are two main disadvantages of manual content analysis which makes it impossible to perform on large collections of texts. The first disadvantage is that the task is time consuming, i.e. it takes time for a human to read and understand an article. The second disadvantage is that manual content analysis requires resources that might be expensive, for instance experts needed for understanding the content of an article if the article is about something beyond common knowledge.

%because some articles needs experts for understanding the content. 
%first has to be read and understood and then we could summarize the content of the text or categorize it under relevant topics.
Automatic content analysis is based on a different approach; instead of reading and understanding the text, the machine looks for predefined properties of the text (in our case known words or phrases) and uses these properties to determine the meaning of the text. This requires some predefined connection between the properties and their associated categories. This approach has disadvantages as well; computers lack commonsense knowledge usually known to ordinary humans, for instance physical description or function of objects. % TODO: Insert reference here.
Color is an example of a physical description computers have problems with determine. Most humans would understand that the phrase \emph{same color as the sun} means yellow, while computers would need specific information about the sun being yellow to conclude the same. 

Another disadvantage with automatic content analysis is dealing with disambiguation. Some words have more than one meaning, and the meaning is usually found from the context or the other words in the sentence. The task of determining the true meaning of a word or sentence is a difficult process which becomes harder if the sentences are complex. 

% TODO: Insert example of complex sentence.

%The easiest way to describe the meaning of the text is to group texts with similar content together, in other word categorize the texts. 

%Some of the advantages with automatic content analysis are that miss


%There are different ways to perform automatic content analysis, our approach is to find the most likely category for texts given as input by first categorizing all articles from Wikipedia. 

%involves using categorization of articles from Wikipedia to determine 