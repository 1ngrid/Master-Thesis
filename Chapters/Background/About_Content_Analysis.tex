\section{Automatic content analysis}

\subsection{What is Content Analysis?}
Content analysis is the task of analysing and understanding collections of texts, in other words finding out what a text "is about". The task can be performed by both humans and computers, where both of the approaches have their advantages and disadvantages.

The concept of manual content analysis is easy, where the task is split into first reading and understanding the text, and then summarizing the content of the text and/or categorize it into suitable categories. For instance an article about \emph{Ole-Johan Dahl} (the famous Norwegian computer scientist \cite{Olejohandahleng}) would probably be summarized as an article about a famous Norwegian computer scientist and might be categorized under the category "Norwegian computer scientists" if this category is present.  
%as an article about authors of children books and Swedish people. 
There are two main disadvantages of manual content analysis which make manual content analysis impossible on large copllections of texts. The first disadvantage is that the task is time consuming, i.e. it takes time to read and understand an article for humans. The second disadvantage is that manual content analysis requires resources that might be expensive, for instance experts needed for understanding the content of an article if the article is about something beyond common knowledge.

%because some articles needs experts for understanding the content. 
%first has to be read and understood and then we could summarize the content of the text or categorize it under relevant topics.
Automatic content analysis is based on a different approach; instead of reading and understanding the text, the machine looks for known words or phrases and uses these to determine the meaning of the text. This approach has disadvantages as well; computers lack commonsense knowledge usually known to ordinary humans, for instance physical description or function of objects. Another disadvantage with automatic content analysis is dealing with disambiguation. Some words have more than one meaning where the meaning is usually found from the other words in the sentence. The task of determining the true meaning of a word is a difficult process which becomes harder if the sentence is complex. 

The easiest way to describe the meaning of the text is to group texts with similar content together, in other word categorize the texts. 

%Some of the advantages with automatic content analysis are that miss


%There are different ways to perform automatic content analysis, our approach is to find the most likely category for texts given as input by first categorizing all articles from Wikipedia. 

%involves using categorization of articles from Wikipedia to determine 