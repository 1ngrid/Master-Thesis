\section{Wikipedia}
%It has already been mentioned that content analysis needs a keyword list for recognizing words or phrases that are useful for classification. We require that the list is so large that it contains almost all the words that give information of possible categories for the content where the keyword is found.  We have chosen to use Wikipedia to create such a keyword list. 
Wikipedia is a free, online encyclopedia and community that was created by Jimmy Wales in 2001. The encyclopedia is edited by the Wiki-principle, which means that everyone can create and edit articles. To understand the importance of Wikipedia it is worth mentioning that the web page has been ranked as the fifth globally most important web page (New York Times, February 2014), with  more than 30 million articles and almost 500 million unique users a month \cite{wiki:wikipedia}. 

Wikipedia contains a multitude of articles within many subjects and is maintained by thousands of people. Hence, the idea is to base the list on all the titles in Wikipedia, but the list has to be modified to contain only relevant titles. It is for instance not relevant to have common words in the keyword list which will occur in most articles and not provide any useful information. It is also important to remove or weight down ambiguous words, i.e., words that could confuse the categorization process or apply wrong information. 

One of the main advantages of using Wikipedia is the underlying structure that is already provided. All articles are already categorized which gives information about the content of the article connected to the title. 

%TODO: Utdype mer om vekting osv. 