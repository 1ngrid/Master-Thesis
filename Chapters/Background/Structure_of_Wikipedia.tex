\subsection{Structure of Wikipedia}
The structure of Wikipedia is web based, where articles with similarities are linked together. Since Wikipedia is language-based, articles only link to other articles within the same language. Wikipedia does also have a category structure, where all articles are classified under at least one category. A category could have articles, but could also have subcategories, where the subcategories have their own articles and subcategories. Together they form a large category graph, which is an abstract structure that shows the relationships between the categories. All Wikipedia articles are placed under the most describing categories, as an example Ole Johan Dahl (Norwegian computer scientist \cite{Olejohandahleng}) is placed under the category \emph{Norwegian computer scientists} which is under the parent category \emph{Computer scientists by nationality} which is under \emph{Computer Scientists}. 

The category graph is created so there is a link between a category and its subcategories. There is no beginning of the category graph, but there are some categories which have most other categories as their subcategories. These can be thought of as beginning categories, also called root categories, and are important when we want to look through all categories in the graph and observe the relationships between them.  Two categories that can be viewed as potential root categories are \emph{Fundamental Categories} or \emph{Main Topic Classifications}. If one of these are chosen as the root category, we can continue through the graph by looking at its subcategories and proceed by looking at each of the subcategory's subcategories an so on.
%An important 
%The easiest way of looking at all categories in the graph is to choose a root category and follow the links to its subcategories and then continue to look 

Figure \ref{fig: subcat_lindgren} is an example of a structure for the category \emph{Astrid Lindgren}, the Swedish writer of children books. The figure shows a tree structure for the category from the category graph. The figure shows that the category \emph{Astrid Lindgren} has 10 pages directly under the category, and 4 subcategories: \emph{Astrid Lindgrens characters} (9 pages), \emph{Films based on works by Astrid Lindgren} (1 subcategory and 23 pages), \emph{Works by Astrid Lindgren} (2 subcategories and 7 pages) and \emph{Pippi Longstocking} (1 subcategory and 10 pages).  This means that there are directly or indirectly 59 pages under the category \emph{Astrid Lindgren} without counting pages under the next level of subcategories. 


%is created and how it is fetched from the page for category information.\footnote{\categorytree}

\begin{figure}[H]
\centering
\includegraphics[height=2.5cm]{Chapters/Background/Astrid_Lindgren}
\caption{Subcategories of the category \emph{Astrid Lindgren}. }
\label{fig: subcat_lindgren}
\end{figure}

% HVORFOR IKKE BRUKE WIKIPEDIA SINE KATEGORIER!
Wikipedia articles are already classified under categories, but the set containing all Wikipedia categories cannot be used as a final categorization. The category set in Wikipedia is too large for such usage, where some categories do not provide information of the actual content, and some are too specified. There are also cases where articles are categorized under categories where the combination of the categories does not provide any new information. An example is the article of \emph{Ole Johan Dahl}. Some of the article's categories are showed in figure \ref{fig: olejohandahl_categories}. In this example the article is both placed in the category \emph{People from Mandal, Norway} and in the category \emph{Norwegian Computer Scientists}. These categories both provide information about him being Norwegian, so it would be sufficient to put him in the category \emph{Computer Scientists}. The categories shown are also quite specific, and it might be desirable with more general categories. 

% TODO: Mamam: Dette var vanskelig å forstå


%of an article where the categories provide the same information, i.e., we already know that he was from Norway since he is in the category \emph{People from Mandal, Norway}, so it would be enough to add that he was a computer scientist instead of specifying that he was a Norwegian computer scientist. 

%categorization between articles and categories cannot be used as a pre-defined category set. 
%Another reason for creating a new independent category set is that Wikipedia categories are not guaranteed to be in the desirable final category set. 

Another reason for creating a new independent category set is that the Wikipedia categories are not guaranteed to be in the desirable final category set. Hence it is essential that the classifier creates a connection from the article and to a category that is know to exist in the set. The classifier should instead be based on the category information provided by Wikipedia. 

%We will therefore need a mapper to a category we know exists. 
%but we cannot use either the categorization from articles to categories nor 
%this categorization is not ideal. It is not possible to use the categorization since a topic might lead
%Another reason why it is not possible to use 
%the category set in Wikipedia as the predefined category set because the category set in Wikipedia is too large. Some categories do not  provide information of the actual content, and some are too specified. There are also cases where articles are categorized under categories where the combination of the categories does not provide any new information. An example is the article of \emph{Ole Johan Dahl}. Some of the article's categories are showed in figure \ref{fig: olejohandahl_categories}. This is an example of an article where the categories provide the same information, i.e., we already know that he was from Norway since he is in the category \emph{People from Mandal, Norway}, so it would be enough to add that he was a computer scientist instead of specifying that he was a Norwegian computer scientist. 


\begin{figure}[H]
\centering
\includegraphics[width=\textwidth]{Dumps/imgs/olejohandahl-categories.png}
\caption[Categories for an Wikipedia article]{Some of the categories for the article of Ole Johan Dahl}
\label{fig: olejohandahl_categories}
\end{figure}

%set is not ideal as a predefined category set for our classifier. 
%There are two main reasons why the categories cannot be based on the categories from Wikipedia. 
%A reason for this is that an article in Wikipedia can be categorized under more than one category and these categories might not be the relevant category set. 
%In many cases are the categories directly subcategories of another category, but in some cases could it be a larger path until a common parent category and the category structure would therefore have to be flatten to make sure it is not classified under conflicting categories. 

%Hva tenkte jeg her? Another reason is that Wikipedia contains


Instead of creating a categorization from the Wikipedia titles and to the most describing categories from Wikipedia's category set, we want to create a connection to a category in a predefined category set. This set of categories should be presented as the desired output and be so simple that it can be understood by the users of the program. The category set should also contain a special category \emph{unknown} for texts where no category is found.

Another requirement for the category set is that it has to be so well-defined and specific that it conserves as much information as possible about the categorized text. The solution to the problem should also be able to change the predefined category set to another predefined set depending on the texts that are being categorized and the context of the classification.
%that a text is not categorized under conflicting categories, for instance the category "sport" and the category "not sport". We also want the categories to be specific to have more information about the categorized text. 
Hence the best result of the classifier would be found if the set satisfies all of these requirements with the best possible trade off between specialization and generalization. 