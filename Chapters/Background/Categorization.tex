\section{Categorization}
The process of grouping collections of text into categories is called categorization and can be done by either humans or computers. Computer categorization is the technique of teaching a classifier how to decide the category of any input \cite{wiki:categorization}. The idea of this process is to find patterns which makes the machine able to predict the category or class of the input. Such patterns could be similarities between input or decision rules \cite{wiki:classification}. It is desirable to optimize the results of the classifier, hence make the classifier as accurate as possible. 
%make the classifier as good as possible,

%Categorization with machine learning can be split into further types; statistical classification which is a \emph{supervised} learning process, and clustering which can be performed as both %an \textit{supervised} and a \emph{supervised} and an \emph{unsupervised} learning process. Supervised learning is a technique where the machine is given a training data set, where the set contains the correct output in addition to the input we want to classify. The classifier uses this data to learn the machine how to behave, also called training the classifier. Unsupervised learning, on the other hand, is the task of trying to find a hidden structure in unlabeled data. The main difference between the two types is that unlabeled data gives no feedback to the classifier, hence the classifier has to assume that it is correctly classified without receiving feedback. It is also possible to have classification processes which are combined of the two, where prior knowledge is given to the classifier. 
%added to the classification process for better results. 

%the data is unlabeled is no feedback sent to the classifier. 
%The classifier will therefore not know if the result is correct, but will continue to classify assuming that the classification performed so far is correct.  

Our problem consists of two categorization problems. The first is to classify Wikipedia articles to the most describing category from a desirable category set. The second categorization problem is to categorize any input article based on the occurrence of titles of wikipedia articles in the text and the corresponding categories for these articles. 
%the main categorization where any article given as input should be categorized to its most describing category.  

\subsubsection{Categorization of all Wikipedia articles}
The classifier needs to know what all Wikipedia article are about to be able to categorize them. Our assumption is that the meaning of the Wikipedia article can be found by looking at the underlying structure of Wikipedia i.e, the article's categories and the category structure. A good result is achieved by building the classifier with heuristics and rules for categorization. 
%the categorization by finding rules and apply heuristics. 

\begin{comment}

Categorizing the articles can be done 

When the most describing categories of the article is found

Here the task is to create classifier that looks at the underlying structure of Wikipedia to describe the categories and then 

into their most describing categories. To be able to do this, it is essential to look at the underlying structure of Wikipedia to determine what the article is about. 

Since there are no training sets available, the categorization has to be unsupervised. Wikipedia still has an underlying structure which is useful for determining the most likely categories for all Wikipedia articles. This means that we can use this as heuristic for our categorization. 
\end{comment}

%The categorization needed to determine the categories for each article is a statistical classification. Wikipedia's structure is available and this can be used to find the most likely categories. 

\subsubsection{Categorization of any article}
Categorizing articles based on occurrence of Wikipedia article titles are the main goal of the implementation. The main assumption is that an article containing many titles categorized to the same category is more likely to be categorized to the same category. A text could contain keywords belonging to different categories, so some rules or heuristics might be needed to decide the category of the text. Hence, our approach is creating a dictionary-based classifier that has a connection between keywords and their describing categories.

\begin{comment}
Mye nyttig: 

To be able to categorize any article given as input, it is essential to have a list of keywords which give some indication of possible categories for the article. A keyword list should contain words or phrases useful for classification, and we have chosen the Wikipedia titles to be such a list. 

\end{comment}

%. The main assumption 

%The next classification problem is to classify 
%Our main assumption for content analysis is that articles which contains the same keywords also belong to some of the same categories in Wikipedia. This means that we want create a group of these articles so that similar articles are grouped together. Supervised classification requires, as already mentioned, a training set. The training set of our problem can be defined as articles in Wikipedia since they are already connected to a category within Wikipedia. The task is to create the classifier that use  this information and is able to classify all other articles. 

%does not have a training set because it is almost impossible to create a training set representiing such a large data set. We still have, however,  information about the underlying category structure of the articles in Wikipedia. The goal is therefore to use this information to group similar articles together.
%Our problem is not suitable for supervised machine learning. Trying to solve the problem with supervised classification would lead to some problems that are difficult to solve; it is for instance almost impossible to create a training set to represent such a large data set and it is therefore not possible to create a classification model based on the data. The categorization should therefore be done with unsupervised machine learning, for instance clustering.

%The formal definition of cluster analysis or clustering is the task of grouping similar elements together.Hence the group or cluster should contain elements that share similarities or that are more similar to each other than to the rest of the elements. %This means that elements within a group are more similar to each other than to the rest of the elements, or that the elements within the group have some similarities that make them stand out from the others. 
%Our problem could therefore be defined as a clustering problem, where each cluster or group is the articles which contains some of the same keywords. We want to sort the texts in such a way that texts with similar content are classified to the same cluster and therefore to same category. The problem needs a mapping process so that collections of texts get clustered together within the predefined set of categories. The predefined set of categories will change depending on the purpose of the classification, for instance would advertisement need a different set of categories than categorization of news articles. A proposal to a predefined category set for advertisement is the category set of IAB. 