\section{Interactive Advertising Bureau (IAB)}
The predefined category set should be well-defined and fit for the purposes of the task. Since the focus of this project is improving advertising, the predefined category set should be a category set useful for advertising. 


%The machine learning need a predefined set of categories for the clustering. It is already mentioned that Wikipedia has articles stored under categories and that the categories form a tree or graph structure. The problem is that there are too many categories that are not relevant for our categorization.

%The problem is therefore using IAB's categories for the clustering. 

IAB is a business organization that develops, researches and maintains industry standards for the online advertising industry. The organization works for creating, coalescing and maintaining standards and practices in online advertising. In addition, IAB research and share knowledge on the advertisement, and is responsible for distributing 86 \% of all the online advertisement in the US \cite{IABabout}.

IAB provides different guidelines for advertising, including \emph{Quality Assurance Guidelines Taxonomy} (QAGT). This taxonomy is a well-defined for advertising, and can be viewed as a category set. The set is split into two \emph{layers} also called \emph{tiers}. The layers are created for varying the grade of speciality between first tier (a general or broad level) and the second tier (a deepening level). The first tier contains a total of 23 categories, examples are \emph{Business} and \emph{Food \& Drinks}. The second tier contains 371 subcategories, where each subcategory is a more specified category of a category in the first tier. 

%, where the categories are subcategories of a category in the first tier.
Figure \ref{fig:IAB} shows the taxonomy of IAB as defined on their web page where the first tier is all the category names written in white (e.g. \emph{Food \& Drinks}) and the second tier is followed under the first tier (eg. \emph{American Cuisine}). Table \ref{tab:taxonomyascategories} is an example of how parts of the taxonomy for \emph{Food \& Drinks} and \emph{Hobbies \& Interests} is written as a category set, where the second tier is placed under the first tier. This means that an article mapping to \emph{Chinese Cuisine} maps to the category \emph{Food \& Drinks/Chinese Cuisine}.

\begin{table}[h]
\centering
\begin{tabular}{l|l}
%\textbf{Tier 1} & \textbf{Tier 2} \\ \hline
\textbf{Food \& Drinks} & \textbf{Hobbies \& Interests} \\ \hline
American cuisine & Art/Technology\\
Barbecues \& Grilling & Arts \& Crafts\\
Cajun/Creole & Beadwork \\
Chinese Cuisine & Birdwatching\\
Cocktails/Beer & Board Games/Puzzles\\
Coffee/Tea & Candle \& Soap Making\\
Cuisine-Specific & Card Games\\
Desserts \& Baking & Chess \\
... & ...
\end{tabular}
\caption{Example of how the IAB taxonomy changed to a category set}
\label{tab:taxonomyascategories}
\end{table}




%The category set from IAB's taxonomy is a well-defined category set to use of our clustering problem.  




% I stedet kan vi bruke Wikipedia-kategoriene for en sjekk for å se om v har kategoriesert rett?
%
\begin{figure}[t]
%\centering
\begin{subfigure}{\textwidth}
\includegraphics[width=\textwidth]{Chapters/Background/Taxonomy-1.png}
%\caption{Categories of the IAB Taxonomy}
%\label{fig:IAB1}
%\end{figure}
\end{subfigure}
\begin{subfigure}{\textwidth}
%\begin{figure}[H]
\centering
%\newline
\includegraphics[width=\textwidth]{Chapters/Background/Taxonomy-2.png}
\end{subfigure}
\caption{Categories of the IAB Taxonomy}
\label{fig:IAB}
%\label{fig:IAB-categories}
\end{figure}
%Figure \ref{fig:IAB-categories} 


%When categorizing a collection of texts, similar texts will be in the same cluster and therefore in the same category. This can be used to determine the content of the text. 

%The content analysis need a list of keyword to look for in texts. 

