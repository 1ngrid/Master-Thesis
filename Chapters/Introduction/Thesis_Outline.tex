\section{Thesis Outline}
\label{sec:thesis_outline}
We consider chapter 2 to be an \emph{introduction} to the project by describing the definition and purpose of content analysis. The chapter is called \emph{Background Materials} because it also covers the basic material needed for understanding the purpose of the project as well as the methods used in the implementation. The background material includes a basic introduction to the \emph{categorization problems} we want to solve, \emph{Wikipedia} and its underlying structure, a brief introduction to the taxonomy of \emph{Interactive Advertising Bureau} (IAB), and finally how our results are found with help from \emph{Cxense}.

Chapter 3 is dedicated to \emph{Related Works}, mostly concerned with Wikipedia categorization or extracting semantic knowledge from Wikipedia categories. The chapter also contains a discussion on whether knowledge from the previous works can be used in this project. 

We consider chapter 4-5 as \emph{Methods}. Chapter 4 focuses on the methods for representing the structure, grading different paths, and evaluating the results. Chapter 5 focuses on details of the implementation of the project, and gives a deeper discussion of the problems encountered and possible solutions. Chapter 5 describes the process of finding the full path of all Wikipedia articles in detail, how to determine the meaning of articles by grading the category paths, and the processes of mapping Wikipedia article titles to categories.

\emph{Results and Discussion} is covered in chapter 6, including improvements of the implementation and discussion of the results. It also evaluates which categories are easily detected and compares our results with other text categorizations based on Wikipedia.

Finally, chapter 7 contains our \emph{conclusion} for the project; whether a text can be determined based on occurrences of Wikipedia article titles or not. The chapter also covers possible \emph{further works} for obtaining even better results, and desirable features for the project. 

\begin{comment}
covers the background material needed for understanding the purpose of the project and the methods used in the implementation. The background material covers the purpose and definition of content analysis, categorization, Interactive Advertising Bureau, and finally Wikipedia and its structure. 

In chapt

After , chapter 3 

After describing the challenges, the next natural step is to look at previous work and research which is given in chapter 3, along with discussing if these findings can be used in our project. 

Chapter 4 is focusing on the implementation of the project and explains the approach for solving the problem. The chapter describes the process of finding full path of all Wikpedia articles from Wikipedia categories, how the paths are graded to find the most relevant paths for each article and the mapping process to the desirable output category set. 

Chapter 5 shows the results and improvements made to the implementation, which led to  

\end{comment}