\section{Thesis Outline}
We consider chapter 2 to be an introduction to the project by describing the definition and purpose of content analysis. The chapter is called \emph{Background Material} because it also covers the basic material needed for understanding the purpose of the project as well as the methods used in the implementation. This includes a basic introduction to the \emph{categorization problems}, \emph{Wikipedia} and its underlying structure, and finally \emph{Interactive Advertising Bureau} (IAB). 

Chapter 3 is dedicated to similar previous works, mostly concerned with Wikipedia categorization or extracting semantic knowledge from Wikipedia categories. The chapter also contains a discussion whether knowledge from the previous works can be used in this project. 

We consider chapter 4-5 as \emph{Methods}. Chapter 4 focus on the methods for representing the structure, grading different paths, and evaluating the results. Chapter 5 is focusing on details of the implementation of the project, and [går dypere inn] in problems encountered and how solutions. This chapters describes the process of finding full path of all Wikipedia articles, and how the meaning of the articles can be found from grading the paths found in the underlying structure of Wikipedia categories. The last part in this chapter is the processes tried fro mapping the Wikipedia article titles to categories from the category set based in IAB's taxonomy. 

\emph{Results and Discussion} is covered in chapter 6, including improvements of the implementation and discussion of the results.

Finally, chapter 7 contains our \emph{conclusion} for the project, followed by further works. 

\begin{comment}
covers the background material needed for understanding the purpose of the project and the methods used in the implementation. The background material covers the purpose and definition of content analysis, categorization, Interactive Advertising Bureau, and finally Wikipedia and its structure. 

In chapt

After , chapter 3 

After describing the challenges, the next natural step is to look at previous work and research which is given in chapter 3, along with discussing if these findings can be used in our project. 

Chapter 4 is focusing on the implementation of the project and explains the approach for solving the problem. The chapter describes the process of finding full path of all Wikpedia articles from Wikipedia categories, how the paths are graded to find the most relevant paths for each article and the mapping process to the desirable output category set. 

Chapter 5 shows the results and improvements made to the implementation, which led to  

\end{comment}