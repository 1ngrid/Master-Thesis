\section{Motivation}
\label{sec:motivation}
Imagine the possibilities if your computer was able to understand what you wanted to do at all times. This could be a computer that knows your address so it can remind you to take the last bus home from friends, or it could read emails from your boss and remind you of deadlines. The computer would need to be intelligent to perform such tasks. The study of creating intelligent computer software is called \emph{Artificial Intelligence} (AI) and is one of the most discussed fields in modern time. 

There are some challenges before computers today is considered intelligent. One of the main challenges is the task of making computers understand natural language. This task is commonly called \emph{Natural Language Processing} (NLP) and defined as the task of getting computers to perform useful tasks involving human language \cite[p.~35]{jurafsky2000speech}. 

Our idea is that computers may perform better in many settings if they are able to determine the meaning of a text. Thus, the goal of this study has been to develop an automatic content categorization which could take any article as input, and determine the most likely category based on its content. Our approach for determining the most likely category is by creating a dictionary-based classifier from Wikipedia, where the titles of Wikipedia articles are used as entries, and each entry is connected to one or more suitable categories describing the content of the Wikipedia article. 

