\section{Classification of tweets}
One of the similar projects is described in the article \emph{Entity Extraction, Linking, Classification, and Tagging for Social Media: A Wikipedia-Based Approach}\cite{entityextraction}. The initial problem described is a content analysis problem, where the goal is to categorize tweets based on their content. The chosen solution to the problem was automatic lssification of tweets, where the categorization is based on the content. This problem resemble our problem since both problems are based on understanding texts by recognizing keywords that provide information of the most likely categories. 


%where they wanted to let the computer understand what the tweets are about and then sort them based on their content. The chosen solution to the problem was to use automatic classification of the tweets, where the tweets were categorized by their content. This problem is very similar to our problem because both the problems are based on understanding texts by recognizing keywords that provide information of categories, i.e., what the tweets are about.

The article describes the categorization as the machine learning process where tweets with similar content are placed in the same class. Understanding the content requires the machine to have some basic knowledge, which is usually called a knowledge base or a repository for information. The atuhors chose to use Wikipedia as the knowledge base, for a numereous of reason, including that it is the largest online encyclopedia, that it is based on volunteering hence rapidly updated and since it is constantly crawling and therefore is able to have a fresh, dynamic and timely knowledge base. 


%The authors chose to use Wikipedia as their knowledge base (a repository for information) for finding information of the different categories. The reasons given for why they chose Wikipedia are similar to our reasons;  it is the largest online encyclopedia, it is based on volunteering which means that it is rapidly updated, and it is constantly crawling which is important since it is an advantage to have a fresh, dynamic and timely knowledge base. 

A difference between classifying tweets and articles are the preprocessing phase. Tweets require lots of preprocessing since they are quite short (max 160 characters). 


%while articles need preprocessing in order to 

The processing of the tweets required a lot of preprocessing before they could be classified content, especially since tweets are quite sort (max 160 characters).  The preprocessing of the tweets contained several steps before the actual categorization could start, including language detection, cleaning of the tweets (removing everything that is not text), and a tokenizing process (separating sentences into tokens, where a token is defined as a sequence of characters, usually normal words). The described preprocessing is similar to the preprocessing intended for our content analysis because the classifier will only find the keywords if they are an exact match. The classifier depend on tokenizing and cleaning of the words in the text to make them similar to keywords in the keyword list.  

The described tweet classification required some structure to keep information of the tweets and their possible categories. The solution was to create a structure of mentions where a mention is defined on the form ($m_{i}$, $n_{i}$, $s_{i}$), where $m_{i}$ is the string in the tweet we refer to, $n_{i}$ is the node in the knowledge base, and  $s_{i}$ is the< score of the node. All tokens with a connection to the knowledge base (i.e., Wikipedia) were considered relevant, while the others were removed to reduce the complexity. The structure of mentions could be too complex for our case with collections of texts since each text can be much longer than 160 characters, but the idea of keeping track of possible categories is the same. 

A scoring function was used for deciding categories for the tweets, where all the mentions were filtered and some hand-crafted rules were applied. Our project would also need some function to decide what categories are relevant if many categories are proposed. 
