\subsection{Extracting Semantic Relationships between Wikipedia Categories}

\subsubsection{What is the article about?}
Hvordan man kan gjøre Wikipedia søk bedre: Hente semantic informasjon og analysere linker mellom ategoriene. 

Find the type of semantic connections between Wikipedia categories. (sort between weak, average and strong).  Two measures: number of links between pages in two categories, connectivity ratio.

Extracting semantically important relationships: 
goal: emphasize the meaningful relationships between categories and disregard unimportant ones. 


strong relationship: A should conceptually have at least one semantic link to B
average: 50\% of the pages should have semantic links to B
weak: less than 50 \%

Number of links between categories is a good indicator of the level of semantic relationship: number of pages in source category which have at least one link to any page in the target category. 

CONC: Inset: obtained stronger semantic relationships in comparison to outset. 



\subsubsection{What can be added to our solution? How could our solution be better based on their findings?}
If we are able to represent the relationships between Wikipedia categories, we might be able to choose better paths to represent the meaning of the Wikipedia articles. 

We could take advantage of linkage within the documents as well - better know the meaning of the article. 


\subsubsection{Possible quotations}
Pages in Wikipedia are explicitly assigned to one or more \emph{Categories}. 

While most of them are created to provide efficient navitation of the Wikipedia contents, they also represent some semantic relationships between pages and categories. 

Wikipedia forms a signel conntected graph without isolated components or outliers. 



\textbf{Describe the graph: }
Graph: nodes are categories and the edges are hyperlinks. 
--> nodes are categories and the edges are links between categories 



