\chapter*{Abstract}
Automatic categorization of content is an important %piece of 
functionality in online advertising and automated content recommendations, both for ensuring contextual relevancy of placements and for building up behavioral profiles for users that consume the content. Within the advertising domain, the taxonomy tree that content is classified into is defined with some commercial application in mind to somehow reflect the advertising platform’s ad inventory. The nature of the ad inventory and the language of the content might vary across brokers (i.e., the operator of the advertising platform), so it was of interest to develop a system that can easily bootstrap the development of a well-working classifier. 

We developed a dictionary-based classifier based on titles from Wikipedia articles where the titles represent entries in the dictionary. The idea of the dictionary-based classifier is so simple that it can be understood by users of the program also those who lack technical experience. Further, it has the advantage that its users easily can expand the dictionary with desirable words for specific advertisement purposes. The process of creating the classifier includes a processing of all Wikipedia article titles to a form more likely to occur in documents, before each entry is graded to their most describing Wikipedia category path. The Wikipedia category paths are further mapped to categories based on the taxonomy of Interactive Advertising Bureau (IAB), which are categories relevant for advertising. The results of this process is a dictionary with entries connected to categories from the taxonomy, and forms the base of our classifier. Finally, we explored the possibilities of using Wikipedia's internal links to translate the English classifier's dictionary to a Norwegian dictionary.

The evaluation of the classifier was performed on \texttt{rappler.com} for the English classifier and \texttt{adressa.no} for the Norwegian classifier. The results of the classifiers were compared with a class tag within the url structure of published articles, and we could see that the classifiers were able to correctly categorize most articles. 
%The results show that the classifier is able to correctly categorize most articles. 
%the vast majority of
However, there is room for further improvement of the classifier in order to achieve higher evaluation scores. This is partly because our dictionary-based classifier is a one-to-many classifier, while we compare the results to a one-to-one classification.

Overall, we found that we are able to create a varied and thorough dictionary by just exploring %the utilising
the titles of Wikipedia articles, and that the classifier gives a good indication of the content of articles. 


\begin{comment}

form??? 

"explain" -> tanken bak er enkel. 


Brokers are often not very technical and by experience will have severe problems developing training sets or otherwise contribute to the process, so any required involvement from their side has to be relatively simple. Furthermore, it is a practical requirement that the classifier can “explain” its classification to the broker in some way, and that the broker can have a simple way to manually override or influence the classification of known problem cases.

We explore the use of Wikipedia to develop a simple dictionary-based classifier. A dictionary-based classifier offers a simple way to “explain” the classification, and allows the classification vocabulary (i.e., the dictionary entries) to be easily edited. Wikipedia exists in a large number of languages, has a large number of article keywords covering most domains, and explicitly associates article names with categories. We describe a set of tools that automate the process of building up dictionaries that map Wikipedia keywords (or cleansed versions thereof) into Wikipedia categories (or modified versions thereof). Creating a mapping that maps Wikipedia categories into the broker’s custom taxonomy tree is a relatively straightforward task that brokers (or people working on their behalf) are assumed capable of. We will here use predefined a taxonomy, the one provided by IAB, as a working example of such a custom taxonomy tree.

Given such a classifier, we describe an experiment using a real advertising platform to validate its use in real life using real data. We also provide a brief overview of related work described in the literature.

\end{comment}