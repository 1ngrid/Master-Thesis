\chapter*{Abstract}
Automatic categorization of content is an important %piece of 
functionality in online advertising and automated content recommendations, both for ensuring contextual relevancy of placements and for building up behavioral profiles for users that consume the content. Within the advertising domain, the taxonomy tree that content is classified into is defined with some commercial application in mind and needs to somehow reflect the advertising platform’s ad inventory. The nature of the ad inventory and the language of the content might vary across brokers (i.e., the operator of the advertising platform), so it is of interest to develop a system that can easily bootstrap the development of a well-working classifier. Brokers are often not very technical and by experience will have severe problems developing training sets or otherwise contribute to the process, so any required involvement from their side has to be relatively simple. Furthermore, it is a practical requirement that the classifier can “explain” its classification to the broker in some way, and that the broker can have a simple way to manually override or influence the classification of known problem cases.

We explore the use of Wikipedia to develop a simple dictionary-based classifier. A dictionary-based classifier offers a simple way to “explain” the classification, and allows the classification vocabulary (i.e., the dictionary entries) to be easily edited. Wikipedia exists in a large number of languages, has a large number of article keywords covering most domains, and explicitly associates article names with categories. We describe a set of tools that automate the process of building up dictionaries that map Wikipedia keywords (or cleansed versions thereof) into Wikipedia categories (or modified versions thereof). Creating a mapping that maps Wikipedia categories into the broker’s custom taxonomy tree is a relatively straightforward task that brokers (or people working on their behalf) are assumed capable of. We will here use predefined a taxonomy, the one provided by IAB, as a working example of such a custom taxonomy tree.

Given such a classifier, we describe an experiment using a real advertising platform to validate its use in real life using real data. We also provide a brief overview of related work described in the literature.