\section{Results of the Mapping}
\subsubsection{Evaluation of Mapping Wikipedia Categories to Output Categories}
The results from this approach were not so good for two main reasons. 
%The results from this approach from this approach were not good for two main reasons: 
The first reason is that it is difficult to perform matching based on words. A perfect result could only be achieved if the computer knows all synonyms, inflections and the true meaning of all words. 

The other problem was with ambiguous words in the category names. An example of this is the categories shown in figure \ref{fig:ambiguous_category_name} where both categories contain the word \emph{Cicero}, but where the first category is for the suburb of Illinois and the other is for the Roman philosopher. Creating mapping rules for these names would be a difficult task. 

\begin{figure}[h]
\centering
\begin{lstlisting}
Category:Cicero, Illinois
Category:Cicero
\end{lstlisting}
\caption{Example of two category names which contains the same word with different meaning, and should be classified to different categories.}
\label{fig:ambiguous_category_name}
\end{figure}


The conclusion for this approach is that it might be possible to create a mapping between each Wikipedia Category and one or more desirable output categories, but this would need a very specified third tier in the IAB category and lots of rules. The task would therefore resemble a manual classification and is not a good approach. 
%-> It is impossible to create a third tier to satisfy this. 
