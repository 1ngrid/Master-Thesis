
\section{Grading Categories}
Many Wikipedia articles can be reached from categories that are not descriptive of the content of the article. We found multiple paths to all Wikipedia articles, but some of them were less descriptive than others. Thus, a grading was done to find the most relevant paths for each article. 

\subsection{Grading based on Inlinks and Outlinks}
%\subsubsection{Inlinks and Outlinks of Categories}
Each category in Wikipedia has a set of super categories (categories that lead to the current category), and a set of subcategories (categories that can be reached from the current category). The super categories and subcategories leading to the current category form a set. The size of this set can be annotated as 
\begin{itemize}
\item \emph{Inlink number} = number of super categories (parent categories)
\item \emph{Outlink number} = number of subcategories (child categories)
\end{itemize}
Figure \ref{fig:Categorywparentandsub2} illustrates how  inlink number and outlink number are connected to a category. 

%We assume  that a category with high inlink and outlink number are more likely to be visited when looking for paths for an article. 

\begin{figure}[h]
\centering
\includegraphics[width=\textwidth]{Chapters/Methods/category_parent_sub}
\caption[Example of \emph{inlink number} and \emph{outlink number} for a category]{Example of how a category has links from parent categories and links to its subcategories. The \emph{inlink number} for the category is 4 and the \emph{outlink number} for the category is 3.}
\label{fig:Categorywparentandsub2}
\end{figure}

We created two assumptions based on this: 
%Two assumptions can be made from this. 
\begin{enumerate}
\item Categories with high inlink number can be reached from categories that are not about the same topic. 
\item Categories with a high outlink number are more likely to reach articles not necessarily connected to the category name since they can reach far in all the subcategories' directions.
\end{enumerate}
%The first assumption is that The other assumption is that  

All categories should be given a score based on their inlink and outlink number, where low score values are given to categories within narrow topics (high inlink and outlink number), and higher score values are given to categories that cover more general topics (low inlink and outlink number). 

%Categories with high inlink and outlink numbers should be given a lower score than categories seldom reached. 

%They cover a general topic, while categories with a low inlink number and low outlink number describe a narrow topic.

\subsubsection{Scoring paths}
Grading based on inlink and outlink number is done by finding the inlink and outlink number for all categories in the structure, and by finding the average inlink and outlink numbers for all categories. The scoring is weighted based on the values of the inlink and outlink numbers. This gives the following formula (equation \ref{eq:categoryscore}) for finding the score of a given category, where $\xoverline{C_{in}}$ is the average inlink number and $\xoverline{C_{out}}$ is the average outlink number.   


\begin{equation} \label{eq:categoryscore}
Score_{C} = \frac{inlink_{c} + outlink_{c}}{\xoverline{C_{in}} + \xoverline{C_{out}}}
\end{equation}

This means that the path score of a path $P$ is the sum of all scores for each category in the path (see equation \ref{eq:scoreinput}). 

\begin{equation} \label{eq:scoreinput}
Pathscore_{P} = \sum_{c} Score_{C}
\end{equation}

The problem with equation \ref{eq:scoreinput} is that short paths will be favored since there are fewer scores to be added together. A way of avoiding favoritism of short paths is by normalizing the path scores. 

\subsection{Normalized Grading based on Inlink and Outlink Numbers}
Grading based on inlink number and outlink number favors short paths even if the paths contains categories considered as bad. One way of handling this problem is by normalizing the score of each path. Equation \ref{eq:normscoreinput} is a way of normalizing the path score of path $P$ so the length of the path does not determine the relevance of the path. 

% TODO: Write something about normalization - why is it good for grading?

\begin{equation} \label{eq:normscoreinput}
Pathscore_{P} = \frac{1}{N} \sum_{c} Score_{C}
\end{equation}
where $N$ is the number of categories in the path.


\subsection{Deciding Relevant Paths}
There are different ways of deciding the relevant paths among all graded paths. One way is by choosing a threshold for the path score. If the path score is lower than a given threshold, it is marked as relevant, while a path score higher than the threshould means that it is not relevant. A threshold can be found by deciding how many paths should be considered relevant.

One way of doing this is by finding the scores of all paths. and sort the scores from lowest to highest (see \ref{eq:sortedscores}). Then a $k$ has to be decided to how many paths are believed to be relevant of all paths, for instance one could assume that only 10\% of the paths are relevant, which leads to $k = .10 \cdot n$. 

\begin{equation} \label{eq:sortedscores}
Sorted\_scores = \left[ S_{1}, S_{2}, ... , S_{k}, ... , S_{n} \right]
\end{equation}



\begin{equation} \label{eq:threshold}
T = Sorted\_scores[k]
\end{equation}


The problem with this method is that not all articles are guaranteed to have any relevant paths. The other problem is that the score of the path will vary a lot within different fields, since some of the Wikipedia articles are categorized under very specified categories. 
% TODO: Finn en kilde som er enig med meg. 

% Problem: 
% Finne hvor mange pather som er tilgjengelig. 

Another approach is to choose the best $k$ paths for each Wikipedia article. This approach is independent of the values on other articles' path score which means all Wikipedia articles are guaranteed at least one path. The disadvantage is that some paths might be marked as relevant even though their path score is lower than path scores marked as irrelevant by other articles. Another disadvantage is that articles with many good paths will still have to choose the best $k$ paths and good paths might be lost. 

\begin{comment}
Fordeler: ser ikke på de andre
alle articler får minst en score. 

Ulemper: mange gode - hvilken er best?
Kan ikke vite om scoren er god

\end{comment}