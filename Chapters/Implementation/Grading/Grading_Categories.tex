\section{Grading of Categories}
\label{sec:grading_of_categories}

Most of the articles can be reached from categories that are not descriptive of the content at all. The article about \emph{Ole-Johan Dahl} (the Norwegian programmer) can be reached from the category \emph{people}, but also found from the categories \emph{politics} and \emph{arts} (see Figure \ref{fig:olejohandahl_paths}). This means that not all paths are good for describing the content of the Wikipedia articles. Thus, the next step is to grade the paths, to find the paths most likely to describe the content.

%grade the paths to find the paths most helpful for describing the article's content. 

\begin{figure}[h]
\centering
\begin{lstlisting}
ole-johan dahl:
*people/people categories by parameter/categories by nationality/academics by nationality/norwegian academics/faculty by university or college in norway/university of oslo faculty

[...]

*politics/political activism/leadership/management/quality/software quality/formal methods/formal methods people

[...]

*arts/aesthetics/design/software design/data modeling/formal methods/formal methods people

\end{lstlisting}
\caption[Example of variety in article paths]{Some of the full paths found for the article about \emph{Ole-Johan Dahl}.}
\label{fig:olejohandahl_paths}
\end{figure}

