\subsection{Grading based on Inlinks and Outlinks} % eller: Grading based on inlink and outlinks 
\subsubsection{Inlinks and Outlinks of Categories}
Each category in Wikipedia has a set of parent categories i.e., categories that lead to the current category, and a set of subcategories i.e., categories that can be reached from the current category. The size of these sets for a given category can be notated as 
\begin{itemize}
\item \emph{Inlink} = number of parent categories
\item \emph{Outlink} = number of subcategories
\end{itemize}
Figure \ref{fig:Categorywparentandsub} is a demonstration of how  \emph{inlink} and \emph{outlink} are connected to a category, and gives the idea that a catgory with high \emph{inlink} and \emph{outlink} are more likely to be visited when looking for paths for an article. 
\begin{figure}[h]
\centering
\includegraphics[width=\textwidth]{Chapters/Implementation/Grading/category_parent_sub}
\caption[Example of \emph{inlink} and \emph{outlink} for a category]{Example of how a category has links from parent categories and links to its subcategories. \emph{inlink} for the category is 4 and \emph{outlink} for the category is 3.}
\label{fig:Categorywparentandsub}
\end{figure}


The first assumption is that categories with high \emph{inlink} can be reached from categories that are not about the same. 

%great rift valley: 5, 31
\begin{figure}[h]
\centering
\begin{lstlisting}
INSERT EXAMPLE HERE: 
\end{lstlisting}
\caption{Caption}
\label{fig:my_label}
\end{figure}


\begin{figure}[h]
\centering
\begin{lstlisting}
ole-johan dahl:
*politics/political activism/leadership/management/quality/software quality/formal methods/formal methods people
\end{lstlisting}
\caption{Example of how \emph{politics} can reach the article about \emph{Ole-Johan Dahl}}
\label{fig:politicstosoftware}
\end{figure}
%quality: 9, 3
%management: 31, 5
The next assumption is that categories with a high number for \emph{outlink} are more likely to reach categories not relevant since they can reach far in all the subcategories' directions. Figure \ref{fig:politicstosoftware} shows how the Wikipedia article about \emph{Ole-Johan Dahl} can be reached from the category \emph{politics}. One of the categories with a high \emph{outlink} is the category \emph{management}, which has \emph{outlink} as 31 and hence be reached many categories. 

\subsubsection{Grading based on Inlinks and Outlinks}
The assumption that categories with high \emph{inlink} and \emph{outlink} are more often visited leads to the thought that these categories should have a lower score than categories that are more rarly visited. 

The first approach was therefore to find the \emph{inlink} and \emph{outlink} of all categories in the structure. These numbers had to be compared with the average number of \emph{inlink} and \emph{outlink} to know whether the number is high or low (see Table \ref{tab:avginlinkoutlink}). 

\begin{table}[h]
\centering
\begin{tabular}{c|c}
\textbf{Average number of \emph{inlink}} & \textbf{Average number of \emph{outlink}}\\ \hline
 5 & 2 \\
\end{tabular}
\caption{Average number of inlink and outlink}
\label{tab:avginlinkoutlink}
\end{table}

The score for each category was then 

\begin{equation} \label{eq:scoreinout}
Score_{C} = \frac{inlink_{c} + outlink_{c}}{\bar{C_{in}} + \bar{C_{out}}}
\end{equation}
where $\bar{C_{in}}$ is the average \emph{inlink} and $\bar{C_{out}}$ is the average \emph{outlink}.

The scoring from formula \ref{eq:scoreinout} means that paths with categories rarely visited will be favoured, hence given a lower score. 

\subsubsection{Evaluation of the scores}
None of the categories can have a score of 0 since this means they cannot be reached or reach other categories. The lowest score found was 0.376010, which was given to all categories with only one parent and zero subcategories, a total of 104,471 categories.  The category with the highest score is the category \emph{Albums by Artist}, which is the category with most subcategories (17,393), hence a score of $6,512.120784$. Hence the range of the scores is <0.376010, 6,512.120784> where all category scores lie within this range. 

Figure \ref{fig:scorevalue} shows how many categories are found for each of the possible score values. The figure shows that there are many categories with a low score value, while there are only a few categories for the higher score value. The categories with high score value will have a high impact on the article path, hence the path will have a smaller chance of being chosen.

\begin{figure}[h]
\centering
\includegraphics[width=\textwidth]{Chapters/Implementation/Grading/Scorevalue_numberofcategories}
\caption{Number of categories for each possible score value}
\label{fig:scorevalue}
\end{figure}

%This means that the score for all categories are between 0.376010 and 6512.120784.

%This means that the scores for all categories are in the range of 0 and

%Maxgrade: 6512.120784 (albums by artist)
%Mingrade: 0.376010 (user bho-4)
\subsubsection{Problems with the simplified grader}
Since it is desirable with the lowest score as possible, the first problem encountered was that the program favoured short paths. Figure \ref{fig:short_path_favour} gives example of how the shortest path for the article \emph{Alexander Hughes} (English football player  \cite{wiki:alexanderhughes}) is favoured. These paths are not very descriptive of the article, where only the third best path gives information that he is connected to football. Instead, we would like to see if longer paths are better. 

\begin{comment}
\begin{figure}[h]
\centering
\begin{lstlisting}
argentines of serb descent:
* humans/ethnic groups/ethnology/ethnicity stubs/ (27.824755)
* culture/ethnicity/ethnicity stubs/ (29.704807)
* society/ethnicity/ethnicity stubs/ (34.592939)
\end{lstlisting}
\caption
\label{fig:short_path_favour}
\end{figure}


Problem med Alexander Huges: han er fotballspiller og dette kommer ikke så godt fra. 

asd.,kas.kdfj

\end{comment}

\begin{figure}[h]
\centering
\begin{lstlisting}
alexander hughes:
*people/people categories by parameter/people by time/births by year/year of birth missing (28.200766)
*nature/life/births by year/year of birth missing (28.576777)
*health/health by city/health in edinburgh/sport in edinburgh/sports teams in edinburgh/football clubs in edinburgh/heart of midlothian f.c./heart of midlothian f.c. players (37.22501)
\end{lstlisting}
\caption[Grading favouring short paths]{Example of how the grading based in inlink and outlink numbers favours short paths.}
\label{fig:short_path_favour}
\end{figure}
