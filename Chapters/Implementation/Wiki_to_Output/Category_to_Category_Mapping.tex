%\subsection{Mapping Wikipedia Categories to Desirable Output Categories}
\subsection{Deciding Output Categories based on Wikipedia Categories}
%The first step is to decide the output categories. %in other word what categories
The first approach was to create a mapping between each Wikipedia category and one or more categories in the desirable output category set. The idea was that a matching could be performed by matching Wikipedia category names and a output category name. The task of mapping each Wikiepedia category to desirable output categories is too big to be done manually since the Wikipedia category set contains 1 201 373 categories. This means that the process should be automated. One way of doing this is by looking at similarities in the words contained in the Wikipedia category and in the output category.

\subsubsection{Expanding the IAB category}
The categories in the IAB taxonomy were chosen as the desirable output category set for our task. This taxonomy  only consists of two category layers, which are not specified enough for creating a matching based on the category names. Hence the IAB taxonomy was extended with a third and more specified layer to improve the category mapping process. 

%was added to the taxonomy to 
%had to be created for this task, where this layer is specified 
%we want to classify all the Wikipedia categories to. The categories in the IAB taxonomy is not specified enough to categorize all the categories, hence it is necessary to add a third layer to the taxonomy. This layer has to be more specified to make it easier to categorize all the Wikipedia categories. 
%The next layer has to be modified to 

%fit the set of Wikipedia categories, and to be helpful for categorizing correct. 

This third layer can be viewed as common knowledge given to the machine. The second layer \emph{Europe} is an example of a layer where the machine lacks common knowledge since it does not know what countries are part of Europe. Expansion of this tier could be creating a third tier containing all European countries, which means that all Wikipedia categories containing a name of an European country should map to the category \emph{Europe}.

%An example of an expansion to the second layer \emph{Europe} is to add all European countries to its third layer since the machine lacks common knowledge about what countries 

%Such a layer can be viewed as giving the machine common knowledge. An example of 

%One of the second layers in the IAB taxonomy is \emph{Europe} under the first layer \emph{Travel}. The computer lacks common knowledge about what countries are in Europe, hence some information has to be provided to this layer so it can recognize countries in Europe. One way of doing this is by adding all European countries to a third layer under the category \emph{Europe}. 

%Since the task is categorization of Wikipedia, knowledge has been provided from other sources. List of all countries where found from \url{http://www.internetworldstats.com/list1.htm}. 

\subsubsection{Lemmatization}
%The set cof Wikipedia categories contains of 
%Our set of Wikipedia categories contains 1 201 373 categories. 

Figure \ref{fig:catmapping_exactmatch} shows how a matching between Wikipedia categories and output categories, where the output category name \emph{sports} are found as a word in the Wikipedia category name \emph{ministry of yougth affaris and sports}.

\begin{figure}[h]
\centering
\begin{lstlisting}
ministry of youth affairs and sports
sports
\end{lstlisting}
\caption[Exact match on mapping between Wikipedia category and output category]{Exact match on mapping between Wikipedia category and output category, where the output category is found in the Wikipedia category.}
\label{fig:catmapping_exactmatch}
\end{figure}
The problem with this approach is that words like \emph{sport} will not be an exact match of the word \emph{sports}, hence this Wikipedia category will not be included under the desirable output category. The next step is therefore to find matches between the categories regardless of the declension of the word. This part is called lemmatization and is defined as the process where different inflected forms of a word are grouped together \cite{wiki:lemmatisation}\cite[p.~30-33]{iirbook}. There are various lists for lemmatization available online, and a list was chosen from  \url{http://www.lexiconista.com/datasets/lemmatization/} which provided a list of common lemmatization. Both the words in the Wikipedia categories and the desirable output categories were processed by reading the lemmatization file and checked whether the words could be reduced. Figure \ref{fig:catmapping_lemmamatch} shows example of a match found after lemmatization is performed. 

\begin{figure}[h]
\centering
\begin{lstlisting}
sailors at the 1956 summer olympics
*olympics
*sailing
\end{lstlisting}
\caption[Example of match after lemmatization]{Example of a match between Wikipedia category and output category after lemmatization, where \emph{sailors} match with \emph{sailing}}
\label{fig:catmapping_lemmamatch}
\end{figure}

\begin{comment}
\subsubsection{Categories not relevant for classification}
Not all categories are suitable for classification, some categories are still just relevant for maintaining a well-structured encyclopedia. Example of such categories are \emph{container categories}, which are categories only containing subcategories. All container categories where found by looking at the file asdfasdf  . Some of these categories have already been removed because they are also hidden categories, but a total of 69 023 categories could be disregarded for this purpose. 
%Lots of categories are associated with years, but not 
%The next step was to mark all categories associated with years. These categories usually are  
\end{comment}

\subsubsection{Evaluation of Mapping Wikipedia Categories to Output Categories}
The results from this approach were not so good for two main reasons. 
%The results from this approach from this approach were not good for two main reasons: 
The first reason is that it is difficult to perform matching based on words. A perfect result could only be achieved if the computer knows all synonyms, inflections and the true meaning of all words. 

The other problem was with ambiguous words in the category names. An example of this is the categories shown in figure \ref{fig:ambiguous_category_name} where both categories contain the word \emph{Cicero}, but where the first category is for the suburb of Illinois and the other is for the Roman philosopher. Creating mapping rules for these names would be a difficult task. 

\begin{figure}[h]
\centering
\begin{lstlisting}
Category:Cicero, Illinois
Category:Cicero
\end{lstlisting}
\caption{Example of two category names which contains the same word with different meaning, and should be classified to different categories.}
\label{fig:ambiguous_category_name}
\end{figure}


The conclusion for this approach is that it might be possible to create a mapping between each Wikipedia Category and one or more desirable output categories, but this would need a very specified third tier in the IAB category and lots of rules. The task would therefore resemble a manual classification and is not a good approach. 
%-> It is impossible to create a third tier to satisfy this. 

