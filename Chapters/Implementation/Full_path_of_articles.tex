\section{Finding Full Path of Articles}
The goal of our implementation was to create a dictionary where the entries are created from the titles of Wikipedia articles, and each entry leads to one or more describing categories. Wikipedia already contains an underlying category structure which is useful to decide the relevant categories for each article. The first step was therefore to find the full path of each article in Wikipedia, where the path is given from the categories that lead to every article. 


% Creating a representation of the underlying structure. 

\subsection{Creating the Underlying Category Structure}
Finding full paths of all articles require information about Wikipedia's structure between categories, and between categories and articles. Thus, the first step was to create a way of representing the available information about the Wikipedia structure. % so the relevant information could be extracted. 
The file \enwikicatlink contains the information needed to create a database table \emph{categorylinks} filled with all links between categories, all links between articles and files, and all links between categories and articles. All information about the links are inserted in the database table through \texttt{INSERT} statements where all entries are on the form\\\\
(\emph{cl\_from},\emph{cl\_to},\emph{cl\_sortkey},\emph{cl\_timestamp},\emph{cl\_sortkey\_prefix},\emph{cl\_collation},\emph{cl\_type})\\\\
    Table \ref{tab:insertdescription} describes the meaning of all the \texttt{INSERT} statement fields \cite{wiki:categorylinkstable} and \ref{fig:entryexample} shows an example of an entry in the \texttt{INSERT} statement, where the link between the category with id \emph{12} and the  page \emph{Anarchism} is inserted into the table \emph{categorylinks}.  


\begin{figure}[h]
\centering
\begin{lstlisting}
(12,'Anarchism',' \nANARCHISM','2014-11-20 17:57:05',
' ','uppercase','page')
\end{lstlisting}
\caption[\texttt{INSERT} statement entry in \texttt{enwiki-latest-categorylinks.sql.gz}]{Example of an \texttt{INSERT} statement entry in \enwikicatlink. The link is between the category with id \emph{12} and the page \emph{Anarchism}.}
\label{fig:entryexample}
\end{figure}


\begin{table}[ht]
\renewcommand{\arraystretch}{1.25}
\begin{tabularx}{\textwidth}{l|X}
\textbf{Entry field} &  \textbf{Description} \\ \hline
cl\_from & Stores the page.page\_id of the article where the link was placed. \\ \hline
cl\_to & Stores the name (excluding namespace prefix) of the desired category \\  \hline
cl\_sortkey & Stores the title by which the page should be sorted in a category list. \\ \hline
cl\_timestamp & Stores the time at which that link was last updated in the table. \\ \hline
cl\_sortkey\_prefix & Empty string if a page is using the default sortkey or readable version of cl\_sortkey. \\ \hline
cl\_collation & What collation is in used. \\ \hline
cl\_type & What type of article is this (file, subcat (subcategory) or page (normal page)).
\end{tabularx}
\\[10pt]
\caption[Description of entry fields in \emph{Categorylinks}]{Description of entry fields in \texttt{INSERT} statements in \emph{Categorylinks}.}
\label{tab:insertdescription}
\end{table}


\begin{figure}[ht]
\begin{lstlisting}
INSERT INTO `categorylinks` VALUES 
(0,'','','2014-01-16 15:23:19','','','page'),
(10,'Redirects_from_moves','ACCESSIBLECOMPUTING','2014-10-26 04:50:23','','uppercase','page'),
(10,'Redirects_with_old_history','ACCESSIBLECOMPUTING','2010-08-26 22:38:36','','uppercase','page'),
(10,'Unprintworthy_redirects','ACCESSIBLECOMPUTING','2010-08-26 22:38:36','','uppercase','page'),
(12,'Anarchism',' \nANARCHISM','2014-11-20 17:57:05',' ','uppercase','page')
\end{lstlisting}
\caption[Excerpt from \texttt{enwiki-latest-categorylinks.sql.gz}]{Excerpt from the file \texttt{enwiki-latest-categorylinks.sql.gz} where each \texttt{INSERT} statement contains many links.}
\label{fig:categorylinks}
\end{figure}

Each \texttt{INSERT} statement consists of multiple links for insertion in the database table as we can see in figure \ref{fig:categorylinks}. The \texttt{INSERT} statements has to be split so that all links are separated into new statements, and only relevant links are kept, i.e.,  links with \emph{cl\_type} as \emph{subcat} (link between categories) or \emph{page} (link between category and article). The file \enwikicatlink contains 10 938 \texttt{INSERT} statements, which in total consists of 88172914 links. Table \ref{tab:enwikicatlinks} shows the number of different links found in the file. 

\begin{table}[ht]
\renewcommand{\arraystretch}{1.25}
\begin{tabularx}{\textwidth}{l|c|c|c} %X}
\textbf{Betweeen...} & \textbf{categories} &  \textbf{categories and articles} & \textbf{articles and files} \\\hline
 & 1 648 873 & 21 846 996 & 5 262 146
\end{tabularx}
\\[10pt]
\caption{Number of links found within the different link types.}
\label{tab:enwikicatlinks}
\end{table}

%A total number of #### \texttt{INSERT} statements were found in the \enwikicatlink 

%It is clear that there are lots of links within As table \ref{tab:enwikicatlinks} 

Table \ref{tab:enwikicatlinks} shows that there are lots of links between categories and between categories and articles. Some of these links are connected to categories for maintaining the encyclopedia. These categories are called \emph{hidden categories} and should be removed to reduce the complexity.
