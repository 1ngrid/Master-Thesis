\section{Full path of articles}
The goal of our implementation was to create a dictionary where the entries are created from the titles of Wikipedia articles, and each entry leads to one or more describing categories. Wikipedia already contains an underlying category structure which is useful to decide the relevant categories for each article. The first step was therefore to find the full path of each article in Wikipedia, where the path is given from the categories that lead to every article. 

\subsection{Accessing Information from Wikipedia}
There are two ways of accessing Wikipedia’s encyclopedic information; the most common way is to enter the webpage and search for the information needed, but it is also possible to download database dumps and access them directly to find information. All Wikipedia articles, images and categories are stored in a database which is accessed whenever a user search for information online, and the information retrieved from the database is returned to the webpage for example in the form of an article. To ensure that all data is safe at all times, files containing the information needed to recover the database is stored and regularly updated.\cite{wiki:databasedownload} This type of backup is called a database dump and is available for anyone interested at \url{http://dumps.wikimedia.org/enwiki/}.
%When a user search for information on the webpage this database is accessed. 
%which are accessed when a user are searching for an article online. 
%A database dump is therefore a backup of the database, and usually stored in the case of some data is lost\footnote{TODO Insert some link here. }. This backup is available for anyone interested at \footnote{TODO:insert link}. 

The files associated with the database dumps contains different information needed i.e., some files contains all the articles' titles, some contains information about which images belong to which articles and so on. Together they provide all information needed to restore Wikipedia if data is lost. 
%Just a few of these files where relevant for our task, the information needed was links between categories, between categories and articles and some special information about page properties. 
As mentioned, the first step is to find the full path of all articles. Since the Wikipedia articles are placed under categories describing their content, the full path of each article can be found by following the links between categories until an article is found. Table \ref{tab:databasedumpfiles} shows the files determined to be relevant for our task and a short description on what they contain. 


%This depends on creating a way to represent the structure of the categories and the articles. 

%and we can therefore define an article's path as the way to reach it from a given category. 
%The first step towards classification of Wiipedia articles is to find all full paths for the articles. There will be more than one way to reach many of the articles. 

%\begin{code}
%[INSERT EXAMPLE]
%\end{code}

%This task depends on different files from Wikipedia and should be split into smaller steps, hence several programs were made to complete the first task. 

%Several files were needed for the task, and the files depended on the language chosen. English is the language with most articles in Wikipedia, hence English were chosen and the 

%The files needed for this task we

\begin{table}[ht]
\renewcommand{\arraystretch}{1.25}
\begin{tabularx}{\textwidth}{l|X}
\textbf{File name} & \textbf{Information contained}\\ \hline
\texttt{enwiki-latest-categorgylinks.sql.gz} &  Containing information about links between categories, and between categories and articles. \\ \hline
\texttt{enwiki-latest-page.sql.gz} & Containing information about all pages in Wikipedia, including the type of page (category, article, user) and whether the page is a redirecting page or not\\ \hline
\texttt{enwiki-latest-page\_props.sql.gz} & Containing information about the properties of each page, including if the category is a hidden category or if the page a disambiguation page.
\end{tabularx}
\\[10pt]
\caption[Relevant files from Wikipedia database dump]{The relevant files from the Wikipedia database dump and a short description on what they contain}
\label{tab:databasedumpfiles}
\end{table}

%\begin{table}[h]
%{\renewcommand\arraystretch{1.25}
%\begin{tabular}{|l|l|l|} 
%\textbf{File name} & \textbf{Information contained} \\% \multicolumn{2}{l|}{Opret Server} \\ \hline
%\texttt{enwiki-latest-categorylinks.sql.gz }&  \multicolumn{2}{p{4cm}|}{\raggedright At oprette en server med bestemte regler som %tillader folk at spille sammen. More Text more text More Text} 
%\end{tabular}}


%\textbf{File name} & \textbf{Information contained} \\ \hline
%\texttt{enwiki-latest-categorylinks.sql.gz} & Containing information about links between categories, and between categories and articles. \\ 
%\texttt{enwiki-latest-page.sql.gz} & lkajsdf
%\end{tabular}
%s\caption{Caption}
%\label{tab:my_label}
%\end{table}

%\begin{itemize}
%\item \enwikicatlink
%\item[] Containing links between categories, and categories and articles which is needed to create a structure.
%\item \enwikipage
%\item[] Containing information about all pages in Wikipedia, including the type of page (category, article, user) and whether the page %is a redirecting page or not.
%\item \enwikipageprops
%\item[] Containing more information about the properties of each page, including if the category is a hidden category of if the page is a disambiguation page. 
%\end{itemize}

%All of these files are downloaded from \path{http://dumps.wikimedia.org/enwiki/} 

% Creating a representation of the underlying structure. 
%\section{Parsing through the dumps}

\subsection{Creating the Underlying Category Structure}
Finding full paths of all articles require information about Wikipedia's structure between categories and articles. The first step was therefore to create a structure to represent this information. The file \texttt{enwiki-latest-categorylinks.sql.gz} contains the 
%information needed for this task. It consists of the code 
information needed to create a database table \emph{categorylinks} and then insert the information about links between categories, between articles and files, and between categories and articles into the table with through \texttt{INSERT} statements. %All this information is stored in the table with \texttt{INSERT} statements. 
This means that all information about the relationships can be extracted from the \texttt{INSERT} statements, and the program \texttt{Split\_catlink.py} was created for this task. The main purpose of the program is to sort the different types of links so that links between categories are saved to one file and links between categories and articles are stored to another file. 

The structure of the \texttt{enwiki-latest-categorylinks.sql.gz} is build
%to make sure 
so each \texttt{INSERT} statement represents many links as Figure \ref{fig:categorylinks} show. This means that the first step is to sort out the relevant information about each link, which is the type of link (given as \emph{page}, \emph{subcat} or \emph{file}), what the link links from (ex: \emph{Redirects\_from\_moves} and what the link links to (ex: \emph{ACCESSIBLECOMPUTING}). 

%[caption={Excerpt from the file \texttt{enwiki-latest-categorylinks.sql.gz} where each \texttt{INSERT} statement contains many links}, label={code:categorylinks}]
\begin{figure}[ht]
\begin{lstlisting}
INSERT INTO `categorylinks` VALUES 
(0,'','','2014-01-16 15:23:19','','','page'),
(10,'Redirects_from_moves','ACCESSIBLECOMPUTING','2014-10-26 04:50:23','','uppercase','page'),
(10,'Redirects_with_old_history','ACCESSIBLECOMPUTING','2010-08-26 22:38:36','','uppercase','page'),
(10,'Unprintworthy_redirects','ACCESSIBLECOMPUTING','2010-08-26 22:38:36','','uppercase','page'),
(12,'Anarchism',' \nANARCHISM','2014-11-20 17:57:05',' ','uppercase','page')
\end{lstlisting}
\caption[Excerpt from \texttt{enwiki-latest-categorylinks.sql.gz}]{Excerpt from the file \texttt{enwiki-latest-categorylinks.sql.gz} where each \texttt{INSERT} statement contains many links}
\label{fig:categorylinks}
\end{figure}

Our task is only interested in links between categories (marked as \emph{subcat}) and links between category and article (marked as \emph{page}). Wikipedia contains lots of underlying structure to help the volunteering editors and this structure is not relevant for our task. 


%HER!! 
The first step was therefore to skip all categories used for editing or maintaining the encyclopedia. 

\begin{comment}

i.e., \emph{articles}, \emph{redirects}, \emph{wikidata} etc. Table \ref{tab:withoutwords} shows how the numbers of links are reduced when these words are removed, but these words are not the only way to removed irrelevant links. 
 
\begin{table}[ht]
\centering
\begin{tabular}{l|c|c}
\textbf{Links between...} & \textbf{W the words} & \textbf{W/o the words}  \\ \hline
 \textbf{subcategories} & 3.341.524 & 2.826.815  \\
 \textbf{articles and categories} & 78.535.744 & 27.804.524
\end{tabular}
\caption[Number of links without words common for editing]{Number of links when common words for editing are removed}
\label{tab:withoutwords}
\end{table}

\end{comment}