\section{Id Mapping}
The files containing the results becomes extremely large due to the size of the results. When writing all the results to file, the files becomes extremely large. All paths of all Wikipedia articles is more than 20 GB of compressed data. It is desirable to reduce the space needed for storing all results on the computer. The solution was to create an id mapping for each category name and article name. Id mapping gives all names a unique id, and instead of writing the full path of category names to the file, the full paths with category ids is written to file. 

The id mapping is implemented by creating a counter that assigns numbers to each category name or article name that is not found yet, i.e., a unique number represents each name. Figure \ref{fig:idmapping} shows an excerpt of the id mapping created for our purpose, where the id \emph{4600570} corresponds to the article about \emph{Ole-Johan Dahl}, which means that this id is used everywhere \emph{Ole-Johan Dahl} is used in paths. 

\begin{figure}[h]
\centering
\begin{lstlisting}
...
4600566 roger matthews
4600567 pesticide drift
4600568 roxy theatre (clarksville, tennessee)
4600569 papadindar
4600570 ole-johan dahl
4600571 red square (university of washington)
...
\end{lstlisting}
\caption[Id mapping example]{Excerpt of the id mapping between id and the name of all categories and articles.}
\label{fig:idmapping}
\end{figure}

Id mapping is storage efficient because category names and article names usually are a  lot longer than their representing ids. 

Working with ids is also faster in many implementations concerning lookups in the program. This depends on the structure chosen for the programs, but when using dictionaries as done in our implementation, ids will perform faster than if using full names. An example of this can be seen in figure \ref{fig:id_lookup} where the time to find all categories from the category with id 177678 (corresponding to the category \emph{people}) is 0.955 minutes. Figure \ref{fig:fullname_lookup} shows the time needed to find the same paths for the category when using full names for categories and articles, which is found to be 1.559 minutes. Comparing the times shows that the time is a lot faster when using ids, which is important when many paths have to found.

\begin{figure}[h]
\centering
\begin{lstlisting}
[INFO] Finding all article paths from 177678

[INFO] Time to find all article paths: 0.955 min
\end{lstlisting}
\caption[Time for all paths for \emph{people} when using ids]{Time needed for finding all paths from the category 177678 (corresponding to the category \emph{people}) when ids are used in the program.}
\label{fig:id_lookup}
\end{figure}


\begin{figure}[h]
\centering
\begin{lstlisting}
[INFO] Finding all article paths from people

[INFO] Time to find all article paths: 1.559 min
\end{lstlisting}
\caption[Time for all paths for \emph{people} when using full names]{Time needed for finding all paths from the category people when using full names).}
\label{fig:fullname_lookup}
\end{figure}
 
The last reason to use ids instead of full names is that the full names may include characters useful for describing paths, for instance the characters "/" which is a common way of describing full paths. 

% Fordel 2: Kan bruke "/" in the text. 

