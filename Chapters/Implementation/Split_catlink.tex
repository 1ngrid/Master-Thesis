% Creating a representation of the underlying structure. 
%\section{Parsing through the dumps}

\subsection{Creating the Underlying Category Structure}
Finding full paths of all articles require information about Wikipedia's structure between categories and articles. The first step was therefore to create a structure to represent this information. The file \texttt{enwiki-latest-categorylinks.sql.gz} contains the 
%information needed for this task. It consists of the code 
information needed to create a database table \emph{categorylinks} and then insert the information about links between categories, between articles and files, and between categories and articles into the table with through \texttt{INSERT} statements. %All this information is stored in the table with \texttt{INSERT} statements. 
This means that all information about the relationships can be extracted from the \texttt{INSERT} statements, and the program \texttt{Split\_catlink.py} was created for this task. The main purpose of the program is to sort the different types of links so that links between categories are saved to one file and links between categories and articles are stored to another file. 

The structure of the \texttt{enwiki-latest-categorylinks.sql.gz} is build
%to make sure 
so each \texttt{INSERT} statement represents many links as Figure \ref{fig:categorylinks} show. This means that the first step is to sort out the relevant information about each link, which is the type of link (given as \emph{page}, \emph{subcat} or \emph{file}), what the link links from (ex: \emph{Redirects\_from\_moves} and what the link links to (ex: \emph{ACCESSIBLECOMPUTING}). 

%[caption={Excerpt from the file \texttt{enwiki-latest-categorylinks.sql.gz} where each \texttt{INSERT} statement contains many links}, label={code:categorylinks}]
\begin{figure}[ht]
\begin{lstlisting}
INSERT INTO `categorylinks` VALUES 
(0,'','','2014-01-16 15:23:19','','','page'),
(10,'Redirects_from_moves','ACCESSIBLECOMPUTING','2014-10-26 04:50:23','','uppercase','page'),
(10,'Redirects_with_old_history','ACCESSIBLECOMPUTING','2010-08-26 22:38:36','','uppercase','page'),
(10,'Unprintworthy_redirects','ACCESSIBLECOMPUTING','2010-08-26 22:38:36','','uppercase','page'),
(12,'Anarchism',' \nANARCHISM','2014-11-20 17:57:05',' ','uppercase','page')
\end{lstlisting}
\caption[Excerpt from \texttt{enwiki-latest-categorylinks.sql.gz}]{Excerpt from the file \texttt{enwiki-latest-categorylinks.sql.gz} where each \texttt{INSERT} statement contains many links}
\label{fig:categorylinks}
\end{figure}

Our task is only interested in links between categories (marked as \emph{subcat}) and links between category and article (marked as \emph{page}). Wikipedia contains lots of underlying structure to help the volunteering editors and this structure is not relevant for our task. 


%HER!! 
The first step was therefore to skip all categories used for editing or maintaining the encyclopedia. 

\begin{comment}

i.e., \emph{articles}, \emph{redirects}, \emph{wikidata} etc. Table \ref{tab:withoutwords} shows how the numbers of links are reduced when these words are removed, but these words are not the only way to removed irrelevant links. 
 
\begin{table}[ht]
\centering
\begin{tabular}{l|c|c}
\textbf{Links between...} & \textbf{W the words} & \textbf{W/o the words}  \\ \hline
 \textbf{subcategories} & 3.341.524 & 2.826.815  \\
 \textbf{articles and categories} & 78.535.744 & 27.804.524
\end{tabular}
\caption[Number of links without words common for editing]{Number of links when common words for editing are removed}
\label{tab:withoutwords}
\end{table}

\end{comment}