\section{Other Languages}
We have argued that one of the main advantages with Wikipedia is that it is a multilingual encyclopedia. This means that it is desirable to take advantage of this and create dictionary-based classifiers for other languages. It is desirable to take advantage of the work already performed for creating the dictionary-based classifier for the English Wikipedia, especially since this this is the most common language.


Many Wikipedia articles are available in multiple languages. The file \enwikilanglinks contains information about the language links for the English Wikipedia. All links are represented as entries in \texttt{INSERT} statements on the form  
\begin{center}
(\emph{il\_from}, \emph{il\_lang}, \emph{il\_title}).
\end{center}

Table \ref{tab:langlinkdesc} contains the description of all the \texttt{INSERT} statements  \cite{wiki:langlinks}, and \ref{fig:langlinkexample} is an example of entries translating English articles to French. This means that it is possible to find all links from English and to another language by finding the language's \emph{language code}. 

\begin{table}[ht]
\renewcommand{\arraystretch}{1.25}
\begin{tabularx}{\textwidth}{l|X}
\textbf{Entry field} &  \textbf{Description} \\ \hline
 il\_from & page\_id of the referring page.\\ \hline
 il\_lang & Language code of the target, in the ISO 639-1 standard. \\  \hline
 il\_title & Title of the target, including namespace 
 (FULLPAGENAMEE style).
\end{tabularx}
\\[10pt]
\caption[Description of the entry fields in the table \emph{Langlink}]{Description of the entries in the table \emph{Langlink}.}
\label{tab:langlinkdesc}
\end{table}

\begin{figure}[h]
\centering
\begin{lstlisting}
INSERT INTO `langlinks` VALUES 
(10642344,'fr','Muro de Aguas'),
(1666460,'fr','Muro de Alcoy'),
(32877065,'fr','Muro en Cameros')
\end{lstlisting}
\caption[Example of langlink \texttt{INSERT} statement]{Example of entries for linking the English ids to the corresponding French articles.}
\label{fig:langlinkexample}
\end{figure}
% INSERT INTO `langlinks` VALUES (865173,'fr','Liste des stations de radio en Autriche'),(6922017,'fr','Liste des stations de radio en Belgique'),(3560441,'fr','Liste des stations de radio en Bielorussie'),
%INSERT INTO `langlinks` VALUES (24190880,'fr','Daniel Zimmermann (personnalité politique)'),(10957102,'fr','Daniel Zueras'),(
\subsection{Creating a Norwegian Dictionary-based Classifier}
We chose to create Norwegian dictionary-based classifier to test out the idea in real life. The main reason to choose Norwegian is that the results can be manually evaluated since we are familiar with Norwegian.

To see if this idea worked, we created a Norwegian dictionary-based classifier based on the English list of entries. 
 

%Since Wikipedia is available in multiple languages, we can take advantage of this and easily reuse the categorization results found 



First I found all links to norwegian pages by finding all insertions marked with \emph{no}. Then all page-ids are stored together with the title of the norwegian article. 
 
('2968022', 'no', 'Barbados under Sommer-OL 2000')
 
\begin{figure}[h]
\centering
\begin{lstlisting}
378466      Fidel V. Ramos
287145      Jacques Cazotte
24984364    Gnathothlibus
287149      Nyala
2653965     Papuacedrusslekten
370256      Timurid-dynastiet
34417918    Kategori:Personer fra provinsen Valencia
33931676    Druide
778284      Flora (gudinne)
4369469     Aloandia
1246681     Edge (magasin)
5980        Karbonsluk
1902206     Nansenprisen
\end{lstlisting}
\caption{Caption}
\label{fig:my_label}
\end{figure}
 
Then I needed to find the corresponding English article name for the page id, which was done by looping through relevant pages in the \enwikicatlink file since all entries are on the form: 
 
\begin{figure}[h]
\centering
\begin{lstlisting}
INSERT INTO `categorylinks` VALUES (0,'','','2014-01-16 15:23:19','','','page'),(10,'Redirects_from_moves','ACCESSIBLECOMPUTING','2014-10-26 04:50:23','','uppercase','page')
\end{lstlisting}
\caption{Caption}
\label{fig:my_label}
\end{figure}
 
id 10 equals accesiblecomputing. 

\begin{figure}[h]
\centering
\begin{lstlisting}
bicycle kick        brassespark
davis phinney       davis phinney
phasi charoen       district  phasi charoen
chanthaburi         chanthaburi
hindnubben          hindnubben
kamrup district     kamrup (distrikt)
\end{lstlisting}
\caption{Caption}
\label{fig:my_label}
\end{figure}

At the end I loaded the finished dictionary and compared the names and created a large dictionary with the english names where there were norwegian titles available and ended with a dictionary like this: 

\begin{figure}[h]
\centering
\begin{lstlisting}
"speilreflekskamera": [
      "technology & computing/cameras & camcorders"
    ], 
    "henry hermansen": [
      "sports/skiing"
    ], 
    "joost wichman": [
      "sports/mountain biking"
    ], 
    "punt e mes": [
      "food & drinks/wine"
    ], 
    "mikroorganisme": [
      "science/biology"
    ], 
\end{lstlisting}
\caption{Caption}
\label{fig:my_label}
\end{figure}
