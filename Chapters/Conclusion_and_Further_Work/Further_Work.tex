\section{Further Works}
There exists many desirable extensions for our dictionary-based classifier that might improve the results or expand the usage. The most important future improvements for our classifier is solving ambiguity in a better way and extending the dictionary by exploring more than just the titles of Wikipedia articles. Expanding the usage is possible if the classifier is well-defined for other languages than just Wikipedia. 

\subsubsection{Disambiguation}
Our project removed all ambiguous titles. This means that we loose information that might be valuable for classification purposes. Instead of removing the titles, we could keep the titles that are most relevant for our classification, for instance the titles that have the longest articles. We studied different projects for solving disambiguation, and some of their findings could be applied to find the most likely meaning of an ambiguous entry. 

\subsubsection{Stubs}
Another possible change for the program is to remove Wikipedia stubs. Wikipedia stubs are pages that are too short to be considered articles. Wikipedia contains 1 913 507 stubs \cite{wiki:allstubs}, and these articles might provide ambiguous information which are more likely to be removed since they contain so little information that they are not considered articles.  


\subsubsection{Explore more information from Wikipedia}
We have only looked at the titles when creating the dictionary-based classifier. Another extension would be to explore the actual content of the articles before they are classified to the most describing categories. Better categorization of the Wikipedia articles could lead to better results for the classifier, which could improve the results. 

\subsubsection{Adding information}
We have only used information from Wikipedia, but it might be desirable to extend our classifier with information in addition to Wikipedia. Keywords from other sources might improve the results like it did in \cite{entityextraction} when adding information from MusicBrainz, City DB, Yahoo! Stocks, Chrome and Adam. 

\subsubsection{Improve Mapping}
The mapping between keywords and categories could also be improved by creating better decision rules between category paths and IAB categories. We chose to grade our Wikipedia article titles by using inlink and outlink number. Another improvement could be to try other grading algorithms, which might be better for determining the content of the articles. 
%It could also be improved by trying other grading algorithms. 

\subsubsection{Extending for more languages}
The results and implementation is created for English Wikipedia, hence only useful for English articles. We created a Norwegian dictionary-based classifier by using the internal Wikipedia links to translate the English classifier's dictionary to Norwegian. We noticed that the Norwegian classifier lacked important information for being able to classify Norwegian articles, and concluded that this is probably because special Norwegian keywords are missing from the dictionary. 

Thus, a desirable extension would be to create a more general approach for creating the dictionary so that it could be applied to other languages as well. Most of our programs are not dependent on language, except for the mapping rules. A good extension would be to create a language independent mapping process which could create dictionary-based classifiers from Wikipedia in multiple languages. 


%\subsection{Automatic mapping from Wikipedia titles}
%The mapping process from Wikipedia and to desirable output is a 

